% SIAM Supplemental File Template
\documentclass[review,supplement,hidelinks,onefignum,onetabnum]{siamart220329}

% SIAM Shared Information Template
% This is information that is shared between the main document and any
% supplement. If no supplement is required, then this information can
% be included directly in the main document.


% Packages and macros go here
\usepackage{lipsum}
\usepackage{amsfonts}
\usepackage{graphicx}
\usepackage{epstopdf}
\usepackage{algorithmic}
\ifpdf
  \DeclareGraphicsExtensions{.eps,.pdf,.png,.jpg}
\else
  \DeclareGraphicsExtensions{.eps}
\fi

% Add a serial/Oxford comma by default.
\newcommand{\creflastconjunction}{, and~}

% Used for creating new theorem and remark environments
\newsiamremark{remark}{Remark}
\newsiamremark{hypothesis}{Hypothesis}
\crefname{hypothesis}{Hypothesis}{Hypotheses}
\newsiamthm{claim}{Claim}

% Sets running headers as well as PDF title and authors
\headers{A second order numerical methods for Reisz-Fractional Elliptic Equation on graded mesh }{Jianxing Han and Minghua Chen}

% Title. If the supplement option is on, then "Supplementary Material"
% is automatically inserted before the title.
\title{second-order error analysis for Fractional Laplacian via Riesz Derivatives on graded meshes\thanks{Submitted to the editors DATE.
}}

% Authors: full names plus addresses.
\author{Jianxing Han\thanks{School of Mathematics and Statistics, Lanzhou University, Lanzhou 730000, PR China
  (\email{hanjx2023@mail.lzu.edu.cn}).}
  \and Minghua Chen\thanks{School of Mathematics and Statistics, Lanzhou University, Lanzhou 730000, PR China 
  (\email{chen@mail.lzu.edu.cn}).}
}

\usepackage{amsopn}
\DeclareMathOperator{\diag}{diag}


%%% Local Variables: 
%%% mode:latex
%%% TeX-master: "ex_article"
%%% End: 


\externaldocument[][nocite]{ex_article}

% Optional PDF information
\ifpdf
\hypersetup{
  pdftitle={Supplementary Materials: An Example Article},
  pdfauthor={D. Doe, P. T. Frank, and J. E. Smith}
}
\fi

\begin{document}

\maketitle

\section{A detailed example}

Here we include some equations and theorem-like environments to show
how these are labeled in a supplement and can be referenced from the
main text.
Consider the following equation:
\begin{equation}
  \label{eq:suppa}
  a^2 + b^2 = c^2.
\end{equation}
You can also reference equations such as \cref{eq:matrices,eq:bb} 
from the main article in this supplement.

\lipsum[100-101]

\begin{theorem}
An example theorem.
\end{theorem}

\lipsum[102]
 
\begin{lemma}
An example lemma.
\end{lemma}

\lipsum[103-105]

Here is an example citation: \cite{KoMa14}.

\section[Proof of Thm]{Proof of \cref{thm:bigthm}}
\label{sec:proof}

\lipsum[106-112]

\section{Additional experimental results}
\Cref{tab:smfoo} shows additional
supporting evidence. 

\begin{table}[htbp]
\footnotesize
  \caption{Example table.}\label{tab:smfoo}
\begin{center}
  \begin{tabular}{|c|c|c|} \hline
   Species & \bf Mean & \bf Std.~Dev. \\ \hline
    1 & 3.4 & 1.2 \\
    2 & 5.4 & 0.6 \\ \hline
  \end{tabular}
\end{center}
\end{table}

\begin{proof}
  We use the similar skill in the last section, but more complicated.
  for \(j=N\),
  Let
  \begin{equation}
    y_{i\to N}(x) = \frac{x^{1/r} - Z_i}{Z_1} h_N + T, \quad Z_i = T^{1/r}\frac{i}{N}, x_N = T
  \end{equation}
  and
  \begin{equation}
    y_{i\to N-1}(x) = (x^{1/r} + Z_{N-1-i})^{r}, \quad Z_{N-1-i} = T^{1/r}\frac{N-1-i}{N}
  \end{equation}
  Thus,
  \begin{gather*}
    y_{i\to N}(x_{i-1}) = x_{N-1}, \quad  y_{i\to N}(x_{i}) = x_{N}, \quad y_{i\to N}(x_{i+1}) = x_{N+1} \\
    y_{i\to N-1}(x_{i-1}) = x_{N-2}, \quad  y_{i\to N}(x_{i}) = x_{N-1}, \quad y_{i\to N}(x_{i+1}) = x_{N}
  \end{gather*}
  Then, define
  \begin{equation}
    y_{i\to N}^\theta(x) = \theta y_{i\to N-1}(x) + (1-\theta) y_{i\to N}(x)
  \end{equation}
  \begin{equation}
    h_{i\to N}(x) = y_{i\to N}(x) - y_{i\to N-1}(x)
  \end{equation}
  We have
  \begin{gather}
    {y_{i\to N-1}}'(x) = y_{i\to N-1}^{1-1/r}(x) x^{1/r-1}  \\
    {y_{i\to N-1}}''(x) = \frac{1-r}{r} y_{i\to N-1}^{1-2/r}(x) x^{1/r-2} Z_{N-1-i}  \\
    {y_{i\to N}}'(x) = \frac{1}{r}\frac{h_N}{Z_1} x^{1/r-1}  \\
    {y_{i\to N}}'(x) = \frac{1-r}{r^2}\frac{h_N}{Z_1} x^{1/r-2}
  \end{gather}
  \begin{equation}
    P_{i\to N}^\theta(x) = (h_{i\to N}(x))^3 \frac{|y_{i\to N}^\theta(x)-x|^{1-\alpha}}{\Gamma(2-\alpha)} u''(y_{i\to N}^\theta(x)) 
  \end{equation}
  \begin{equation}
    Q_{i\to N}^\theta(x) = (h_{i\to N}(x))^4 \frac{|y_{i\to N}^\theta(x)-x|^{1-\alpha}}{\Gamma(2-\alpha)} 
  \end{equation}
  Similar with \cref{lmm:Tij-express-as-int-of-function}, we can get for \(l=-1,0,1\),
  \begin{equation}
    \begin{aligned}
      T_{i+l, N+l} & = \int_{0}^{1} -\frac{\theta (1-\theta)}{2} {P_{i\to N}^\theta}(x_{i+l}) d\theta                                                                                               \\
              & + \int_{0}^{1} \frac{\theta (1-\theta)}{3!}{Q_{i\to N}^\theta}(x_{i+l})( \theta^2  u'''(\eta_{N+l, 1}^\theta) - (1-\theta)^2 u'''(\eta_{N+l, 2}^\theta)) d\theta
    \end{aligned}
  \end{equation}
  So we have
  \begin{equation}
    \begin{aligned}
      V_{iN} & = \frac{2}{h_{i} + h_{i+1}}  \left( \frac{1}{h_{i+1}} T_{i+1, N+1} - \left(\frac{1}{h_{i}}+\frac{1}{h_{i+1}}\right)  T_{i,N} + \frac{1}{h_{i}} T_{i-1, N-1} \right)                                                      \\
             & = \int_{0}^{1} -\frac{\theta (1-\theta)}{2} D_h^2 P_{i\to N}^\theta(x_i)  d\theta                                                                                                                                           \\
             & \quad +  \int_{0}^1 \frac{\theta^3 (1-\theta)}{3!} \frac{2}{h_{i} + h_{i+1}}\left( \frac{{Q_{i\to N}^\theta}(x_{i+1}) u'''(\eta_{N+1,1}^\theta) - {Q_{i\to N}^\theta}(x_{i}) u'''(\eta_{N,1}^\theta)}{h_{i+1}}\right)  d\theta \\
             & \quad -  \int_{0}^1 \frac{\theta^3 (1-\theta)}{3!} \frac{2}{h_{i} + h_{i+1}}\left( \frac{{Q_{i\to N}^\theta}(x_{i}) u'''(\eta_{N,1}^\theta) - {Q_{i\to N}^\theta}(x_{i-1}) u'''(\eta_{N-1,1}^\theta)}{h_{i}}\right)  d\theta   \\
             & \quad -  \int_{0}^1 \frac{\theta (1-\theta)^3}{3!} \frac{2}{h_{i} + h_{i+1}}\left( \frac{{Q_{i\to N}^\theta}(x_{i+1}) u'''(\eta_{N+1,2}^\theta) - {Q_{i\to N}^\theta}(x_{i}) u'''(\eta_{N,2}^\theta)}{h_{i+1}}\right)  d\theta \\
             & \quad +  \int_{0}^1 \frac{\theta (1-\theta)^3}{3!} \frac{2}{h_{i} + h_{i+1}}\left( \frac{{Q_{i\to N}^\theta}(x_{i}) u'''(\eta_{N,2}^\theta) - {Q_{i\to N}^\theta}(x_{i-1}) u'''(\eta_{N-1,2}^\theta)}{h_{i}}\right)  d\theta
    \end{aligned}
  \end{equation}
  To estimate \(D_h^2 P_{i\to N}^\theta(x_i)={P_{i\to N}^\theta}''(\xi), \xi \in [x_{i-1}, x_{i+1}]\),
\end{proof}



\bibliographystyle{siamplain}
\bibliography{references}


\end{document}
