% SIAM Article Template
\documentclass[review,hidelinks,onefignum,onetabnum]{siamart220329}

% Information that is shared between the article and the supplement
% (title and author information, macros, packages, etc.) goes into
% ex_shared.tex. If there is no supplement, this file can be included
% directly.
\usepackage{amssymb}
% SIAM Shared Information Template
% This is information that is shared between the main document and any
% supplement. If no supplement is required, then this information can
% be included directly in the main document.


% Packages and macros go here
\usepackage{lipsum}
\usepackage{amsfonts}
\usepackage{graphicx}
\usepackage{epstopdf}
\usepackage{algorithmic}
\ifpdf
  \DeclareGraphicsExtensions{.eps,.pdf,.png,.jpg}
\else
  \DeclareGraphicsExtensions{.eps}
\fi

% Add a serial/Oxford comma by default.
\newcommand{\creflastconjunction}{, and~}

% Used for creating new theorem and remark environments
\newsiamremark{remark}{Remark}
\newsiamremark{hypothesis}{Hypothesis}
\crefname{hypothesis}{Hypothesis}{Hypotheses}
\newsiamthm{claim}{Claim}

% Sets running headers as well as PDF title and authors
\headers{A second order numerical methods for Reisz-Fractional Elliptic Equation on graded mesh }{Jianxing Han and Minghua Chen}

% Title. If the supplement option is on, then "Supplementary Material"
% is automatically inserted before the title.
\title{second-order error analysis for Fractional Laplacian via Riesz Derivatives on graded meshes\thanks{Submitted to the editors DATE.
}}

% Authors: full names plus addresses.
\author{Jianxing Han\thanks{School of Mathematics and Statistics, Lanzhou University, Lanzhou 730000, PR China
  (\email{hanjx2023@mail.lzu.edu.cn}).}
  \and Minghua Chen\thanks{School of Mathematics and Statistics, Lanzhou University, Lanzhou 730000, PR China 
  (\email{chen@mail.lzu.edu.cn}).}
}

\usepackage{amsopn}
\DeclareMathOperator{\diag}{diag}


%%% Local Variables: 
%%% mode:latex
%%% TeX-master: "ex_article"
%%% End: 


% Optional PDF information
\ifpdf
\hypersetup{
  pdftitle={Second-order numerical method for two-dimensional two-sided space fractional convection diffusion equation},
  pdfauthor={D. Doe, P. T. Frank, and J. E. Smith}
}
\fi

% The next statement enables references to information in the
% supplement. See the xr-hyperref package for details.

\externaldocument[][nocite]{ex_supplement}

% FundRef data to be entered by SIAM
%<funding-group specific-use="FundRef">
%<award-group>
%<funding-source>
%<named-content content-type="funder-name"> 
%</named-content> 
%<named-content content-type="funder-identifier"> 
%</named-content>
%</funding-source>
%<award-id> </award-id>
%</award-group>
%</funding-group>

\begin{document}

\maketitle

% REQUIRED
\begin{abstract}
  \textcolor{gray}{
    This is an example SIAM \LaTeX\ article. This can be used as a
    template for new articles.  Abstracts must be able to stand alone
    and so cannot contain citations to the paper's references,
    equations, etc.  An abstract must consist of a single paragraph and
    be concise. Because of online formatting, abstracts must appear as
    plain as possible. Any equations should be inline.
  }
\end{abstract}

% REQUIRED
\begin{keywords}
  example, \LaTeX
\end{keywords}

% REQUIRED
\begin{MSCcodes}
  ??????????????????
\end{MSCcodes}


% Chapter 1: Introduction
\section{Introduction}

For \(\Omega=(0,2T)\), \(1<\alpha<2\),
\begin{equation} \label{eq:equation}
  \begin{cases}
    (-\Delta)^{\frac{\alpha}{2}} u(x) = f(x), & x \in \Omega                      \\
    u(x) = 0,                                 & x \in \mathbb{R} \setminus \Omega
  \end{cases}
\end{equation}
where
\begin{equation} \label{def:operator}
  (-\Delta)^{\frac{\alpha}{2}} u(x) = -\frac{\partial^\alpha u}{\partial |x|^\alpha}
  = -\kappa_\alpha \frac{d^2}{dx^2} \int_\Omega \frac{|x-y|^{1-\alpha}}{\Gamma(2-\alpha)}u(y) dy
\end{equation}
\begin{equation} \label{def:kappa}
  \kappa_\alpha = -\frac{1}{2\cos(\alpha\pi/2)} > 0
\end{equation}






% Chapter 2:
\section{Preliminaries: Numeric scheme and main results}




% Section 2.1: Numerical scheme
\subsection{Numeric Format}
\label{sec:numformat}


\begin{equation} \label{def:xj}
  x_i = \begin{cases}
    T \left(\frac{i}{N}\right)^r   ,      & 0 \le i \le N  \\
    2T - T \left(\frac{2N-i}{N}\right)^r, & N \le i \le 2N
  \end{cases}
\end{equation}
where $r\ge 1$ .
And let
\begin{equation} \label{def:hj}
  h_j = x_{j} - x_{j-1}, \quad 1\le j \le 2N
\end{equation}

Let $\{\phi_j(x)\}_{j=1}^{2N-1}$ be standard hat functions, which are basis of the piecewise linear function space.
\begin{equation}
  \phi_j(x) = \begin{cases}
    \frac{1}{h_j} (x-x_{j-1}),      & x_{j-1} \le x \le x_{j} \\
    \frac{1}{h_{j+1}} (x_{j+1}-x) , & x_{j} \le x \le x_{j+1} \\
    0,                              & \text{otherwise}
  \end{cases}
\end{equation}
And then, define the piecewise linear interpolant of the true solution \(u\) to be
\begin{equation}
  \Pi_hu(x) := \sum_{j=1}^{2N-1} u(x_j) \phi_j(x)
\end{equation}

For convience, we denote
\begin{equation} \label{def:I2-a}
  I^{2-\alpha} u(x) := \frac{1}{\Gamma(2-\alpha)}\int_{\Omega} |x-y|^{1-\alpha} u(y) dy
\end{equation}
and
\begin{equation} \label{def:Dh2}
  D_h^2 u(x_i) := \frac{2}{h_i + h_{i+1}} \left( \frac{1}{h_i} u(x_{i-1}) - \left( \frac{1}{h_i} + \frac{1}{h_{i+1}} \right) u(x_i) + \frac{1}{h_{i+1}} u(x_{i+1}) \right)
\end{equation}

Now, we discretise \cref{eq:equation} by replacing \(u(x)\) by a continuous piecewise linear function
\begin{equation}
  u_h(x) := \sum_{j=1}^{2N-1} u_j \phi_j (x)
\end{equation}
whose nodal values \(u_j\) are to be determined by collocation at each mesh point \(x_i\) for \(i = 1, 2,...,2N - 1\):
\begin{equation} \label{def:discrete_equation}
  -\kappa_\alpha D_h^{\alpha} u_h(x_i) := -\kappa_\alpha D_h^2 I^{2-\alpha} u_h(x_i) = f(x_i) =: f_i
\end{equation}
Here,
\begin{equation}
  -\kappa_\alpha  D_h^{\alpha} u_h(x_i) 
  = \sum_{j=1}^{2N-1} -\kappa_\alpha  D_h^2 I^{2-\alpha} \phi_j(x_i)\; u_j 
  = \sum_{j=1}^{2N-1} a_{ij} \; u_j
\end{equation}
where
\begin{equation} \label{def:aij}
  a_{ij} = -\kappa_\alpha  D_h^2 I^{2-\alpha} \phi_j(x_i) \quad\text{for}\quad i,j = 1, 2,...,2N-1
\end{equation}

We have replaced \((-\Delta)^{\alpha/2} u(x_i ) = f (x_i )\) in \cref{eq:equation} by \(-\kappa_\alpha D_h^\alpha u_h(x_i ) = f (x_i )\) in \cref{def:discrete_equation}, 
with truncation error 
\begin{equation} \label{def:truncation_error}
  \tau_i := -\kappa_\alpha \left( D_h^{\alpha} \Pi_h u(x_i) - \frac{d^2}{dx^2}I^{2-\alpha} u(x_i) \right) \quad\text{for}\quad i = 1, 2,...,2N - 1
\end{equation}
where \(-\kappa_\alpha D_h^{\alpha} \Pi_h u(x_i) =  \sum_{j=1}^{2N-1} -\kappa_\alpha D_h^{\alpha} \phi_j(x_i) u(x_j) = \sum_{j=1}^{2N-1} a_{ij} u(x_j)\).

The discrete equation \eqref{def:discrete_equation} can be written in matrix form
\begin{equation} \label{eq:equation_matrix}
  AU = F
\end{equation}
where $A=(a_{ij}) \in \mathbb{R}^{(2N-1) \times (2N-1)}$, $U=(u_1, \cdots, u_{2N-1})^T$ is unknown and $F=(f_1, \cdots, f_{2N-1})^T$.

We can deduce \(a_{ij}\),
\begin{equation} \label{eq:aij}
  \begin{aligned}
    a_{ij} &= -\kappa_\alpha  D_h^2 I^{2-\alpha} \phi_j(x_i) \\
    &= -\kappa_\alpha \frac{2}{h_{i} + h_{i+1}}
    \left( \frac{1}{h_{i}} \tilde{a}_{i-1,j} - \left( \frac{1}{h_{i}} + \frac{1}{h_{i+1}} \right) \tilde{a}_{i,j} +  \frac{1}{h_{i+1}} \tilde{a}_{i+1, j} \right)
  \end{aligned}
  \end{equation}
where
\begin{equation} \label{eq:tildeaij}
  \begin{aligned}
    \tilde{a}_{ij} &= I^{2-\alpha} \phi_j(x_i) \\
    &= \frac{1}{\Gamma(4-\alpha)}
    \left( \frac{|x_{i}-x_{j-1}|^{3-\alpha}}{h_{j}} -\left( \frac{1}{h_{i}} + \frac{1}{h_{i+1}} \right)|x_i-x_{j}|^{3-\alpha} +  \frac{|x_{i}-x_{j+1}|^{3-\alpha}}{h_{j+1}} \right)
  \end{aligned}
\end{equation}




We shall finally introduce some notation.

For  convenience, we use the notation \( \simeq \)  . 
That \(x_1\simeq y_1\),
means that \(c_1 x_1 \le y_1 \le C_1 x_1\) 
for some constants \(c_1\) and \(C_1\) that are independent of \(N\).

Meanwhile, let
\textcolor{red}{
\begin{equation} \label{def:Kyx}
  K_y(x) := \frac{|y-x|^{1-\alpha}}{\Gamma(2-\alpha)}
\end{equation}
}
We define the difference quotients
\begin{equation} \label{def:Dhgxi}
  D_h g(x_i) := \frac{g(x_{i+1}) - g(x_i)}{h_{i+1}}, \quad D_{\bar{h}}g(x_i) := \frac{g(x_{i})-g(x_{i-1})} {h_{i}}
\end{equation}
Thus
\begin{gather*}
  D_h g(x_i) = D_{\bar{h}}g(x_{i+1})  \\
  D_h^2 g(x_i) 
  = \frac{2}{h_i + h_{i+1}} \left( D_{h} g(x_i) - D_{\bar{h}} g(x_i) \right) 
  = \frac{2}{h_i + h_{i+1}} \left( D_{h} g(x_i) - D_{h} g(x_{i-1}) \right)
\end{gather*}

And for \(j=1, 2,...,2N\), we define
\begin{equation} \label{def:yjt}
  y_j^\theta = (1-\theta)x_{j-1} + \theta x_{j-1}, \quad \theta \in (0,1)
\end{equation}



% Section 2.2: Regularity
\subsection{Regularity of the true solution}
\label{sec:regularity}


For any \(\beta > 0\), 
we use the standard notation \(C^\beta(\bar{\Omega}),  C^\beta(\mathbb{R})\), etc., for Hölder spaces
and their norms and seminorms.
When no confusion is possible, 
we use the notation \(C^\beta (\Omega)\) to refer to \(C^{k,\beta'} (\Omega)\), where \(k\) is the greatest integer such that \(k<\beta\) and where \(\beta' = \beta - k\).
The Hölder spaces \(C^{k, \beta'}(\Omega)\) are defined as the subspaces of \(C^k(\Omega)\) consisting of functions whose \(k\)-th order partial derivatives are locally Hölder continuous\cite{Gilbarg1977} with exponent \(\beta'\) in \(\Omega\),
where \(C^k(\Omega)\) is the set of all \(k\)-times continuously differentiable functions on open set \(\Omega\).



\textcolor{red}{
\begin{definition}[delta dependent norm \cite{ROSOTON2014275}]
  ...
\end{definition}
}

\textcolor{blue}{
  \begin{lemma} \label{thm:regularity-f}
    Let \(f\in C^\beta(\Omega), \beta>2\) be such that \(\|f\|_{\beta}^{(\alpha/2)}<\infty\), then for \(l=0, 1, 2\)
    \begin{equation}
      |f^{(l)}(x)| \le \|f\|_{\beta}^{(\alpha/2)} \begin{cases}
        x^{-l-\alpha/2}, & \text{if } 0 < x \le T \\ (2T-x)^{-l-\alpha/2}, & \text{if } T\le x < 2T
      \end{cases}
    \end{equation}
  \end{lemma}
}

\textcolor{purple}{
  \begin{theorem}[Regularity up to the boundary \cite{ROSOTON2014275}] \label{thm:regularity}
    Let \(\Omega\) be a bounded domain, and \(\beta>0\) be such that neither \(\beta\) nor \(\beta+\alpha\) is an integer. Let \(f \in C^\beta (\Omega)\) be such that \( \|f\|_{\beta}^{(\alpha/2)} < \infty\), and \(u \in C^{\alpha/2} (\mathbb{R}^n)\) be a solution of \cref{eq:equation}. Then, \(u \in C^{\beta+\alpha} (\Omega)\) and
    \begin{equation}
      \|u\|_{\beta+\alpha}^{(-\alpha/2)} \le C \left( \|u\|_{C^{\alpha/2}(\mathbb{R})} + \|f\|_{\beta}^{(\alpha/2)} \right)
    \end{equation}
    where \(C\) is a constant depending only on \(\Omega\), \(\alpha\), and \(\beta\).
  \end{theorem}
  \begin{corollary} \label{cor:regularity-u}
    Let $u$ be a solution of \eqref{eq:equation} where $f\in L^\infty(\Omega)$ and $\|f\|_{\beta}^{(\alpha/2)} < \infty$. Then, for any $x\in\Omega$ and \(l=0, 1, 2, 3, 4\)
    \begin{equation}
      |u^{(l)}(x)| \le \|u\|_{\beta+\alpha}^{(-\alpha/2)} \begin{cases}
        x^{\alpha/2-l}, & \text{if } 0 < x \le T \\ (2T-x)^{\alpha/2-l}, & \text{if } T\le x < 2T
      \end{cases}
    \end{equation}
  \end{corollary}
}

And in this paper bellow, without special instructions, we allways assume that
\begin{equation} \label{cond:regularity}
  f \in L^\infty(\Omega)\cap C^{\beta}(\Omega) \quad \text{and} \quad \|f\|_{\beta}^{(\alpha/2)} < \infty, \text{  with  } \alpha + \beta > 4
\end{equation}


% section 3: Main results
\subsection{Main results}
\label{sec:main}


Here we state our main results; the proof is
deferred to \cref{sec:proof-truncation-error} and \cref{sec:proof-convergence}.

\textcolor{blue}{
  Let's denote \(h=\frac{1}{N}\), we have
  \begin{theorem}[Local Truncation Error] \label{thm:truncation-error}
    If $u(x)$ is a solution of the equation \eqref{eq:equation} where $f$ satisfy the regular condition \eqref{cond:regularity}, then there exists 
    $C_1(T,\alpha, r, \|u\|_{\beta+\alpha}^{(-\alpha/2)}, \|f\|_{\beta}^{(\alpha/2)})$ and 
    $C_2(T,\alpha, r,  \|u\|_{\beta+\alpha}^{(-\alpha/2)})$, such that
    the truncation error \cref{def:truncation_error} satisfies
    \begin{equation} \label{eq:truncation-error}
      \begin{aligned}
        |\tau_i| :=& | -\kappa_\alpha D_h^{\alpha} \Pi_hu(x_i) - f(x_i) | \\
        \le & C_1  h^{\min\{\frac{r\alpha}{2}, 2\}} \begin{cases}
                                                      x_i^{-\alpha},       & 1\le i\le N \\
                                                      (2T-x_i)^{-\alpha} , & N<i\le 2N-1
                                                    \end{cases} \\
            & + C_2(r-1) h^2
        \begin{cases}
          |T-x_{i-1}|^{1-\alpha}, \quad 1\le i\le N \\
          |T-x_{i+1}|^{1-\alpha} , \quad N<i\le 2N-1
        \end{cases}
      \end{aligned}
    \end{equation}
  \end{theorem}
}


\textcolor{blue}{
  \begin{theorem}[Global Error]\label{thm:convergence}
    The discrete equation \eqref{def:discrete_equation} has sulotion
    and there exists a positive constant $C=C(T,\alpha, r, \|u\|_{\beta+\alpha}^{(-\alpha/2)}, \|f\|_{\beta}^{(\alpha/2)})$
    such that the error between the numerial solution $U$ with the exact solution $u(x_i)$ satisfies
    \begin{equation} \label{eq:error}
      \max_{1\le i \le 2N-1} |u_i - u(x_i)| \le C h^{\min\{\frac{r\alpha}{2}, 2\}}
    \end{equation}
    That means the numerial method has convergence order ${\min\{\frac{r\alpha}{2}, 2\}}$.
  \end{theorem}
}

\textcolor{red}{
\begin{remark}
  ...
\end{remark}
}

\section{Local Truncation Error}
\label{sec:proof-truncation-error}

\subsection{Proof of \cref{thm:truncation-error}}

The truncation error of the discrete format can be written as
\begin{equation} \label{eq:truncerrordepart}
  \begin{aligned}
    - \kappa_\alpha D_h^{\alpha} \Pi_h u(x_i) - f(x_i)
     & = -\kappa_\alpha (D_h^2 I^{2-\alpha} \Pi_h u(x_i) - \frac{d^2}{dx^2} I^{2-\alpha} u(x_i))                                      \\
     & = - \kappa_\alpha D_h^2 I^{2-\alpha} (\Pi_h u - u)(x_i) - \kappa_\alpha (D_h^2 - \frac{d^2}{dx^2}) I^{2-\alpha} u(x_i)   \\
  \end{aligned}
\end{equation}


% \subsection{Estimate of $-\kappa_\alpha (D_h^2 - \frac{d^2}{dx^2}) I^{2-\alpha} (x_i)$}

\textcolor{blue}{
  \begin{theorem} \label{lmm:trunerror2}
    There exits a constant $C=C(T,\alpha, r, \|f\|_{\beta}^{(\alpha/2)})$ such that
    \begin{equation}
      \left|-\kappa_\alpha (D_h^2 - \frac{d^2}{dx^2}) I^{2-\alpha}u (x_i) \right| 
      \le C h^2 \begin{cases}
        x_i^{-\alpha/2-2/r} ,     & 1\le i\le N    \\
        (2T-x_i)^{-\alpha/2-2/r}, & N\le i\le 2N-1
      \end{cases}
    \end{equation}
  \end{theorem}
}
  \begin{proof}
    Since \(f\in C^2(\Omega)\) and
    \begin{equation}
      \frac{d^2}{dx^2} ( - \kappa_\alpha I^{2-\alpha}u(x)) = f(x),  \quad x \in \Omega,
    \end{equation}
    we have \(I^{2-\alpha}u \in C^4(\Omega)\).
    Therefore, using equation \eqref{eq:Dh2simd4} of \cref{lmm:Dh2simd2}, for \(1\le i\le 2N-1\), we have
    \begin{equation}
      \begin{aligned}
        &-\kappa_\alpha (D_h^2 - \frac{d^2}{dx^2}) I^{2-\alpha}u (x_i)
         = \frac{h_{i+1}-h_{i}}{3} f'(x_i) \\
         & \quad + \frac{2}{h_i + h_{i+1}}\left(\frac{1}{h_i} \int_{x_{i-1}}^{x_{i}} f''(y) \frac{(y-x_{i-1})^3}{3!} dy + \frac{1}{h_{i+1}} \int_{x_{i}}^{x_{i+1}} f''(y) \frac{(y-x_{i+1})^3}{3!} dy\right) \\
      \end{aligned}
    \end{equation}
    By \cref{lmm:hi1-hi}, \cref{thm:regularity-f} and \cref{lmm:trucerr2d2f},  we get the result.
  \end{proof}



% \subsection{Estimate of $R_i$}
\label{subsec:Ri}


And now define
\begin{equation} \label{eq:Ri}
  \begin{aligned}
    R_i & := D_h^2 I^{2-\alpha}(u-\Pi_h u)(x_i) 
  \end{aligned}
\end{equation}


We have some results about the estimate of $R_i$
\textcolor{blue}{
  \begin{theorem} \label{thm:Ri-ilessN/2}
    For \(1\le i < N/2\), there exists $C=C(T, \alpha, r, \|u\|_{\beta+\alpha}^{(-\alpha/2)})$ such that
    \begin{equation}
      R_i \le \begin{cases}
        C h^2 x_i^{-\alpha/2-2/r} ,             & \alpha/2 - 2/r + 1 > 0 \\
        C h^2 (x_i^{-1-\alpha}\ln(i) + \ln(N)), & \alpha/2 - 2/r + 1 = 0 \\
        C h^{r\alpha/2+r} x_i^{-1-\alpha},        & \alpha/2 - 2/r + 1 < 0
      \end{cases}
    \end{equation}
  \end{theorem}
}
\textcolor{blue}{
  \begin{theorem} \label{thm:Ri-N/2le-i-leN}
    For \(N/2 \le i\le N\), there exists constant $C=C(T, \alpha, r, \|u\|_{\beta+\alpha}^{(-\alpha/2)})$ such that
    \begin{equation}
      R_i \le C(r-1) h^2|T-x_{i-1}|^{1-\alpha}  + \begin{cases}
        C h^2,             & \alpha/2-2/r+1 > 0 \\
        C h^2 \ln(N) ,     & \alpha/2-2/r+1 = 0 \\
        C h^{r\alpha/2+r}, & \alpha/2-2/r+1 < 0
      \end{cases}
    \end{equation}
  \end{theorem}
}

And for \(N<i\le 2N-1\), it is symmetric to the previous case.

% \subsection{Truncation error}

Combine \cref{lmm:trunerror2}, \cref{thm:Ri-ilessN/2} and \cref{thm:Ri-N/2le-i-leN},
the proof of \cref{thm:truncation-error} completed.
% For \(1\le i \le N\)
% \begin{equation}
%   \begin{aligned}
%     R_i \le C_2(r-1)h^2 |T-x_{i-1}|^{1-\alpha} + \begin{cases}
%                                                    C_1 h^2 x_i^{-\alpha/2-2/r},             & r\alpha/2+r-2 > 0 \\
%                                                    C_1 h^2 (x_i^{-1-\alpha}\ln(i)+ \ln(N)), & r\alpha/2+r-2 = 0 \\
%                                                    C_1 h^{r\alpha/2+r} x_i^{-1-\alpha/2},   & r\alpha/2+r-2 < 0
%                                                  \end{cases}
%   \end{aligned}
% \end{equation}

% But,
% \begin{gather}
%   h^2 x_i^{-\alpha/2-2/r}
%   \le T^{\alpha/2-2/r} \begin{cases}
%     h^2 x_i^{-\alpha},           & \text{if} \quad  r\alpha/2-2 \ge 0 \\
%     h^{r\alpha/2} x_i^{-\alpha}, & \text{if} \quad  r\alpha/2-2 \le 0
%   \end{cases}    \\
%   h^{r\alpha/2+r} x_i^{-1-\alpha} \le T^{-1} h^{r\alpha/2} x_i^{-\alpha}, \quad \text{if} \quad  r\alpha/2-2 \le 0    \\
% \end{gather}

% And when \(r\alpha/2 - 2 = -r < 0\),
% \begin{equation}
%   \begin{aligned}
%     h^2 x_i^{-1-\alpha} \ln(i) h^{-r\alpha/2} x_i^{\alpha}
%      & = h^r x_i^{-1} \ln(i)                   \\
%      & = T^{-1} \frac{\ln(i)}{i^r} \le C(T, r)
%   \end{aligned}
% \end{equation}
% and
% \begin{equation}
%   h^2 \ln(N) h^{-r\alpha/2} x_i^{\alpha} = h^r \ln(N) x_i^{\alpha} \le T^{\alpha} \frac{\ln(N)}{N^r} \le C(T, \alpha, r)
% \end{equation}

% So for \(1\le i \le N\),
% \begin{equation}
%   \begin{aligned}
%     R_i \le C_2(r-1)h^2 |T-x_{i-1}|^{1-\alpha} + C_1 h^{\min\{\frac{r\alpha}{2}, 2\}} x_i^{-\alpha}
%   \end{aligned}
% \end{equation}
% And for \(i\ge N\), it is symmetric for \(i\) and \(2N-i\).






We prove \cref{thm:Ri-ilessN/2} and \cref{thm:Ri-N/2le-i-leN} in next subsections below.

\subsection{\textcolor{purple}{Mesh Transport Functions}}
\label{subsec:mesh-transport-functions}




\textcolor{purple}{
\begin{definition} [Mesh Transport Functions]
  \begin{equation} \label{def:yij}
    y_{i,j}(x) = \begin{cases}
      (x^{1/r} + Z_{j-i})^r               & i< N, j< N \\
      \dfrac{x^{1/r} - Z_i}{Z_1} h_N + x_N  & i< N, j=N \\
      2T - (Z_{2N-(j-i)} - x^{1/r})^r     & i< N, j> N \\
      \left(\dfrac{Z_1}{h_N}  (x-x_N) + Z_j \right)^r  & i=N, j< N \\
      x,                                  & i=N, j=N \\
      % 2T - \left(\dfrac{Z_1}{h_N}  (2T-x-x_N) + Z_{2N-j} \right)^r   & i=N , j>N
    \end{cases}
  \end{equation}
  where
  \begin{equation} \label{def:Zj}
    Z_{j} := T^{1/r}\frac{j}{N}
  \end{equation}
  And
  \begin{equation} \label{def:hij}
    h_{i,j}(x) = y_{i,j}(x) - y_{i,j-1}(x)
  \end{equation}
  \begin{equation} \label{def:yijt}
    y_{i,j}^\theta(x) = (1-\theta) y_{i,j-1}(x) + \theta y_{i,j-1}(x), \quad \theta \in (0, 1)
  \end{equation}
\end{definition}
}

We give some properties of mesh transport functions.
\begin{lemma}
  Obviously,
  \begin{gather}
    y_{i,j}(x_{i-1}) = x_{j-1}, \quad y_{i,j}(x_{i}) = x_{j}, \quad y_{i,j}(x_{i+1}) = x_{j+1} \\
    h_{i,j}(x_{i-1}) = h_{j-1}, \quad h_{i,j}(x_{i}) = h_j, \quad h_{i,j}(x_{i+1}) = h_{j+1} \\
    y_{i,j}^\theta(x_{i-1}) = y_{j-1}^\theta, \quad y_{i,j}^\theta(x_{i}) = y_{j}^\theta, \quad y_{i,j}^\theta(x_{i+1}) = y_{j+1}^\theta
  \end{gather}
\end{lemma}

\begin{lemma}
  For \(1\le i \le 2N-1, 2\le j \le 2N-1\),
  \begin{equation}
    h_{i,j}(\xi) \simeq h_j, \quad \text{for  } \xi \in (x_{i-1}, x_{i+1})
  \end{equation}
  For \(1\le i, j \le 2N-1\),
  \begin{equation}
    |y_{i,j}(\xi) - \xi| \simeq |x_j - x_i| \quad \text{for  } \xi \in (x_{i-1}, x_{i+1})
  \end{equation}
\end{lemma}


\subsection{Proof of \cref{thm:Ri-ilessN/2}}

% \begin{equation}
%   \begin{aligned}
%     D_h^2 I^{2-\alpha}(u-\Pi_h u)(x_i)
%      & = D_h^2(\int_{0}^{2T}(u(y)-\Pi_hu(y)) \frac{|y-x_i|^{1-\alpha}}{\Gamma(2-\alpha)} dy) \\
%   \end{aligned}
% \end{equation}
For convience, let's denote
\begin{equation} \label{def:Tij}
  T_{ij} = \int_{x_{j-1}}^{x_{j}} (u(y) - \Pi_hu(y)) \frac{|y-x_i|^{1-\alpha}}{\Gamma(2-\alpha)} dy, \quad i=0, \cdots ,2N,\; j=1, \cdots , 2N
\end{equation}
Also for simplicity, we denote
\begin{definition}
  \begin{equation}
    S_{ij} =  \frac{2}{h_{i} + h_{i+1}}  \left( \frac{1}{h_{i}}  T_{i-1,j} - \left(\frac{1}{h_{i}} + \frac{1}{h_{i+1}}\right)  T_{i,j} + \frac{1}{h_{i+1}} T_{i+1,j} \right)
  \end{equation}
  then
  \begin{equation}
    R_i = \sum_{j=1}^{2N} S_{ij}
  \end{equation}
\end{definition}

\textcolor{blue}{
  \begin{lemma} \label{lmm:sumSi-ilessN/2-2ip1-N}
    There exists a constant \(C=C(T, \alpha, r, \|u\|_{\beta+\alpha}^{(-\alpha/2)})\) such that for  \(1 \le i < N/2\),
    \begin{equation}
      \begin{aligned}
        \sum_{j=\max\{2i+1, i+3\}}^{N} S_{ij} \le C h^2 x_i^{-\alpha/2-2/r}
      \end{aligned}
    \end{equation}
  \end{lemma}
}
  \begin{proof}
    For  \(\max\{2i+1, i+3\} \le j \le N\), by \cref{lmm:Dyjleh2ya/2m2/r} and \cref{lmm:Dh2ymx1malecy1ma}
    \begin{equation}
      \begin{aligned}
        S_{ij} & = \int_{x_{j-1}}^{x_{j}}(u(y) - \Pi_hu(y)) D_h^2 K_y (x_i) dy \\
        % &\le C h^2 \int_{x_{j-1}}^{x_{j}} y^{\alpha/2-2/r} \frac{|y-x_{i+1}|^{-1-\alpha}}{\Gamma(-\alpha)} dy \\
               & \le C h^2 \int_{x_{j-1}}^{x_{j}} y^{\alpha/2-2/r} \frac{y^{-1-\alpha}}{\Gamma(-\alpha)} dy                       \\
               & = C h^2 \int_{x_{j-1}}^{x_{j}} y^{-\alpha/2-2/r-1} dy
      \end{aligned}
    \end{equation}
    Therefore,
    \begin{equation}
      \begin{aligned}
        \sum_{j=\max\{2i+1, i+3\}}^{N} S_{ij}
         & \le C h^2 \int_{x_{2i}}^{x_{N}} y^{-\alpha/2-2/r-1} dy                     \\
         & = \frac{C}{\alpha/2+2/r} h^2 ( x_{2i}^{-\alpha/2-2/r} - T^{-\alpha/2-2/r}) \\
         & \le \frac{C}{\alpha/2+2/r} 2^{r(-\alpha/2-2/r)} h^2 x_i^{-\alpha/2-2/r}
      \end{aligned}
    \end{equation}
  \end{proof}

\textcolor{blue}{
  \begin{lemma} \label{lmm:sumSi-ilessN/2-Np1-2N}
    Thert exists a constant \(C=C(T, \alpha, r, \|u\|_{\beta+\alpha}^{(-\alpha/2)})\) such that for  \(1 \le i < N/2\),
    \begin{equation}
      \begin{aligned}
        \sum_{j=N+1}^{2N} S_{ij}
        \le \begin{cases}
              C h^2,             & \alpha/2-2/r+1 > 0 \\
              C h^2 \ln(N) ,     & \alpha/2-2/r+1 = 0 \\
              C h^{r\alpha/2+r}, & \alpha/2-2/r+1 < 0
            \end{cases}
      \end{aligned}
    \end{equation}
  \end{lemma}
}
\begin{proof}
  For \(1 \le i < N/2, N+1\le j\le 2N-1\), by equation \cref{lmm:Dyjleh22T-y} and \cref{lmm:Dh2ymx1malecy1ma}
  \begin{equation*}
    \begin{aligned}
      S_{ij} & = \int_{x_{j-1}}^{x_{j}}(u(y) - \Pi_hu(y)) D_h^2 K_y (x_i) dy \\
             & \le \int_{x_{j-1}}^{x_{j}} C h^2 (2T-y)^{\alpha/2-2/r} y^{-1-\alpha} dy                                          \\
             & \le C h^2  T^{-1-\alpha} \int_{x_{j-1}}^{x_{j}} (2T-y)^{\alpha/2-2/r} dy
    \end{aligned}
  \end{equation*}
  \begin{equation}
    \begin{aligned}
      \sum_{j=N+1}^{2N-1} S_{ij}
       & \le C T^{-1-\alpha} h^2 \int_{x_{N}}^{x_{2N-1}} (2T-y)^{\alpha/2-2/r}  dy                       \\
       & \le CT^{-1-\alpha} h^2 \begin{cases}
                                  \frac{1}{\alpha/2-2/r+1} T^{\alpha/2-2/r+1},        & \alpha/2-2/r+1 > 0 \\
                                  \ln(T)-\ln(h_{2N}),                                 & \alpha/2-2/r+1 = 0 \\
                                  \frac{1}{|\alpha/2-2/r+1|} h_{2N}^{\alpha/2-2/r+1}, & \alpha/2-2/r+1 < 0
                                \end{cases} \\
       & = \begin{cases}
             \frac{C}{\alpha/2-2/r+1}T^{-\alpha/2-2/r} \; h^2,                & \alpha/2-2/r+1 > 0 \\
             CrT^{-1-\alpha} h^2 \ln(N),                                      & \alpha/2-2/r+1 = 0 \\
             \frac{C}{|\alpha/2-2/r+1|} T^{-\alpha/2-2/r} \; h^{r\alpha/2+r}, & \alpha/2-2/r+1 < 0
           \end{cases}
    \end{aligned}
  \end{equation}
  And by \cref{lmm:Dyj1}
  \begin{equation*}
    S_{i,2N} \le C T^{-1-\alpha} h_{2N}^{\alpha/2+1} = C T^{-\alpha/2} h^{r\alpha/2+r}
  \end{equation*}
  And when \(\alpha/2-2/r+1\ge 0\),
  \begin{equation*}
    h^{r\alpha/2+r} \le h^2
  \end{equation*}
  Summarizes, we get the result.
\end{proof}



For \(i=1, 2\).

\begin{lemma} \label{lmm:R1R2}
  By \cref{lmm:sumSij13} , \cref{lmm:sumSi-ilessN/2-2ip1-N}  and \cref{lmm:sumSi-ilessN/2-Np1-2N}   we get
  \begin{equation}
    \begin{aligned}
      R_1 & = \sum_{j=1}^{3} S_{1j} + \sum_{j=4}^{2N} S_{1j} \\
          & \le C h^2 x_1^{-\alpha/2-2/r} +
      \begin{cases}
        C h^2,             & \alpha/2-2/r+1 > 0 \\
        C h^2 \ln(N) ,     & \alpha/2-2/r+1 = 0 \\
        C h^{r\alpha/2+r}, & \alpha/2-2/r+1 < 0
      \end{cases}
    \end{aligned}
  \end{equation}
  \begin{equation}
    \begin{aligned}
      R_2 & = \sum_{j=1}^{4} S_{2j} + \sum_{j=5}^{2N} S_{2j} \\
          & \le C h^2 x_2^{-\alpha/2-2/r} +
      \begin{cases}
        C h^2,             & \alpha/2-2/r+1 > 0 \\
        C h^2 \ln(N) ,     & \alpha/2-2/r+1 = 0 \\
        C h^{r\alpha/2+r}, & \alpha/2-2/r+1 < 0
      \end{cases}
    \end{aligned}
  \end{equation}
\end{lemma}


For \(3\le i < N/2\), we have a new separation of \(R_i\), Let's denote \(k=\lceil\frac{i}{2}\rceil\).

\includegraphics[width=0.9\textwidth]{Depart_ManimCE_v0.18.1.png}

\begin{equation}
  \begin{aligned}
    R_i
    = & \sum_{j=1}^{2N} \frac{2}{h_i + h_{i+1}}
    \left( \frac{1}{h_{i+1}} T_{i+1, j}
    - (\frac{1}{h_{i}}+\frac{1}{h_{i+1}}) T_{i,j}
    +  \frac{1}{h_{i}} T_{i-1, j} \right)                  \\
    = & \sum_{j=1}^{k-1} \frac{2}{h_i + h_{i+1}}
    \left( \frac{1}{h_{i+1}} T_{i+1, j}
    - (\frac{1}{h_{i}}+\frac{1}{h_{i+1}}) T_{i,j}
    +  \frac{1}{h_{i}} T_{i-1, j} \right)                  \\
      & + \frac{2}{h_i + h_{i+1}}
    \left( \frac{1}{h_{i+1}} (T_{i+1, k} +  T_{i+1, k+1})
    - (\frac{1}{h_{i}}+\frac{1}{h_{i+1}}) T_{i,k} \right)  \\
      & + \sum_{j=k+1}^{2i-1} \frac{2}{h_i + h_{i+1}}
    \left( \frac{1}{h_{i+1}} T_{i+1, j+1}
    - (\frac{1}{h_{i}}+\frac{1}{h_{i+1}}) T_{i,j}
    +  \frac{1}{h_{i}} T_{i-1, j-1} \right)                \\
      & + \frac{2}{h_i + h_{i+1}}
    \left( \frac{1}{h_{i}} (T_{i-1, 2i} +  T_{i-1, 2i-1})
    - (\frac{1}{h_{i}}+\frac{1}{h_{i+1}}) T_{i,2i} \right) \\
      & + \sum_{j=2i+1}^{2N} \frac{2}{h_i + h_{i+1}}
    \left( \frac{1}{h_{i+1}} T_{i+1, j}
    - (\frac{1}{h_{i}}+\frac{1}{h_{i+1}}) T_{i,j}
    +  \frac{1}{h_{i}} T_{i-1, j} \right)                  \\
    = & I_1 + I_2 + I_3 + I_4 + I_5
  \end{aligned}
\end{equation}
% where \(I_1\) makes sence only if \(i\ge 3\).



\textcolor{blue}{
  \begin{lemma} \label{lmm:Ri-I1}
    There exists a constant \(C=C(T, \alpha, r, \|u\|_{\beta+\alpha}^{(-\alpha/2)})\) such that for \(3\le i \le N, k=\lceil\frac{i}{2}\rceil\)
    \begin{equation}
      |I_1| = |\sum_{j=1}^{k-1} S_{ij}| \le \begin{cases}
        C h^2 x_i^{-\alpha/2-2/r} ,        & \alpha/2-2/r+1 > 0 \\
        C h^2 x_i^{-1-\alpha} \ln(i),      & \alpha/2-2/r+1 = 0 \\
        C h^{r\alpha/2+r} x_i^{-1-\alpha}, & \alpha/2-2/r+1 < 0
      \end{cases}
    \end{equation}
  \end{lemma}
}
\begin{proof}
  by \cref{lmm:Dyj1} , \cref{lmm:Dh2xmy1malecx1ma}
  \begin{equation}
    S_{i1} \le C x_1^{\alpha/2} x_1 x_i^{-1-\alpha} = C x_1^{\alpha/2+1} x_i^{-1-\alpha} = C T^{\alpha/2+1} h^{r\alpha/2+r} x_i^{-1-\alpha}
  \end{equation}
  For  \(2 \le j \le k-1\), by \cref{lmm:Dyjleh2ya/2m2/r} and \cref{lmm:Dh2xmy1malecx1ma}
  \begin{equation}
    \begin{aligned}
      S_{ij} & = \int_{x_{j-1}}^{x_{j}}(u(y) - \Pi_hu(y)) D_h^2 K_y(x_i) dy \\
             & \le C h^2 \int_{x_{j-1}}^{x_{j}} y^{\alpha/2-2/r} \frac{x_i^{-1-\alpha}}{\Gamma(-\alpha)} dy                     \\
             & = C h^2 x_i^{-1-\alpha} \int_{x_{j-1}}^{x_{j}} y^{\alpha/2-2/r} dy
    \end{aligned}
  \end{equation}
  Therefore,
  \begin{equation}
    \begin{aligned}
      I_1 & = \sum_{j=1}^{k-1} S_{ij} = S_{i1} + \sum_{j=2}^{k-1} S_{ij}                                                                 \\
          & \le C h^{r\alpha/2+r} x_i^{-1-\alpha} + C h^2 x_i^{-1-\alpha} \int_{x_1}^{x_{\lceil\frac{i}{2}\rceil-1}} y^{\alpha/2-2/r} dy \\
          & \le C h^{r\alpha/2+r} x_i^{-1-\alpha} + C h^2 x_i^{-1-\alpha} \int_{x_1}^{2^{-r} x_{i}} y^{\alpha/2-2/r} dy                  \\
      % &\le C h^{r\alpha/2+r} x_i^{-1-\alpha} + \frac{C}{\alpha/2-2/r+1} h^2 x_i^{-1-\alpha} (x_1^{\alpha/2-2/r+1} + (2^{-r} x_i)^{\alpha/2-2/r+1} )  \\
      % &\le C (h^{r\alpha/2+r} x_i^{-1-\alpha} + h^2 x_i^{-\alpha/2-2/r} )
    \end{aligned}
  \end{equation}
  But
  \begin{equation}
    \begin{aligned}
      \int_{x_1}^{2^{-r} x_{i}} y^{\alpha/2-2/r} dy
       & \le \begin{cases}
               \frac{1}{\alpha/2-2/r+1} (2^{-r} x_i)^{\alpha/2-2/r+1} , & \alpha/2-2/r+1 > 0 \\
               \ln(2^{-r} x_i) - \ln(x_1),                              & \alpha/2-2/r+1 = 0 \\
               \frac{1}{|\alpha/2-2/r+1|} x_1^{\alpha/2-2/r+1},         & \alpha/2-2/r+1 < 0
             \end{cases}
    \end{aligned}
  \end{equation}
  So we have
  \begin{equation}
    I_1 \le \begin{cases}
      \frac{C}{\alpha/2-2/r+1} h^2 x_i^{-\alpha/2-2/r} ,          & \alpha/2-2/r+1 > 0 \\
      C h^2 x_i^{-1-\alpha} \ln(i),                               & \alpha/2-2/r+1 = 0 \\
      \frac{C}{|\alpha/2-2/r+1|} h^{r\alpha/2+r} x_i^{-1-\alpha}, & \alpha/2-2/r+1 < 0
    \end{cases}
  \end{equation}
\end{proof}

\begin{definition}
  For convience, let's denote
  \begin{equation} \label{def:Vij}
    V_{ij} = \frac{2}{h_{i} + h_{i+1}}  \left( \frac{1}{h_{i+1}} T_{i+1, j+1} - \left(\frac{1}{h_{i}}+\frac{1}{h_{i+1}}\right)  T_{i,j} + \frac{1}{h_{i}} T_{i-1, j-1} \right)
  \end{equation}
\end{definition}

\textcolor{blue}{
  \begin{theorem} \label{Ri-I3-2lN/2}
    There exists a constant \(C=C(T, \alpha, r, \|u\|_{\beta+\alpha}^{(-\alpha/2)})\) such that for \(3\le i < N/2, k=\lceil\frac{i}{2}\rceil\),
    \begin{equation}
      I_3 = \sum_{j=k+1}^{2i-1} V_{ij} \le C h^2 x_i^{-\alpha/2-2/r}
    \end{equation}
  \end{theorem}
}

To estimete \(V_{ij}\), we need some preparations.
\begin{lemma} 
  For \(y\in (x_{j-1}, x_{j})\), we can rewrite 
  \begin{equation}
    y = x_{j-1} + \theta h_j = (1-\theta) x_{j-1} + \theta x_j =: y_j^\theta , \; \theta\in (0,1)
  \end{equation}
  by \cref{lmm:Dyj},
  \begin{equation}
    \begin{aligned}
      T_{ij} & = \int_{x_{j-1}}^{x_{j}} (u(y) - \Pi_hu(y)) \frac{|y-x_i|^{1-\alpha}}{\Gamma(2-\alpha)} dy  \\
             & = \int_{0}^{1} (u(y_j^\theta) - \Pi_hu(y_j^\theta)) \frac{|y_j^\theta-x_i|^{1-\alpha}}{\Gamma(2-\alpha)} h_j d\theta  \\
      %        & = \int_{x_{j-1}}^{x_{j}}
      % -\frac{\theta (1-\theta)}{2} h_j^2 u''(y_j^\theta)\frac{|y_j^\theta-x_i|^{1-\alpha}}{\Gamma(2-\alpha)} \\
      %        & \quad + \frac{\theta (1-\theta)}{3!} h_j^3 \frac{|y_j^\theta-x_i|^{1-\alpha}}{\Gamma(2-\alpha)} ((1-\theta)^2 u'''(\eta_{j1}^\theta) - \theta^2  u'''(\eta_{j2}^\theta)) d y_j^\theta \\
             & = \int_{0}^{1}
      -\frac{\theta (1-\theta)}{2} h_j^3 u''(y_j^\theta)\frac{|y_j^\theta-x_i|^{1-\alpha}}{\Gamma(2-\alpha)}   \\
             & \quad + \frac{\theta (1-\theta)}{3!} h_j^4 \frac{|y_j^\theta-x_i|^{1-\alpha}}{\Gamma(2-\alpha)} ( \theta^2 u'''(\eta_{j1}^\theta) - (1-\theta)^2 u'''(\eta_{j2}^\theta)) d\theta
    \end{aligned}
  \end{equation}
  where \(\eta_{j1}^\theta \in (x_{j-1}, y_j^\theta), \eta_{j2}^\theta \in (y_j^\theta, x_j)\).
\end{lemma}
Now Let's construct a series of functions to represent \(T_{ij}\).
\begin{definition} \label{def:yj-i}
  \begin{equation} \label{def:xi2yj-jlN}
    {y_{j-i}}(x) = (x^{1/r}+ Z_{j-i})^r, \quad Z_{j-i} = T^{1/r}\frac{j-i}{N}
  \end{equation}
  Particularly, for \(i,j\le N-1\),
  \begin{equation*}
    y_{j-i}(x_{i-1}) = x_{j-1}, \quad y_{j-i}(x_{i}) = x_{j}, \quad y_{j-i}(x_{i+1}) = x_{j+1}
  \end{equation*}
  \begin{gather} \label{eq:dy-j<N}
    {y_{j-i}}'(x) = y_{j-i}(x)^{1-1/r} x^{1/r-1}  \\
    {y_{j-i}}''(x) = \frac{1-r}{r} y_{j-i}(x)^{1-2/r} x^{1/r-2} Z_{j-i}  \\
  \end{gather}
  \begin{equation} \label{def:xi2yj-jlNt}
    {y_{j-i}^{\theta}}(x) = (1-\theta) {y_{j-1-i}}(x) + \theta {y_{j-i}}(x)
  \end{equation}
  \begin{equation} \label{def:hj-jlN}
    {h_{j-i}}(x) = {y_{j-i}}(x) - {y_{j-i-1}}(x)
  \end{equation}
  Now, we define
  \begin{equation} \label{def:Pj-itheta-jlN}
    {P_{j-i}^\theta}(x) = ({h_{j-i}}(x))^3 u''({y_{j-i}^\theta}(x)) \frac{|{y_{j-i}^\theta}(x)-x|^{1-\alpha}}{\Gamma(2-\alpha)}
  \end{equation}
  \begin{equation} \label{def:Qj-itheta-jlN}
    {Q_{j-i}^\theta}(x) = ({h_{j-i}}(x))^4 \frac{|{y_{j-i}^\theta}(x)-x|^{1-\alpha}}{\Gamma(2-\alpha)}
  \end{equation}
\end{definition}
And now we can rewrite \(T_{ij}\)
\begin{lemma} \label{lmm:Tij-express-as-int-of-function}
  For \(2\le i \le N, 2 \le j \le N\),
  \begin{equation}
    \begin{aligned}
      T_{ij} & = \int_{0}^{1} -\frac{\theta (1-\theta)}{2} {P_{j-i}^\theta}(x_i) d\theta    \\
             & + \int_{0}^{1} \frac{\theta (1-\theta)}{3!}{Q_{j-i}^\theta}(x_i) \left[ \theta^2  u'''(\eta_{j,1}^\theta) - (1-\theta)^2 u'''(\eta_{j,2}^\theta) \right] d\theta
    \end{aligned}
  \end{equation}
\end{lemma}

Immediately, we can see from \eqref{def:Vij} that
\begin{lemma} \label{lmm:Vij-int}
  For \( 3 \le i,j \le N-1\),
  \begin{equation}
    \begin{aligned}
      V_{ij} & = \frac{2}{h_{i} + h_{i+1}}  \left( \frac{1}{h_{i+1}} T_{i+1, j+1} - \left(\frac{1}{h_{i}}+\frac{1}{h_{i+1}}\right)  T_{i,j} + \frac{1}{h_{i}} T_{i-1, j-1} \right)                                                      \\
             & = \int_{0}^{1} -\frac{\theta (1-\theta)}{2} D_h^2 P_{j-i}^\theta(x_i)  d\theta                                                                                                                                           \\
             & \quad +  \int_{0}^1 \frac{\theta^3 (1-\theta)}{3!} \frac{2}{h_{i} + h_{i+1}}\left( \frac{{Q_{j-i}^\theta}(x_{i+1}) u'''(\eta_{j+1,1}^\theta) - {Q_{j-i}^\theta}(x_{i}) u'''(\eta_{j,1}^\theta)}{h_{i+1}}\right)  d\theta \\
             & \quad -  \int_{0}^1 \frac{\theta^3 (1-\theta)}{3!} \frac{2}{h_{i} + h_{i+1}}\left( \frac{{Q_{j-i}^\theta}(x_{i}) u'''(\eta_{j,1}^\theta) - {Q_{j-i}^\theta}(x_{i-1}) u'''(\eta_{j-1,1}^\theta)}{h_{i}}\right)  d\theta   \\
             & \quad -  \int_{0}^1 \frac{\theta (1-\theta)^3}{3!} \frac{2}{h_{i} + h_{i+1}}\left( \frac{{Q_{j-i}^\theta}(x_{i+1}) u'''(\eta_{j+1,2}^\theta) - {Q_{j-i}^\theta}(x_{i}) u'''(\eta_{j,2}^\theta)}{h_{i+1}}\right)  d\theta \\
             & \quad +  \int_{0}^1 \frac{\theta (1-\theta)^3}{3!} \frac{2}{h_{i} + h_{i+1}}\left( \frac{{Q_{j-i}^\theta}(x_{i}) u'''(\eta_{j,2}^\theta) - {Q_{j-i}^\theta}(x_{i-1}) u'''(\eta_{j-1,2}^\theta)}{h_{i}}\right)  d\theta
    \end{aligned}
  \end{equation}
\end{lemma}


To estimate \(V_{ij}\), we first estimate \(D_h^2 P_{j-i}^\theta(x_i)\) , but
By \cref{lmm:Dh2simd2},
\begin{equation}
  D_h^2 {P_{j-i}^\theta}(x_i) = {P_{j-i}^\theta}''(\xi), \quad \xi\in (x_{i-1}, x_{i+1})
\end{equation}
By Leibniz formula, we calculate and estimate the derivations of
\(h_{j-i}^3(x)\), \(u''(y_{j-i}^{\theta}(x)) \) and \(\frac{|{y_{j-i}^\theta}(x)-x|^{1-\alpha}}{\Gamma(2-\alpha)}\) separately.



Firstly, we have
\begin{lemma} \label{lmm:estimatesofhj-i3}
  There exists a constant \(C=C(T, r)\) such that
  For \(3\le i \le N-1, \lceil\frac{i}{2}\rceil \le j \le \min\{2i, N\}\), \(\xi\in (x_{i-1}, x_{i+1})\),
  \begin{gather}
    h_{j-i}^3(\xi) \le C h^{2} x_{i}^{2-2/r}  h_j \\
    (h_{j-i}^3(\xi))' \le C(r-1)  h^2  x_{i}^{1-2/r} h_{j} \\
    (h_{j-i}^3(\xi))'' \le C(r-1)  h^2  x_{i}^{-2/r} h_{j}
  \end{gather}
\end{lemma}
The proof of this theorem see \cref{lmm:dhj-i3le-Ch2xi1-2r} and \cref{lmm:d2hj-i3leh2xi-2r}

Second,
\begin{lemma} \label{lmm:estimatesofu2yj-itheta}
  There exists a constant \(C=C(T, \alpha, r, \|u\|_{\beta+\alpha}^{(-\alpha/2)})\) such that
  For \(3\le i \le N-1, \lceil\frac{i}{2}\rceil \le j \le \min\{2i, N\}\), \(\xi\in (x_{i-1}, x_{i+1})\),
  \begin{gather}
    u''(y_{j-i}^{\theta}(\xi)) \le C x_i^{\alpha/2-2}    \\
    (u''(y_{j-i}^{\theta}(\xi)))' \le C x_i^{\alpha/2-3} \\
    (u''(y_{j-i}^{\theta}(\xi)))'' \le C x_i^{\alpha/2-4}
  \end{gather}
\end{lemma}
The proof of this theorem see \cref{prf:estimatesofu2yj-itheta}

And Finally, we have
\begin{lemma} \label{lmm:estimatesofy-xi1-a}
  There exists a constant \(C=C(T, \alpha, r)\) such that
  For \(3\le i \le N-1, \lceil\frac{i}{2}\rceil \le j \le \min\{2i, N\}\), \(\xi\in (x_{i-1}, x_{i+1})\),
  \begin{gather}
    |{y_{j-i}^\theta}(\xi)-\xi|^{1-\alpha} \le C | y_j^\theta - x_i|^{1-\alpha} \\
    \left|(|{y_{j-i}^\theta}(\xi)-\xi|^{1-\alpha})'\right| \le C | y_j^\theta - x_i|^{1-\alpha} x_i^{-1} \\
    \left| (|{y_{j-i}^\theta}(\xi)-\xi|^{1-\alpha})'' \right| \le C | y_j^\theta - x_i|^{1-\alpha} x_i^{-2}
  \end{gather}
  where \(y_j^\theta = \theta x_{j-1} + (1-\theta) x_{j}\)
\end{lemma}
The proof of this theorem see \cref{prf:estimatesofy-xi1-a}


\textcolor{blue}{
  \begin{lemma} \label{lmm:d2Pj-itle}
    There exists a constant \(C=C(T, \alpha, r, \|u\|_{\beta+\alpha}^{(-\alpha/2)})\) such that
    For \(3\le i \le N-1, \lceil\frac{i}{2}\rceil+1 \le j \le \min\{2i-1, N-1\}\),
    \begin{equation}
      D_h^2 {P_{j-i}^\theta}(x_{i}) \le C h^2 \frac{|y_j^\theta - x_i|^{1-\alpha}}{\Gamma(2-\alpha)} x_i^{\alpha/2-2-2/r} h_j
    \end{equation}
    where \(y_j^\theta = \theta x_{j-1} + (1-\theta) x_{j}\)
  \end{lemma}
  \begin{proof}
    Since \cref{lmm:Dh2simd2}
    \begin{equation}
      D_h^2 {P_{j-i}^\theta}(x_i) = {P_{j-i}^\theta}''(\xi), \quad \xi\in (x_{i-1}, x_{i+1})
    \end{equation}
    From \eqref{def:Pj-itheta-jlN}, using Leibniz formula and \cref{lmm:estimatesofhj-i3}, \cref{lmm:estimatesofu2yj-itheta} and \cref{lmm:estimatesofy-xi1-a}
  \end{proof}
}

\textcolor{blue}{
  \begin{lemma} \label{lmm:dQj-itle}
    There exists a constant \(C=C(T, \alpha, r, \|u\|_{\beta+\alpha}^{(-\alpha/2)})\) such that for \(3\le i \le N-1\). \\
    For \(\lceil\frac{i}{2}\rceil \le j \le \min\{2i-1, N-1\}\),
    \begin{equation}
      \begin{aligned}
        \frac{2}{h_{i} + h_{i+1}} & \left( \frac{{Q_{j-i}^\theta}(x_{i+1}) u'''(\eta_{j+1}^\theta) - {Q_{j-i}^\theta}(x_{i}) u'''(\eta_{j}^\theta)}{h_{i+1}}\right) \\
                                  & \le C h^2 \frac{|y_j^\theta - x_i|^{1-\alpha}}{\Gamma(2-\alpha)} x_i^{\alpha/2-2-2/r} h_j
      \end{aligned}
    \end{equation}
    And for \(\lceil\frac{i}{2}\rceil+1 \le j \le \min\{2i, N\}\),
    \begin{equation}
      \begin{aligned}
        \frac{2}{h_{i} + h_{i+1}} & \left( \frac{{Q_{j-i}^\theta}(x_{i}) u'''(\eta_{j}^\theta) - {Q_{j-i}^\theta}(x_{i-1}) u'''(\eta_{j-1}^\theta)}{h_{i}}\right) \\
                                  & \le C h^2 \frac{|y_j^\theta - x_i|^{1-\alpha}}{\Gamma(2-\alpha)} x_i^{\alpha/2-2-2/r} h_j
      \end{aligned}
    \end{equation}
    where \(\eta_{j}^\theta \in (x_{j-1}, x_{j})\).
  \end{lemma}
}
proof see \cref{prf:dQj-itle}



\textcolor{blue}{
  \begin{lemma} \label{lmm:Vijleh2xma22r}
    There exists a constant \(C=C(T, \alpha, r, \|u\|_{\beta+\alpha}^{(-\alpha/2)})\)  such that for \(3\le i \le N-1, \lceil\frac{i}{2}\rceil+1 \le j \le \min\{2i-1, N-1\}\),
    \begin{equation} \label{eq:estimateVij-int}
      \begin{aligned}
        V_{ij} & \le C h^2 \int_{0}^{1}  \frac{|y_j^\theta - x_i|^{1-\alpha}}{\Gamma(2-\alpha)} x_i^{\alpha/2-2-2/r} h_j d\theta \\
               & =  C h^2 \int_{x_{j-1}}^{x_{j}}  \frac{|y - x_i|^{1-\alpha}}{\Gamma(2-\alpha)} x_i^{\alpha/2-2-2/r} dy
      \end{aligned}
    \end{equation}
  \end{lemma}
}
\begin{proof}
  Since \cref{lmm:Vij-int}, by \cref{lmm:d2Pj-itle} and \cref{lmm:dQj-itle}, we get the result immediately.
\end{proof}


Now we can prove \cref{Ri-I3-2lN/2} using \cref{lmm:Vijleh2xma22r}, \(k=\lceil\frac{i}{2}\rceil\)
\begin{equation}
  \begin{aligned}
    I_3 & = \sum_{k+1}^{2i-1} V_{ij}
    \le C h^2 \int_{x_{k}}^{x_{2i-1}} \frac{|y - x_i|^{1-\alpha}}{\Gamma(2-\alpha)} x_i^{\alpha/2-2-2/r} dy                                                   \\
        & = C h^2 \left( \frac{|x_{k} - x_i|^{2-\alpha}}{\Gamma(3-\alpha)} + \frac{|x_{2i-1} - x_i|^{2-\alpha}}{\Gamma(3-\alpha)}\right) x_i^{\alpha/2-2-2/r} \\
        & \le C h^2 x_{i}^{2-\alpha} x_i^{\alpha/2-2-2/r} = C h^2 x_i^{-\alpha/2-2/r}
  \end{aligned}
\end{equation}

Now we study \(I_2, I_4\).

  \begin{lemma} \label{lmm:Ri-I2-I4-ilN/2}
    There exists a constant \(C=C(T, \alpha, r, \|u\|_{\beta+\alpha}^{(-\alpha/2)})\) such that for \(3\le i \le N-1, k=\lceil\frac{i}{2}\rceil\),
    \begin{equation}
      I_2 = \frac{2}{h_i + h_{i+1}}
      \left( \frac{1}{h_{i+1}} (T_{i+1, k} +  T_{i+1, k+1})
      - (\frac{1}{h_{i}}+\frac{1}{h_{i+1}}) T_{i,k} \right) \le C h^2 x_i^{-\alpha/2-2/r}
    \end{equation}
    And for \(3\le i < N/2\),
    \begin{equation}
      I_4 = \frac{2}{h_i + h_{i+1}}
      \left( \frac{1}{h_{i}} (T_{i-1, 2i} +  T_{i-1, 2i-1})
      - (\frac{1}{h_{i}}+\frac{1}{h_{i+1}}) T_{i,2i} \right) \le C h^2 x_i^{-\alpha/2-2/r}
    \end{equation}
  \end{lemma}
  \begin{proof}
    In fact,
    \begin{equation}
      \begin{aligned}
         & \frac{1}{h_{i+1}} (T_{i+1, k} +  T_{i+1, k+1})
        - (\frac{1}{h_{i}}+\frac{1}{h_{i+1}}) T_{i,k}                                                                                                  \\
         & = \frac{1}{h_{i+1}} (T_{i+1, k} -  T_{i, k}) + \frac{1}{h_{i+1}} (T_{i+1, k+1} -  T_{i, k}) + (\frac{1}{h_{i+1}} - \frac{1}{h_{i}}) T_{i,k}
      \end{aligned}
    \end{equation}
    While, by \cref{lmm:Dyj} and \cref{lmm:hilexi}
    \begin{equation}
      \begin{aligned}
        \frac{1}{h_{i+1}} (T_{i+1, k} -  T_{i, k})
         & = \int_{x_{k-1}}^{x_k} (u(y)-\Pi_hu(y)) \frac{|x_{i+1}-y|^{1-\alpha} - |x_i-y|^{1-\alpha}}{h_{i+1}\Gamma(2-\alpha)} dy \\
         & \le h_k^2 \textcolor{red}{\max_{\eta\in (x_{k-1}, x_k)}|u''(\eta)|} \int_{x_{k-1}}^{x_k} \frac{|\xi-y|^{-\alpha}}{\Gamma(1-\alpha)} dy, \quad \xi\in(x_i, x_{i+1})    \\
         & \le C  h^2 x_k^{2-2/r} x_{k-1}^{\alpha/2-2} \; h_k|x_i-x_{k}|^{-\alpha}                                                \\
         & \le C  h^2 x_i^{-\alpha/2-2/r} h_k
      \end{aligned}
    \end{equation}
    Thus,
    \begin{equation}
      \frac{2}{h_i + h_{i+1}} \frac{1}{h_{i+1}} |T_{i+1, k} -  T_{i, k}| \le C h^2 x_i^{-\alpha/2-2/r}
    \end{equation}
    From \cref{lmm:Tij-express-as-int-of-function}
    \begin{equation}
      \begin{aligned}
        \frac{1}{h_{i+1}} (T_{i+1, k+1} -  T_{i, k})
         & = \int_{0}^{1} -\frac{\theta(1-\theta)}{2} \frac{P_{k-i}^\theta(x_{i+1}) - P_{k-i}^\theta(x_i)}{h_{i+1}} d\theta                                                                   \\
         & \quad + \int_{0}^1 \frac{\theta^3 (1-\theta)}{3!} \frac{{Q_{k-i}^\theta}(x_{i+1}) u'''(\eta_{k+1,1}^\theta) - {Q_{k-i}^\theta}(x_{i}) u'''(\eta_{k,1}^\theta)}{h_{i+1}}  d\theta   \\
         & \quad -  \int_{0}^1 \frac{\theta (1-\theta)^3}{3!}  \frac{{Q_{k-i}^\theta}(x_{i+1}) u'''(\eta_{k+1,2}^\theta) - {Q_{k-i}^\theta}(x_{i}) u'''(\eta_{k,2}^\theta)}{h_{i+1}}  d\theta
      \end{aligned}
    \end{equation}
    and
    \begin{equation}
      D_h P_{k-i}^\theta(x_i) := \frac{P_{k-i}^\theta(x_{i+1}) - P_{k-i}^\theta(x_i)}{h_{i+1}} = {P_{k-i}^\theta}'(\xi), \quad \xi \in (x_{i}, x_{i+1})
    \end{equation}
    Similar with \cref{lmm:d2Pj-itle}, from \cref{lmm:Tij-express-as-int-of-function}, using Leibniz formula, by \cref{lmm:dhj-i3le-Ch2xi1-2r}, \cref{lmm:estimatesofu2yj-itheta} and \cref{lmm:estimatesofy-xi1-a} we get
    \begin{equation}
      |D_h P_{k-i}^\theta(x_i)| \le C h^2 x_i^{-\alpha/2-2/r} h_k
    \end{equation}
    And with \cref{lmm:dQj-itle}, we can get
    \begin{equation}
      \frac{2}{h_i + h_{i+1}}  \frac{1}{h_{i+1}} |T_{i+1, k+1} -  T_{i, k}| \le C h^2 x_i^{-\alpha/2-2/r}
    \end{equation}
    For the third term, by \cref{lmm:hilexi}, \cref{lmm:hi1-hi} and \cref{lmm:Dyj}
    \begin{equation}
      \begin{aligned}
        \frac{2}{h_i + h_{i+1}} \frac{h_{i+1}-h_{i}}{h_i h_{i+1}} T_{i,k}
         & \le h_i^{-3} h^2 x_i^{1-2/r}  h_k C h_k^2 x_{k-1}^{\alpha/2-2} |x_{k}-x_i|^{1-\alpha} \\
         & \le C h^2 x_i^{-\alpha/2-2/r}
      \end{aligned}
    \end{equation}
    Summarizes, we have
    \begin{equation}
      I_2 \le C h^2 x_i^{-\alpha/2-2/r}
    \end{equation}
    The case for \(I_4\) is similar.
  \end{proof}


Now combine \cref{lmm:R1R2}, \cref{lmm:Ri-I1}, \cref{lmm:Ri-I2-I4-ilN/2}, \cref{Ri-I3-2lN/2}, \cref{lmm:sumSi-ilessN/2-2ip1-N} and \cref{lmm:sumSi-ilessN/2-Np1-2N}, we get \cref{thm:Ri-ilessN/2}.

% For \(3\le i < N/2\)
% \begin{equation}
%   \begin{aligned}
%     R_i & = I_1 + I_2 + I_3 + I_4 + I_5 \\
%         & \le C h^2 x_i^{-\alpha/2-2/r}
%     + \begin{cases}
%         C h^2 x_i^{-\alpha/2-2/r} ,              & r\alpha/2 + r -2 > 0 \\
%         C h^2 (x_i^{-1-\alpha} \ln(i) + \ln(N)), & r\alpha/2 + r -2 = 0 \\
%         C h^{r\alpha/2+r} x_i^{-1-\alpha},       & r\alpha/2 + r -2 < 0
%       \end{cases}
%   \end{aligned}
% \end{equation}

% Combine with \(i=1, 2\), 
% we get for \(1\le i< N/2\)
% \begin{equation}
%   R_i \le \begin{cases}
%     C h^2 x_i^{-\alpha/2-2/r} ,              & r\alpha/2 + r -2 > 0 \\
%     C h^2 (x_i^{-1-\alpha} \ln(i) + \ln(N)), & r\alpha/2 + r -2 = 0 \\
%     C h^{r\alpha/2+r} x_i^{-1-\alpha},       & r\alpha/2 + r -2 < 0
%   \end{cases}
% \end{equation}






\subsection{Proof of \cref{thm:Ri-N/2le-i-leN}}

For \(N/2\le i < N, k=\lceil\frac{i}{2}\rceil\), we have
\begin{equation}
  \begin{aligned}
    R_i
    = & \sum_{j=1}^{2N} \frac{2}{h_i + h_{i+1}}
    \left( \frac{1}{h_{i+1}} T_{i+1, j}
    - (\frac{1}{h_{i}}+\frac{1}{h_{i+1}}) T_{i,j}
    +  \frac{1}{h_{i}} T_{i-1, j} \right)                                                   \\
    = & \sum_{j=1}^{k-1} \frac{2}{h_i + h_{i+1}}
    \left( \frac{1}{h_{i+1}} T_{i+1, j}
    - (\frac{1}{h_{i}}+\frac{1}{h_{i+1}}) T_{i,j}
    +  \frac{1}{h_{i}} T_{i-1, j} \right)                                                   \\
      & + \frac{2}{h_i + h_{i+1}}
    \left( \frac{1}{h_{i+1}} (T_{i+1, k} +  T_{i+1, k+1})
    - (\frac{1}{h_{i}}+\frac{1}{h_{i+1}}) T_{i,k} \right)                                   \\
      & + \sum_{j=k+1}^{N-1} + \sum_{j=N}^{N+1} + \sum_{j=N+2}^{2N-\lceil\frac{N}{2}\rceil}
    \frac{2}{h_i + h_{i+1}}
    \left( \frac{1}{h_{i+1}} T_{i+1, j+1}
    - (\frac{1}{h_{i}}+\frac{1}{h_{i+1}}) T_{i,j}
    +  \frac{1}{h_{i}} T_{i-1, j-1} \right)                                                 \\
      & + \frac{2}{h_i + h_{i+1}}
    \left( \frac{1}{h_{i}} (T_{i-1, 2N-\lceil\frac{N}{2}\rceil+1} +  T_{i-1, 2N-\lceil\frac{N}{2}\rceil})
    - (\frac{1}{h_{i}}+\frac{1}{h_{i+1}}) T_{i,2N-\lceil\frac{N}{2}\rceil+1} \right)        \\
      & + \sum_{j=2N-\lceil\frac{N}{2}\rceil+2}^{2N} \frac{2}{h_i + h_{i+1}}
    \left( \frac{1}{h_{i+1}} T_{i+1, j}
    - (\frac{1}{h_{i}}+\frac{1}{h_{i+1}}) T_{i,j}
    +  \frac{1}{h_{i}} T_{i-1, j} \right)                                                   \\
    = & I_1 + I_2 + I_3^1 + I_3^2 + I_3^3 + I_4 + I_5
  \end{aligned}
\end{equation}

We have estimate \(I_1\) in \cref{lmm:Ri-I1} and  \(I_2\) in \cref{lmm:Ri-I2-I4-ilN/2}.
We can control \(I_3^1\) similar with \cref{Ri-I3-2lN/2} by \cref{lmm:Vijleh2xma22r} where \(2i-1 \ge N-1\)
\begin{lemma} \label{lmm:Ri-I3-i<N}
  There exists a constant \(C=C(T, \alpha, r, \|u\|_{\beta+\alpha}^{(-\alpha/2)})\) such that for \(N/2\le i < N, k=\lceil\frac{i}{2}\rceil\),
  \begin{equation}
    \begin{aligned}
      I_3^1 & = \sum_{j=k+1}^{N-1} V_{ij} \le C h^2 \int_{x_{k}}^{x_{N-1}} \frac{|y-x_i|^{1-\alpha}}{\Gamma(2-\alpha)} x_i^{\alpha/2-2-2/r} dy                   \\
          & = C h^2 \left( \frac{|x_{k} - x_i|^{2-\alpha}}{\Gamma(3-\alpha)} + \frac{|x_{N-1} - x_i|^{2-\alpha}}{\Gamma(3-\alpha)}\right) x_i^{\alpha/2-2-2/r} \\
          & \le C h^2 x_{i}^{2-\alpha} x_i^{\alpha/2-2-2/r} = C h^2 x_i^{-\alpha/2-2/r}
    \end{aligned}
  \end{equation}
\end{lemma}


Let's study \(I_3^3\) before \(I_3^2\).

\begin{equation}
  I_3^3 = \sum_{j=N+2}^{2N-\lceil\frac{N}{2}\rceil} V_{ij}
\end{equation}
Similarly, Let's define a new series of functions
\begin{definition} \label{def:yi2j-jgN}
  For \(i\le N-1, j\ge N+1\), with no confusion, we also denote in this section
  \textcolor{blue}{
  \begin{equation}
    y_{j-i}(x) = 2T - (Z_{2N-j+i} - x^{1/r})^r, \quad Z_{2N-j+i} = T^{1/r} \frac{2N-j+i}{N}
  \end{equation}
  Particularly
  \begin{equation*}
    y_{j-i}(x_{i-1}) = x_{j-1}, \quad y_{j-i}(x_{i}) = x_{j}, \quad y_{j-i}(x_{i+1}) = x_{j+1}
  \end{equation*}
  \textcolor{red}{\(y\to z\)?}
  }
  \begin{gather} \label{eq:dy-j>N}
    {y_{j-i}}'(x) = (2T-y_{j-i}(x))^{1-1/r} x^{1/r-1}  \\
    {y_{j-i}}''(x) = \frac{1-r}{r} (2T-y_{j-i}(x))^{1-2/r} x^{1/r-2} Z_{2N-j+i}  \\
  \end{gather}
  \begin{equation}
    y_{j-i}^\theta(x) = (1-\theta) y_{j-i-1}(x) + \theta y_{j-i}(x)
  \end{equation}
  \begin{equation}
    h_{j-i}(x) = y_{j-i}(x) - y_{j-i-1}(x)
  \end{equation}
  \begin{equation}
    {P_{j-i}^\theta}(x) = ({h_{j-i}}(x))^3 u''({y_{j-i}^\theta}(x)) \frac{|{y_{j-i}^\theta}(x)-x|^{1-\alpha}}{\Gamma(2-\alpha)}
  \end{equation}
  \begin{equation}
    {Q_{j-i}^\theta}(x) = ({h_{j-i}}(x))^4 \frac{|{y_{j-i}^\theta}(x)-x|^{1-\alpha}}{\Gamma(2-\alpha)}
  \end{equation}
\end{definition}

Now we have the same formula \cref{lmm:Vij-int} for \(i\le N-1, j \ge N+2\),
% \begin{equation} \label{eq:Vij-j>N}
%   \begin{aligned}
%     V_{ij} & = \frac{2}{h_{i} + h_{i+1}}  \left( \frac{1}{h_{i+1}} T_{i+1, j+1} - \left(\frac{1}{h_{i}}+\frac{1}{h_{i+1}}\right)  T_{i,j} + \frac{1}{h_{i}} T_{i-1, j-1} \right)                                                      \\
%            & = \int_{0}^{1} -\frac{\theta (1-\theta)}{2} D_h^2 P_{j-i}^\theta(x_i)  d\theta                                                                                                                                           \\
%            & \quad +  \int_{0}^1 \frac{\theta^3 (1-\theta)}{3!} \frac{2}{h_{i} + h_{i+1}}\left( \frac{{Q_{j-i}^\theta}(x_{i+1}) u'''(\eta_{j+1,1}^\theta) - {Q_{j-i}^\theta}(x_{i}) u'''(\eta_{j,1}^\theta)}{h_{i+1}}\right)  d\theta \\
%            & \quad -  \int_{0}^1 \frac{\theta^3 (1-\theta)}{3!} \frac{2}{h_{i} + h_{i+1}}\left( \frac{{Q_{j-i}^\theta}(x_{i}) u'''(\eta_{j,1}^\theta) - {Q_{j-i}^\theta}(x_{i-1}) u'''(\eta_{j-1,1}^\theta)}{h_{i}}\right)  d\theta   \\
%            & \quad -  \int_{0}^1 \frac{\theta (1-\theta)^3}{3!} \frac{2}{h_{i} + h_{i+1}}\left( \frac{{Q_{j-i}^\theta}(x_{i+1}) u'''(\eta_{j+1,2}^\theta) - {Q_{j-i}^\theta}(x_{i}) u'''(\eta_{j,2}^\theta)}{h_{i+1}}\right)  d\theta \\
%            & \quad +  \int_{0}^1 \frac{\theta (1-\theta)^3}{3!} \frac{2}{h_{i} + h_{i+1}}\left( \frac{{Q_{j-i}^\theta}(x_{i}) u'''(\eta_{j,2}^\theta) - {Q_{j-i}^\theta}(x_{i-1}) u'''(\eta_{j-1,2}^\theta)}{h_{i}}\right)  d\theta
%   \end{aligned}
% \end{equation}

Similarly, we first estimate
\begin{equation}
  D_h^2 P_{j-i}^\theta(\xi) = {P_{j-i}^\theta}''(\xi), \quad \xi\in (x_{i-1}, x_{i+1})
\end{equation}


\textcolor{blue}{
  Combine \cref{def:yi2j-jgN}, \cref{lmm:estimate-hj-i-j>N}, \cref{lmm:estimate-d2uj-i-j>N} and \cref{lmm:estimate-|yj-i-xi|^1-a-j>N}, using Leibniz formula, we have
  \begin{lemma} \label{lmm:estimate-Pij-theta-j>N}
    There exists a constant \(C=C(T, \alpha, r, \|u\|_{\beta+\alpha}^{(-\alpha/2)})\) such that
    For \(N/2\le i \le N-1\), \(N+2 \le j \le 2N-\lceil\frac{N}{2}\rceil+1\) ,,  we have
    \begin{equation}
      \begin{aligned}
        |D_h^2 P_{j-i}^\theta(\xi)| & \le C h_j h^2 \Big(|y_j^\theta - x_i|^{1-\alpha}                            \\
                                  & \quad + |y_j^\theta - x_i|^{-\alpha}(|2T - x_i - y_j^\theta| + h_N)     \\
                                  & \quad + |y_j^\theta - x_i|^{-1-\alpha}(|2T - x_i - y_j^\theta| + h_N)^2 \\
                                  & \quad + (r-1) |y_j^\theta - x_i|^{-\alpha}
        \Big)
      \end{aligned}
    \end{equation}
  \end{lemma}
}

And
\textcolor{blue}{
  \begin{lemma} \label{lmm:estimate-Qij-theta-j>N}
    There exists a constant \(C=C(T, \alpha, r, \|u\|_{\beta+\alpha}^{(-\alpha/2)})\) such that
    For \(N/2\le i \le N-1\), \(N+2 \le j \le 2N-\lceil\frac{N}{2}\rceil\), \(\xi\in (x_{i-1}, x_{i+1})\) , we have
    \begin{equation}
      \begin{aligned}
        \frac{2}{h_{i} + h_{i+1}} & \left| \frac{{Q_{j-i}^\theta}(x_{i+1}) u'''(\eta_{j+1}^\theta) - {Q_{j-i}^\theta}(x_{i}) u'''(\eta_{j}^\theta)}{h_{i+1}}\right| \\
                                  & \le C h^2 h_j \left( |y_j^\theta - x_i|^{1-\alpha}  + |y_j^\theta - x_i|^{-\alpha}(|2T - x_i - y_j^\theta| + h_N) \right)
      \end{aligned}
    \end{equation}
    and
    \begin{equation}
      \begin{aligned}
        \frac{2}{h_{i} + h_{i+1}} & \left( \frac{{Q_{j-i}^\theta}(x_{i}) u'''(\eta_{j}^\theta) - {Q_{j-i}^\theta}(x_{i-1}) u'''(\eta_{j-1}^\theta)}{h_{i+1}}\right) \\
                                  & \le C h^2 h_j ( |y_j^\theta - x_i|^{1-\alpha}  + |y_j^\theta - x_i|^{-\alpha}(|2T - x_i - y_j^\theta| + h_N) )
      \end{aligned}
    \end{equation}
  \end{lemma}
}
\begin{proof} \label{prf:estimate-Qij-theta-j>N}
  From \cref{def:yi2j-jgN}, by \cref{lmm:estimate-hj-i-j>N} and \cref{lmm:estimate-|yj-i-xi|^1-a-j>N},
  for \(\xi \in (x_{i}, x_{i+1})\), by Leibniz formula, we have
  \begin{equation}
    \left|{Q_{j-i}^\theta}'(\xi)\right| \le C h^2 h_j^2 ((r-1)|y_j^\theta - x_i|^{1-\alpha}  + |y_j^\theta - x_i|^{-\alpha}(|2T - x_i - y_j^\theta| + h_N))
  \end{equation}
  \begin{equation}
    \left|Q_{j-i}^\theta(\xi)\right| \le C h^2 h_j^2 |y_j^\theta - x_i|^{1-\alpha}
  \end{equation}
  So use the skill in \cref{prf:dQj-itle} with \cref{lmm:estimate-d2uj-i-j>N}
  \begin{equation}
    \begin{aligned}
      \frac{2}{h_{i} + h_{i+1}} & \left( \frac{{Q_{j-i}^\theta}(x_{i+1}) u'''(\eta_{j+1}^\theta) - {Q_{j-i}^\theta}(x_{i}) u'''(\eta_{j}^\theta)}{h_{i+1}}\right) \\
                                & \le C h^2 h_j ( |y_j^\theta - x_i|^{1-\alpha}  + |y_j^\theta - x_i|^{-\alpha}(|2T - x_i - y_j^\theta| + h_N) )                  \\
    \end{aligned}
  \end{equation}
\end{proof}



Combine \cref{lmm:estimate-Pij-theta-j>N}, \cref{lmm:estimate-Qij-theta-j>N} and formula \cref{lmm:Vij-int} for \(i\le N-1, j \ge N+2\), we have
\begin{lemma}
  There exists a constant \(C=C(T, \alpha, r, \|u\|_{\beta+\alpha}^{(-\alpha/2)})\) such that
  For \(N/2\le i \le N-1\), \(N+2 \le j \le 2N-\lceil\frac{N}{2}\rceil+1\)
  \begin{equation}
    \begin{aligned}
      V_{ij} \le & C h^2 \int_{x_{j-1}}^{x_{j}} \Big(  |y-x_i|^{1-\alpha} \\
                 & + |y-x_i|^{-\alpha}(|2T - x_i - y| + h_N)
      + |y-x_i|^{-1-\alpha}(|2T - x_i - y| + h_N)^2                 \\
                 & + (r-1) |y-x_i|^{-\alpha}  \Big) dy
    \end{aligned}
  \end{equation}
\end{lemma}




\textcolor{blue}{
  We can esitmate \(I_3^3\) Now.
  \begin{lemma} \label{thm:Ri-I5-i<N}
    There exists a constant \(C=C(T, \alpha, r, \|u\|_{\beta+\alpha}^{(-\alpha/2)})\) such that
    For \(N/2\le i \le N-1\),  we have
    \begin{equation}
      I_3^3 = \sum_{j=N+2}^{2N-\lceil\frac{N}{2}\rceil} V_{ij} \le C h^2 + C(r-1) h^2 \textcolor{red}{|T-x_{i-1}|^{1-\alpha}}
    \end{equation}
  \end{lemma}
}
\begin{proof}
  \begin{equation}
    \begin{aligned}
      I_3^3 = & \sum_{j=N+2}^{2N-\lceil\frac{N}{2}\rceil} V_{ij}                                   \\
      \le   & C h^2 \int_{x_{N+1}}^{x_{2N-\lceil\frac{N}{2}\rceil}}
      \Big(|y-x_i|^{1-\alpha} \\
            & + |y-x_i|^{-\alpha}(|2T - x_i - y| + h_N)
      + |y-x_i|^{-1-\alpha}(|2T - x_i - y| + h_N)^2   \\
            & + (r-1) |y-x_i|^{-\alpha}\Big) dy    \\
    \end{aligned}
  \end{equation}
  % For \(x_{N+1}\le y \le x_{2N-i}=2T-x_i\),
  % \begin{equation}
  %   y-x_i \simeq T-x_{i} \simeq T-x_{i-1}
  % \end{equation}
  % and
  % \begin{equation}
  %   2T-x_{i} - y + h_N \le 2T-x_i - x_{N+1} + h_N = T-x_i \le T-x_{i-1}
  % \end{equation}
  % So
  % \begin{equation}
  %   \begin{aligned}
  %     J_1 & \le C h^2 (x_{2N-i} - x_{N+1}) (|T-x_{i-1}|^{1-\alpha} + (r-1)|T-x_{i-1}|^{-\alpha}) \\
  %         & \le C h^2 (|T-x_{i-1}|^{2-\alpha} + (r-1)|T-x_{i-1}|^{1-\alpha})                     \\
  %         & \le C h^2 + C(r-1)  h^2 |T-x_{i-1}|^{1-\alpha}
  %   \end{aligned}
  % \end{equation}
  % Otherwise, when \(x_{2N-i} \le y \le x_{2N-\lceil\frac{N}{2}\rceil}\)
  Since
  \begin{equation}
    |2T - x_i - y| + h_{N} \le y-x_i
  \end{equation}
  \begin{equation}
    \begin{aligned}
      I_3^3 & \le C h^2 \int_{x_{N+1}}^{x_{2N-\lceil\frac{N}{2}\rceil}} |y-x_i|^{1-\alpha} + (r-1)|y-x_i|^{-\alpha}    \\
          & \le C h^2 ( T^{2-\alpha} + (r-1)|x_{N+1}-x_i|^{1-\alpha})                             \\
          & \le C h^2 + C(r-1) h^2 |T-x_{i-1}|^{1-\alpha}
    \end{aligned}
  \end{equation}
  % Summarizes two cases, we get the result.
\end{proof}













\newpage
For \(I_3^2\), we have
\textcolor{red}{
  \begin{theorem} \label{thm:Ri-I4-i<N}
    There exists a constant \(C=C(T, \alpha, r, \|u\|_{\beta+\alpha}^{(-\alpha/2)})\) such that, for \(N/2\le i\le N-1\)
    \begin{equation}
      \begin{aligned}
        V_{iN} & = \frac{2}{h_i + h_{i+1}}
        \left( \frac{1}{h_{i+1}} T_{i+1, N+1}
        - (\frac{1}{h_{i}}+\frac{1}{h_{i+1}}) T_{i,N}
        +  \frac{1}{h_{i}} T_{i-1, N-1} \right)                   \\
               & \le C h^2 + C (r-1) h^2 |T - x_{i-1}|^{1-\alpha}
      \end{aligned}
    \end{equation}
  \end{theorem}
}
\begin{proof}
  We use the similar skill in the last section, but more complicated.
  for \(j=N\),
  Let
  \begin{equation}
    {_L}y_{N-1-i}(x) = (x^{1/r} + Z_{N-1-i})^{r}, \quad Z_{N-1-i} = T^{1/r}\frac{N-1-i}{N}
  \end{equation}
  \begin{equation}
    {_0}y_{N-i}(x) = \frac{x^{1/r} - Z_i}{Z_1} h_N + T, \quad Z_i = T^{1/r}\frac{i}{N}, x_N = T
  \end{equation}
  and
  \begin{equation}
    {_R}y_{N+1-i}(x) = 2T - (Z_{N-1+i} - x^{1/r})^r, \quad Z_{N-1+i} = T^{1/r}\frac{N-1+i}{N}
  \end{equation}
  Thus,
  \begin{gather*}
    {_L}y_{N-1-i}(x_{i-1}) = x_{N-2}, \quad  {_L}y_{N-1-i}(x_{i}) = x_{N-1}, \quad {_L}y_{N-1-i}(x_{i+1}) = x_{N} \\
    {_0}y_{N-i}(x_{i-1}) = x_{N-1}, \quad  {_0}y_{N-i}(x_{i}) = x_{N}, \quad {_0}y_{N-i}(x_{i+1}) = x_{N+1} \\
    {_R}y_{N+1-i}(x_{i-1}) = x_{N}, \quad  {_R}y_{N+1-i}(x_{i}) = x_{N+1}, \quad {_R}y_{N+1-i}(x_{i+1}) = x_{N+2}
  \end{gather*}
  Then, define
  \begin{gather}
    {_L}y_{N-i}^\theta(x) = \theta {_L}y_{N-1-i}(x) + (1-\theta) {_0}y_{N-i}(x)   \\
    {_R}y_{N+1-i}^\theta(x) = \theta {_0}y_{N-i}(x) + (1-\theta) {_R}y_{N+1-i}(x)
  \end{gather}
  \begin{gather}
    {_L}h_{N-i}(x) = {_0}y_{N-i}(x) - {_L}y_{N-1-i}(x)   \\
    {_R}h_{N+1-i}(x) = {_R}y_{N+1-i}(x) - {_0}y_{N-i}(x)
  \end{gather}
  We have
  \begin{gather}
    {{_L}y_{N-1-i}}'(x) = {_L}y_{N-1-i}^{1-1/r}(x) x^{1/r-1}  \\
    {{_L}y_{N-1-i}}''(x) = \frac{1-r}{r} {_L}y_{N-1-i}^{1-2/r}(x) x^{1/r-2} Z_{N-1-i}  \\
    {{_0}y_{N-i}}'(x) = \frac{1}{r}\frac{h_N}{Z_1} x^{1/r-1}  \\
    {{_0}y_{N-i}}''(x) = \frac{1-r}{r^2}\frac{h_N}{Z_1} x^{1/r-2} \\
    {{_R}y_{N+1-i}}'(x) = (2T-{_R}y_{N+1-i}(x))^{1-1/r} x^{1/r-1}  \\
    {{_R}y_{N+1-i}}''(x) = \frac{1-r}{r} (2T-{_R}y_{N+1-i}(x))^{1-2/r} x^{1/r-2} Z_{N-1+i}
  \end{gather}
  \begin{gather}
    {_L}P_{N-i}^\theta(x) = ({_L}h_{N-i}(x))^3 \frac{|{_L}y_{N-i}^\theta(x)-x|^{1-\alpha}}{\Gamma(2-\alpha)} u''({_L}y_{N-i}^\theta(x)) \\
    {_R}P_{N+1-i}^\theta(x) = ({_R}h_{N+1-i}(x))^3 \frac{|{_R}y_{N+1-i}^\theta(x)-x|^{1-\alpha}}{\Gamma(2-\alpha)} u''({_R}y_{N+1-i}^\theta(x))
  \end{gather}

  \begin{gather}
    {_L}Q_{N-i}^\theta(x) = ({_L}h_{N-i}(x))^4 \frac{|{_L}y_{N-i}^\theta(x)-x|^{1-\alpha}}{\Gamma(2-\alpha)}  \\
    {_R}Q_{N+1-i}^\theta(x) = ({_R}h_{N+1-i}(x))^4 \frac{|{_R}y_{N+1-i}^\theta(x)-x|^{1-\alpha}}{\Gamma(2-\alpha)}
  \end{gather}
  Similar with \cref{lmm:Tij-express-as-int-of-function}, we can get for \(l=-1,0,1\),
  \begin{equation}
    \begin{aligned}
      T_{i+l, N+l} & = \int_{0}^{1} -\frac{\theta (1-\theta)}{2} {{_L}P_{N-i}^\theta}(x_{i+l}) d\theta                                                                                 \\
                   & + \int_{0}^{1} \frac{\theta (1-\theta)}{3!}{{_L}Q_{N-i}^\theta}(x_{i+l})( \theta^2  u'''(\eta_{N+l, 1}^\theta) - (1-\theta)^2 u'''(\eta_{N+l, 2}^\theta)) d\theta
    \end{aligned}
  \end{equation}
  \begin{equation}
    \begin{aligned}
      T_{i+l, N+1+l} & = \int_{0}^{1} -\frac{\theta (1-\theta)}{2} {{_R}P_{N+1-i}^\theta}(x_{i+l}) d\theta                                                                                     \\
                     & + \int_{0}^{1} \frac{\theta (1-\theta)}{3!}{{_R}Q_{N+1-i}^\theta}(x_{i+l})( \theta^2  u'''(\eta_{N+1+l, 1}^\theta) - (1-\theta)^2 u'''(\eta_{N+1+l, 2}^\theta)) d\theta
    \end{aligned}
  \end{equation}
  So we have
  \begin{equation}
    \begin{aligned}
      V_{i,N}
       & = \int_{0}^{1} -\frac{\theta (1-\theta)}{2} D_h^2 {_L}P_{N-i}^\theta(x_i)   d\theta                                                                                                                                             \\
       & \quad +  \int_{0}^1 \frac{\theta^3 (1-\theta)}{3!} \frac{2}{h_{i} + h_{i+1}}\left( \frac{{{_L}Q_{N-i}^\theta}(x_{i+1}) u'''(\eta_{N+1,1}^\theta) - {{_L}Q_{N-i}^\theta}(x_{i}) u'''(\eta_{N,1}^\theta)}{h_{i+1}}\right)  d\theta \\
       & \quad -  \int_{0}^1 \frac{\theta^3 (1-\theta)}{3!} \frac{2}{h_{i} + h_{i+1}}\left( \frac{{{_L}Q_{N-i}^\theta}(x_{i}) u'''(\eta_{N,1}^\theta) - {{_L}Q_{N-i}^\theta}(x_{i-1}) u'''(\eta_{N-1,1}^\theta)}{h_{i}}\right)  d\theta   \\
       & \quad -  \int_{0}^1 \frac{\theta (1-\theta)^3}{3!} \frac{2}{h_{i} + h_{i+1}}\left( \frac{{{_L}Q_{N-i}^\theta}(x_{i+1}) u'''(\eta_{N+1,2}^\theta) - {{_L}Q_{N-i}^\theta}(x_{i}) u'''(\eta_{N,2}^\theta)}{h_{i+1}}\right)  d\theta \\
       & \quad +  \int_{0}^1 \frac{\theta (1-\theta)^3}{3!} \frac{2}{h_{i} + h_{i+1}}\left( \frac{{{_L}Q_{N-i}^\theta}(x_{i}) u'''(\eta_{N,2}^\theta) - {{_L}Q_{N-i}^\theta}(x_{i-1}) u'''(\eta_{N-1,2}^\theta)}{h_{i}}\right)  d\theta   \\
      % & \quad +  \int_{0}^1 \frac{\theta^3 (1-\theta)}{3!} \frac{2}{h_{i} + h_{i+1}}\left( \frac{{{_R}Q_{N+1-i}^\theta}(x_{i+1}) u'''(\eta_{N+2,1}^\theta) - {{_R}Q_{N+1-i}^\theta}(x_{i}) u'''(\eta_{N+1,1}^\theta)}{h_{i+1}}\right)  d\theta \\
      % & \quad -  \int_{0}^1 \frac{\theta^3 (1-\theta)}{3!} \frac{2}{h_{i} + h_{i+1}}\left( \frac{{{_R}Q_{N+1-i}^\theta}(x_{i}) u'''(\eta_{N+1,1}^\theta) - {{_R}Q_{N+1-i}^\theta}(x_{i-1}) u'''(\eta_{N,1}^\theta)}{h_{i}}\right)  d\theta   \\
      % & \quad -  \int_{0}^1 \frac{\theta (1-\theta)^3}{3!} \frac{2}{h_{i} + h_{i+1}}\left( \frac{{{_R}Q_{N+1-i}^\theta}(x_{i+1}) u'''(\eta_{N+2,2}^\theta) - {{_R}Q_{N+1-i}^\theta}(x_{i}) u'''(\eta_{N+1,2}^\theta)}{h_{i+1}}\right)  d\theta \\
      % & \quad +  \int_{0}^1 \frac{\theta (1-\theta)^3}{3!} \frac{2}{h_{i} + h_{i+1}}\left( \frac{{{_R}Q_{N+1-i}^\theta}(x_{i}) u'''(\eta_{N+1,2}^\theta) - {{_R}Q_{N+1-i}^\theta}(x_{i-1}) u'''(\eta_{N,2}^\theta)}{h_{i}}\right)  d\theta
    \end{aligned}
  \end{equation}
  \(N+1\) is similar.

  We estimate \(D_h^2 {_L}P_{N-i}^\theta(x_i)={{_L}P_{N-i}^\theta}''(\xi) , \xi\in (x_{i-1}, x_{i+1})\),


  \textcolor{orange}{
    \begin{lemma} \label{lmm:estimate-hN-j=N}
      \begin{gather}
        {_L}h_{N-i}^3(\xi) \le C h_N^3 \le C h^3  \\
        {_R}h_{N+1-i}^3(\xi) \le C h_N^3 \le C h^3  \\
        ({_L}h_{N-i}^3(\xi))' \le C (r-1) h_N^2 h \le C (r-1) h^3  \\
        ({_R}h_{N+1-i}^3(\xi))' \le C (r-1) h_N^2 h \le C (r-1) h^3  \\
        ({_L}h_{N-i}^3(\xi))''  \le C (r-1) h^2     \\
        ({_R}h_{N+1-i}^3(\xi))''  \le C (r-1) h^2
      \end{gather}
    \end{lemma}
  }
  \begin{proof}
    \begin{equation}
      {_L}h_{N-i}(\xi) \le \textcolor{red}{2(C?)} h_N, \quad {_R}h_{N+1-i}(\xi) \le 2h_N
    \end{equation}
    \begin{equation}
      \begin{aligned}
        ({_L}h_{N-i}^l(\xi))' & = l {_L}h_{N-i}^{l-1}(\xi) ({{_0}y_{N-i}}'(\xi) - {{_L}y_{N-1-i}}'(\xi))                           \\
                             & = l {_L}h_{N-i}^{l-1}(\xi) \xi^{1/r-1} (\frac{1}{r}\frac{h_N}{Z_1}  - {_L}y_{N-1-i}^{1-1/r}(\xi)) \\
      \end{aligned}
    \end{equation}
    while
    \begin{equation}
      \begin{aligned}
        |\frac{1}{r}\frac{h_N}{Z_1}  - {_L}y_{N-1-i}^{1-1/r}(\xi)|
         & = |\frac{1}{r}\frac{x_N-(x_N^{1/r}-Z_1)^r}{Z_1} - \eta^{1-1/r}| \quad \eta \in [x_{N-2}, x_{N}] \\
         & = T^{1-1/r} |(\frac{N-t}{N})^{r-1} - (\frac{N-s}{N})^{r-1} |  \quad t\in[0, 1], s\in [0, 2]     \\
         & \le T^{1-1/r} |1- (\frac{N-2}N)^{r-1}| \le C T^{1-1/r} (r-1)\frac{2}{N}
      \end{aligned}
    \end{equation}
    Thus,
    \begin{equation}
      \begin{aligned}
        ({_L}h_{N-i}^l(\xi))' & \le C(r-1) h_N^{l-1} x_i^{1/r-1} h
      \end{aligned}
    \end{equation}

    % \begin{equation}
    %   \begin{aligned}
    %     ({_R}h_{N+1-i}^l(\xi))' & = l {_R}h_{N+1-i}^{l-1}(\xi) ({{_R}y_{N+1-i}}'(\xi) - {{_0}y_{N-i}}'(\xi))                                \\
    %                            & = l {_R}h_{N+1-i}^{l-1}(\xi) x_i^{1/r-1} ( (2T-{_R}y_{N+1-i}(\xi))^{1-1/r}  -\frac{1}{r}\frac{h_N}{Z_1}) \\
    %   \end{aligned}
    % \end{equation}
    % Similarly,
    % \begin{equation}
    %   \begin{aligned}
    %     | (2T-{_R}y_{N+1-i})^{1-1/r} - \frac{1}{r}\frac{h_N}{Z_1} |
    %      & = |\eta^{1-1/r} - \frac{1}{r}\frac{x_N-(x_N^{1/r}-Z_1)^r}{Z_1} | \quad \eta \in [x_{N-2}, x_{N}] \\
    %      & = T^{1-1/r} |(\frac{N-s}{N})^{r-1} - (\frac{N-t}{N})^{r-1} |  \quad t\in[0, 1], s\in [0, 2]      \\
    %      & \le T^{1-1/r} |(\frac{N-2}N)^{r-1}-1| \le C T^{1-1/r} (r-1)\frac{2}{N}
    %   \end{aligned}
    % \end{equation}


    And
    \begin{equation}
      \begin{aligned}
        ({_L}h_{N-i}^3(\xi))'' & = 3 {_L}h_{N-i}^2(\xi) {{_L}h_{N-i}}''(\xi) + 6 {_L}h_{N-i}(\xi) ({{_L}h_{N-i}}'(\xi))^2                                                     \\
                              & \le C h_N^2 \frac{1-r}{r} x_i^{1/r-2} (\frac{1}{r}\frac{h_N}{Z_1}  - {_L}y_{N-1-i}^{1-2/r}(\xi)Z_{N-1-i}) + C h_N (r-1)^2 h^2 x_i^{2/r-2}
      \end{aligned}
    \end{equation}

    \begin{equation*}
      \begin{aligned}
        |\frac{h_N}{rZ_1}- {_L}y_{N-1-i}^{1-2/r}(\xi)Z_{N-1-i}| \le T^{1-1/r} + C x_N^{1-2/r} x_N^{1/r} = C T^{1-1/r}
      \end{aligned}
    \end{equation*}
    So
    \begin{equation}
      \begin{aligned}
        ({_L}h_{N-i}^3(\xi))''
         & \le C h_N^2 \frac{1-r}{r} x_i^{1/r-2} + C (r-1)^2 h_N x_i^{2/r-2} h^2 \\
         & \le C (r-1) h_N^2
      \end{aligned}
    \end{equation}

    \({_R}h_{N+1-i}^3(\xi)\) is similar.
  \end{proof}







  \begin{lemma}
    \begin{gather}
      u''({_L}y_{N-i}^\theta(\xi)) \le C x_{N-2}^{-\alpha/2-2} \le C\\
      (u''({_L}y_{N-i}^\theta(\xi)))' \le C \\
      (u''({_L}y_{N-i}^\theta(\xi)))'' \le C
    \end{gather}
  \end{lemma}
  \begin{proof}
    \begin{equation}
      \begin{aligned}
        (u''({_L}y_{N-i}^\theta(\xi)))' & = u'''({_L}y_{N-i}^\theta(\xi)) {{_L}y_{N-i}^\theta}'(\xi)                              \\
                                       & \le C (\theta {{_L}y_{N-1-i}}'(\xi)  + (1-\theta) {{_0}y_{N-i}}'(\xi))                  \\
                                       & \le C x_i^{1/r-1} (\theta {_L}y_{N-1-i}^{1-1/r}(\xi)  + (1-\theta) \frac{h_N}{rZ_1}  ) \\
                                       & \le C x_i^{1/r-1} x_N^{1-1/r}                                                         \\
      \end{aligned}
    \end{equation}
    And
    \begin{equation}
      \begin{aligned}
        (u''({_L}y_{N-i}^\theta(\xi)))'' & = u''''({_L}y_{N-i}^\theta(\xi)) ({{_L}y_{N-i}^\theta}'(\xi))^2 + u'''({_L}y_{N-i}^\theta(\xi)) {{_L}y_{N-i}^\theta}''(\xi)     \\
                                        & \le C x_i^{2/r-2} x_N^{2-2/r} + C \frac{r-1}{r} x_i^{1/r-2} (\theta x_N^{1-2/r} Z_{N-1-i} + (1-\theta) \frac{h_N}{r Z_1}  ) \\
                                        & \le C x_i^{2/r-2} + C (r-1)  x_i^{1/r-2} T^{1-1/r}                                                                          \\
      \end{aligned}
    \end{equation}
  \end{proof}

  \begin{lemma}

    \begin{gather}
      |{_L}y_{N-i}^\theta(\xi) - \xi|^{1-\alpha} \le C |y_N^\theta - x_i|^{1-\alpha} \\
      (|{_L}y_{N-i}^\theta(\xi) - \xi|^{1-\alpha})' \le C |y_N^\theta - x_i|^{1-\alpha} \\
      (|{_L}y_{N-i}^\theta(\xi) - \xi|^{1-\alpha})'' \le C(r-1) |y_N^\theta - x_i|^{-\alpha} + |y_N^\theta - x_i|^{1-\alpha}
    \end{gather}
  \end{lemma}
  \begin{proof}
    \begin{equation}
      \begin{aligned}
        ({_L}y_{N-i}^\theta(\xi) - \xi)'
         & = (\theta ({_L}y_{N-1-i}(\xi) - \xi) + (1-\theta) ({_0}y_{N-i}(\xi) - \xi))'                                                 \\
         & = \theta ({{_L}y_{N-1-i}}'(\xi) - 1 ) + (1-\theta) ({{_0}y_{N-i}}'(\xi) - 1)                                                 \\
         & = \theta \xi^{1/r-1} ({_L}y_{N-1-i}^{1-1/r}(\xi) - \xi^{1-1/r} ) + (1-\theta) \xi^{1/r-1} (\frac{h_N}{rZ_1} - \xi^{1-1/r} ) \\
      \end{aligned}
    \end{equation}
    \begin{equation}
      \begin{aligned}
        ({_L}y_{N-i}^\theta(\xi) - \xi)''
         & = \theta ({{_L}y_{N-1-i}}''(\xi)) + (1-\theta) ( {{_0}y_{N-i}}''(\xi) )                                           \\
         & = \frac{1-r}{r} \xi^{1/r-2}(\theta {_L}y_{N-1-i}^{1-2/r}(\xi) Z_{N-1-i} + (1-\theta) \frac{h_N}{r Z_1}  )  \le 0
      \end{aligned}
    \end{equation}
    And
    \begin{equation}
      |({_L}y_{N-i}^\theta(\xi) - \xi)''| \le C (r-1) \xi^{1/r-2} T^{1-1/r}
    \end{equation}


    % \begin{equation}
    %   \begin{aligned}
    %     ({_L}y_{N-i}^\theta(\xi) - \xi)''
    %      & = \frac{1-r}{r} \xi^{1/r-2} (1-\theta) \frac{h_N}{r Z_1}   \\
    %      &\le C(r-1) |x_N - y_N^\theta| h^{-1}
    %   \end{aligned}
    % \end{equation}










    We have known
    \begin{equation}
      C |x_{N-1} - x_i| \le |{_L}y_{N-1-i}(\xi) - \xi| \le C |x_{N-1} - x_i|
    \end{equation}
    If \(\xi\le x_{N-1}\), then \(({_0}y_{N-i}(\xi) - \xi)'\ge 0\), so
    \begin{equation}
      C|x_{N}-x_{i}|  \le |x_{N-1} - x_{i-1}| \le |{_L}y_{N-i}^\theta(\xi) - \xi| \le |x_{N+1} - x_{i+1}| \le C |x_{N} - x_{i}|
    \end{equation}
    If \(i=N-1\) and \(\xi \in [x_{N-1}, x_{N}]\), then \({_0}y_{N-i}(\xi) - \xi\) is concave, bigger than its two neighboring points, which are equal to \(h_N\), so
    \begin{equation}
      h_N = |x_{N}-x_{N-1}| \le |{_0}y_{N-i}(\xi) - \xi| \le |x_{N+1} - x_{N-1}| = 2h_N
    \end{equation}
    So we have
    \begin{equation}
      |{_L}y_{N-i}^\theta(\xi) - \xi|^{1-\alpha} \le C |y_N^\theta - x_i|^{1-\alpha}
    \end{equation}

    While
    \begin{equation}
      {_L}y_{N-1-i}^{1-1/r}(\xi) - \xi^{1-1/r} \le ({_L}y_{N-1-i}(\xi) - \xi)\xi^{-1/r}
    \end{equation}
    and
    \begin{equation}
      \begin{aligned}
        |\frac{h_N}{rZ_1} - \xi^{1-1/r}| & \le \max\{ |\frac{h_N}{rZ_1} - x_{i-1}^{1-1/r}|, |\frac{h_N}{rZ_1} - x_{i+1}^{1-1/r}|\}             \\
                                         & \le \max \begin{cases}
                                                      T^{1-1/r}-x_{i-1}^{1-1/r} \le |x_N-x_{i-1}|T^{-1/r} \le C |x_N - x_{i}| \\
                                                      |x_{i+1}^{1-1/r} - x_{N-1}^{1-1/r}| \le |x_{i+1} - x_{N-1}|x_{N-1}^{-1/r} \le C |x_N - x_i|
                                                    \end{cases}
      \end{aligned}
    \end{equation}
    So we have
    \begin{equation}
      ({_L}y_{N-i}^\theta(\xi) - \xi)' \le C |y_N^\theta - x_i|
    \end{equation}
    \begin{equation}
      \begin{aligned}
        (|{_L}y_{N-i}^\theta(\xi) - \xi|^{1-\alpha})' & = |{_L}y_{N-i}^\theta(\xi) - \xi|^{-\alpha} ({_L}y_{N-i}^\theta(\xi) - \xi)' \\
                                                     & \le  |y_N^\theta - x_i|^{1-\alpha}
      \end{aligned}
    \end{equation}

    Finally,
    \begin{equation}
      \begin{aligned}
        (|{_L}y_{N-i}^\theta(\xi) - \xi|^{1-\alpha})'' & = (1-\alpha)|{_L}y_{N-i}^\theta(\xi) - \xi|^{-\alpha} ({_L}y_{N-i}^\theta(\xi) - \xi)''                   \\
                                                      & \quad + \alpha(\alpha-1) |{_L}y_{N-i}^\theta(\xi) - \xi|^{-1-\alpha} (({_L}y_{N-i}^\theta(\xi) - \xi)')^2 \\
                                                      & \le C(r-1) |y_N^\theta - x_i|^{-\alpha} + C |y_N^\theta - x_i|^{1-\alpha}                               \\
      \end{aligned}
    \end{equation}
    % \textcolor{red}{And for \(i=N-1\)},
    % \begin{equation}
    %   \begin{aligned}
    %     (|{_L}y_{N-i}^\theta(\xi) - \xi|^{1-\alpha})'' &\le C(r-1)|
    %   \end{aligned}
    % \end{equation}

  \end{proof}

  By the three lemmas above, for \(N/2 \le i \le N-1 \),  we have
  \begin{lemma} \label{lmm:d2Pi2N}
    \begin{equation}
      \begin{aligned}
        D_h^2 {_L}P_{N-i}^\theta(x_i) & = {{_L}P_{N-i}^\theta}''(\xi) \quad \xi\in (x_{i-1}, x_{i+1})                                                          \\
                                     & \le C h^3 |y_N^\theta - x_i|^{1-\alpha} + C(r-1)(h^3 |y_N^\theta - x_i|^{-\alpha} + h^2 |y_N^\theta - x_i|^{1-\alpha})
      \end{aligned}
    \end{equation}
    while \( \theta h_N = y_N^\theta - x_{N-1}\le y_N^\theta - x_i \), we have
    \begin{equation}
      \begin{aligned}
        \theta D_h^2 {_L}P_{N-i}^\theta(x_i)
          & \le C h^3 |y_N^\theta - x_i|^{1-\alpha} + C(r-1)(h^2 |y_N^\theta - x_i|^{1-\alpha})
      \end{aligned}
    \end{equation}
  \end{lemma}
  And
  \begin{lemma}
    \begin{equation}
      \begin{aligned}
        \frac{2}{h_{i} + h_{i+1}} & \left( \frac{{{_L}Q_{N-i}^\theta}(x_{i+1}) u'''(\eta_{N+1}^\theta) - {{_L}Q_{N-i}^\theta}(x_{i}) u'''(\eta_{N}^\theta)}{h_{i+1}}\right) \\
                                  & \le C h^3  |y_N^\theta - x_i|^{1-\alpha}
      \end{aligned}
    \end{equation}
  \end{lemma}
  And immediately with \cref{lmm:Vij-int}, For \(N/2 \le i \le N-1\)

  \begin{equation}
    \begin{aligned}
      V_{iN} & \le C \int_{x_{N-1}}^{x_N} h^2 |y - x_i|^{1-\alpha} + C(r-1) h |y - x_i|^{1-\alpha} dy \\
             & \le C h^2 h_N |T-x_i|^{1-\alpha} + C(r-1) h^2 |x_{N}-x_i|^{1-\alpha}              \\
             & \le C h^2 + C(r-1) h^2 |T-x_{i-1}|^{1-\alpha}
    \end{aligned}
  \end{equation}

  % \textcolor{red}{But expecially, when \(i=N-1\),}
  % \begin{equation}
  %   \begin{aligned}
  %     V_{N-1,N}
  %      & = \int_{0}^{1} -\frac{\theta^{2-\alpha} (1-\theta)}{2}\frac{2}{h_{N-1}+h_{N}} \left( \frac{1}{h_{N-1}} h_{N-1}^{4-\alpha}u''(y_{N-1}^\theta) - (\frac{1}{h_{N-1}}+\frac{1}{h_{N}}) h_N^{4-\alpha} u''(y_N^\theta) + \frac{1}{h_N} h_{N+1}^{4-\alpha} u''(y_{N+1}^\theta)    \right)   d\theta \\
  %      & \quad +  \int_{0}^1 \frac{\theta^3 (1-\theta)}{3!} \frac{2}{h_{i} + h_{i+1}}\left( \frac{{{_L}Q_{N-i}^\theta}(x_{i+1}) u'''(\eta_{N+1,1}^\theta) - {{_L}Q_{N-i}^\theta}(x_{i}) u'''(\eta_{N,1}^\theta)}{h_{i+1}}\right)  d\theta                                                                \\
  %      & \quad -  \int_{0}^1 \frac{\theta^3 (1-\theta)}{3!} \frac{2}{h_{i} + h_{i+1}}\left( \frac{{{_L}Q_{N-i}^\theta}(x_{i}) u'''(\eta_{N,1}^\theta) - {{_L}Q_{N-i}^\theta}(x_{i-1}) u'''(\eta_{N-1,1}^\theta)}{h_{i}}\right)  d\theta                                                                  \\
  %      & \quad -  \int_{0}^1 \frac{\theta (1-\theta)^3}{3!} \frac{2}{h_{i} + h_{i+1}}\left( \frac{{{_L}Q_{N-i}^\theta}(x_{i+1}) u'''(\eta_{N+1,2}^\theta) - {{_L}Q_{N-i}^\theta}(x_{i}) u'''(\eta_{N,2}^\theta)}{h_{i+1}}\right)  d\theta                                                                \\
  %      & \quad +  \int_{0}^1 \frac{\theta (1-\theta)^3}{3!} \frac{2}{h_{i} + h_{i+1}}\left( \frac{{{_L}Q_{N-i}^\theta}(x_{i}) u'''(\eta_{N,2}^\theta) - {{_L}Q_{N-i}^\theta}(x_{i-1}) u'''(\eta_{N-1,2}^\theta)}{h_{i}}\right)  d\theta                                                                  \\
  %   \end{aligned}
  % \end{equation}

  % \textcolor{red}{
  %   while combine \cref{lmm:estimate-hN-j=N}
  %   \begin{equation}
  %     \begin{aligned}
  %        & \frac{2}{h_{N-1}+h_{N}} \left( \frac{1}{h_{N-1}} h_{N-1}^{4-\alpha}u''(y_{N-1}^\theta) - (\frac{1}{h_{N-1}}+\frac{1}{h_{N}}) h_N^{4-\alpha} u''(y_N^\theta) + \frac{1}{h_N} h_{N+1}^{4-\alpha} u''(y_{N+1}^\theta)    \right) \\
  %        & = D_h^2 (h_{N-1 \to N}^{4-\alpha}(x_i)u''(y_{N-1\to N}^\theta(x_i)))                                                                                                                                                          \\
  %        & \le Ch_N^{4-\alpha} + C (r-1) h_N^{3-\alpha} \le C h^{4-\alpha} + C(r-1) h^2 |T-x_{N-1-1}|^{1-\alpha}
  %     \end{aligned}
  %   \end{equation}
  % }

  Similarly with \(j=N+1\).
\end{proof}




\newpage
\(I_4, I_5\) is easy. Similar with \cref{lmm:Ri-I2-I4-ilN/2} and \cref{lmm:sumSi-ilessN/2-Np1-2N}, we have

\textcolor{red}{
  \begin{theorem} \label{thm:Ri-I6-ileN}
    There is a constant \(C=C(T, \alpha, r, \|u\|_{\beta+\alpha}^{(-\alpha/2)})\) such that
    For \(N/2\le i\le N\),
    \begin{equation}
      \begin{aligned}
        I_4 & = \frac{2}{h_i + h_{i+1}}
        \left( \frac{1}{h_{i}} (T_{i-1, 2N-\lceil\frac{N}{2}\rceil+1} +  T_{i-1, 2N-\lceil\frac{N}{2}\rceil})
        - (\frac{1}{h_{i}}+\frac{1}{h_{i+1}}) T_{i,2N-\lceil\frac{N}{2}\rceil+1} \right) \\
            & \le C h^2
      \end{aligned}
    \end{equation}
  \end{theorem}
}
\begin{proof}
  Similar with \cref{lmm:Ri-I2-I4-ilN/2}.
  In fact, let \textcolor{red}{\(m=2N-\lceil\frac{N}{2}\rceil+1\)}
  \begin{equation}
    \begin{aligned}
       & \frac{1}{h_{i}} (T_{i-1, l} +  T_{i-1, l-1})
      - (\frac{1}{h_{i}}+\frac{1}{h_{i+1}}) T_{i,l}                                                                                              \\
       & = \frac{1}{h_{i}} (T_{i-1, l} -  T_{i, l}) + \frac{1}{h_{i}} (T_{i-1, l-1} -  T_{i, l}) + (\frac{1}{h_{i}} - \frac{1}{h_{i+1}}) T_{i,l}
    \end{aligned}
  \end{equation}
  While, by \cref{lmm:Dyj}
  \begin{equation}
    \begin{aligned}
      \frac{1}{h_{i}} (T_{i-1, l} -  T_{i, l})
       & = \int_{x_{l-1}}^{x_l} (u(y)-\Pi_hu(y)) \frac{|x_{i-1}-y|^{1-\alpha} - |x_i-y|^{1-\alpha}}{h_{i}\Gamma(2-\alpha)} dy \\
       & \le C \int_{x_{l-1}}^{x_l} h_l^2 u''(\eta)\; \frac{|\xi-y|^{-\alpha}}{\Gamma(1-\alpha)} dy   , \quad \xi \in (x_{i-1}, x_i)    \\
       & \le C h_l^3 (2T-x_{l-1})^{\alpha/2-2} T^{-\alpha}        \\
       & \le C h_l^3
    \end{aligned}
  \end{equation}
  Thus,
  \begin{equation}
    \frac{2}{h_i + h_{i+1}} \frac{1}{h_{i}} (T_{i-1, l} -  T_{i, l}) \le C h_l^2
  \end{equation}
  For
  \begin{equation}
    \begin{aligned}
      \frac{1}{h_{i}} (T_{i-1,l-1} -  T_{i, l})
       & = \int_{0}^{1} -\frac{\theta(1-\theta)}{2} \frac{h_{l-1}^3|y_{l-1}^\theta-x_{i-1}|^{1-\alpha} u''(\eta_{l-1}^\theta) - h_{l}^3|y_{l}^\theta-x_{i}|^{1-\alpha} u''(\eta_{l}^\theta)}{h_{i}}  d\theta \\
    \end{aligned}
  \end{equation}
  And Similar with \cref{lmm:dQj-itle}, we can get
  \begin{equation}
    \begin{aligned}
      \frac{h_{l-1}^3|y_{l-1}^\theta-x_{i-1}|^{1-\alpha} u''(\eta_{l-1}^\theta) - h_{l}^3|y_{l}^\theta-x_{i}|^{1-\alpha} u''(\eta_{l}^\theta)}{(h_{i}+h_{i+1})h_{i}} \le C h_{l}^2 |y_{l}^\theta-x_i|^{1-\alpha}
    \end{aligned}
  \end{equation}
  So
  \begin{equation}
    \frac{2}{h_i + h_{i+1}}  \frac{1}{h_{i}} (T_{i-1, l-1} -  T_{i, l}) \le C h^2
  \end{equation}
  For the third term, by \cref{lmm:hilexi}, \cref{lmm:hi1-hi} and \cref{lmm:Dyj}
  \begin{equation}
    \begin{aligned}
      \frac{2}{h_i + h_{i+1}} \frac{h_{i+1}-h_{i}}{h_i h_{i+1}} T_{i,l}
       & \le h_i^{-3} h^2 x_i^{1-2/r}  h_l C h_l^2 x_{l-1}^{\alpha/2-2} |x_{l}-x_i|^{1-\alpha} \\
       & \le C h^2
    \end{aligned}
  \end{equation}
  Summarizes, we have
  \begin{equation}
    I_4 \le C h^2
  \end{equation}
\end{proof}

And
\textcolor{blue}{
  \begin{lemma} \label{lmm:Ri-I7-ileN}
    There is a constant \(C=C(T, \alpha, r, \|u\|_{\beta+\alpha}^{(-\alpha/2)})\) such that
    For \(N/2 \le i \le N\),
    \begin{equation}
      \begin{aligned}
        I_5 & = \sum_{j=2N-\lceil\frac{N}{2}\rceil+2}^{2N} S_{ij} \\
            & \le \begin{cases}
                    C h^2,             & \alpha/2-2/r+1 > 0 \\
                    C h^2 \ln(N) ,     & \alpha/2-2/r+1 = 0 \\
                    C h^{r\alpha/2+r}, & \alpha/2-2/r+1 < 0
                  \end{cases}
      \end{aligned}
    \end{equation}
  \end{lemma}
}
\begin{proof}
  For \(i\le N, j\ge 2N-\lceil\frac{N}{2}\rceil+2\), we have
  \begin{equation*}
    \begin{aligned}
      S_{ij} & = \int_{x_{j-1}}^{x_{j}}(u(y) - \Pi_hu(y)) D_h^2 K_y(x_i) dy \\
             & \le \int_{x_{j-1}}^{x_{j}} C h^2 (2T-y)^{\alpha/2-2/r} |y-x_{i+1}|^{-1-\alpha} dy                                 \\
             & \le C h^2  T^{-1-\alpha} \int_{x_{j-1}}^{x_{j}} (2T-y)^{\alpha/2-2/r} dy
    \end{aligned}
  \end{equation*}
  \begin{equation}
    \begin{aligned}
      \sum_{j=2N-\lceil\frac{N}{2}\rceil+2}^{2N-1} S_{ij}
       & \le C T^{-1-\alpha} h^2 \int_{(2-2^{-r})T}^{x_{2N-1}} (2T-y)^{\alpha/2-2/r}  dy                 \\
       & \le CT^{-1-\alpha} h^2 \begin{cases}
                                  \frac{1}{\alpha/2-2/r+1} T^{\alpha/2-2/r+1},        & \alpha/2-2/r+1 > 0 \\
                                  \ln(2^{-r}T)-\ln(h_{2N}),                           & \alpha/2-2/r+1 = 0 \\
                                  \frac{1}{|\alpha/2-2/r+1|} h_{2N}^{\alpha/2-2/r+1}, & \alpha/2-2/r+1 < 0
                                \end{cases} \\
       & = \begin{cases}
             \frac{C}{\alpha/2-2/r+1}T^{-\alpha/2-2/r} \; h^2,                & \alpha/2-2/r+1 > 0 \\
             CrT^{-1-\alpha} h^2 \ln(N),                                      & \alpha/2-2/r+1 = 0 \\
             \frac{C}{|\alpha/2-2/r+1|} T^{-\alpha/2-2/r} \; h^{r\alpha/2+r}, & \alpha/2-2/r+1 < 0
           \end{cases}
    \end{aligned}
  \end{equation}
\end{proof}

Now we can conclude a part of  the theorem \cref{thm:Ri-N/2le-i-leN}  at the beginning of this section.

By
\cref{lmm:Ri-I1}
\cref{lmm:Ri-I2-I4-ilN/2}
\cref{lmm:Ri-I3-i<N}
\cref{thm:Ri-I4-i<N}
\cref{thm:Ri-I5-i<N}
\cref{thm:Ri-I6-ileN}
\cref{lmm:Ri-I7-ileN}
, we have
\begin{theorem}
  \label{thm:estimate-Ri-i<N}
  there exists a constant \(C=C(T, \alpha, r, \|u\|_{\beta+\alpha}^{(-\alpha/2)})\) such that
  for \(N/2\le i\le N-1\),
  \begin{equation}
    \begin{aligned}
      R_i & = I_1 + I_2 + I_3^1 + I_3^2 + I_3^3 + I_4 + I_5      \\
          & \le C(r-1) h^2|T-x_{i-1}|^{1-\alpha}  + \begin{cases}
                                                      C h^2,             & \alpha/2-2/r+1 > 0 \\
                                                      C h^2 \ln(N) ,     & \alpha/2-2/r+1 = 0 \\
                                                      C h^{r\alpha/2+r}, & \alpha/2-2/r+1 < 0
                                                    \end{cases}
    \end{aligned}
  \end{equation}
\end{theorem}

\newpage

% \subsection{$R_i$ for $i=N$}


And what we left is the case \(i=N\). Fortunately, we can use the same department of \(R_i\) above, and it is symmetric. Most of the item has been esitmated by \cref{lmm:Ri-I1} and  \cref{thm:Ri-I6-ileN}, we just need to consider \(I_3, I_4\).

\textcolor{orange}{
  \begin{theorem} \label{thm:RN-I3}
    There exists a constant \(C=C(T, \alpha, r, \|u\|_{\beta+\alpha}^{(-\alpha/2)})\) such that
    \begin{equation}
      \begin{aligned}
        I_3 = \sum_{j=\lceil\frac{N}{2}\rceil+1}^{N-1} V_{Nj} \le C h^2 + C (r-1) h^2 |T-x_{N-1}|^{1-\alpha}
      \end{aligned}
    \end{equation}
  \end{theorem}
}
\begin{proof}
  \begin{definition}
    For \(N/2 \le j < N\) , Let's define
    \begin{equation}
      y_{j}(x) = (\frac{Z_1}{h_N}  (x-x_N) + Z_j)^r, \quad Z_j = T^{1/r} \frac{j}{N}
    \end{equation}
    We can see that is the inverse of the function \({_0}y_{N-i}(x)\) defined in \cref{thm:Ri-I4-i<N}.
    \begin{gather}
      y_j'(x) = y_j^{1-1/r}(x) \frac{rZ_1}{h_N}     \\
      y_j''(x) = y_j^{1-2/r}(x) \frac{r(r-1)Z_1}{h_N}
    \end{gather}
  \end{definition}
  With the scheme we used several times, we can get
  \begin{lemma}
    There exists a constant \(C=C(T, \alpha, r, \|u\|_{\beta+\alpha}^{(-\alpha/2)})\) such that
    For \(N/2 \le j < N\), \(\xi \in [x_{N-1}, x_{N+1}]\),
    \begin{gather}
      h_{j}(\xi)^3 \le C h^3 \\
      (h_{j}^3(\xi))' \le C(r-1) h^3 \\
      (h_{j}^3(\xi))'' \le C(r-1) h^3
    \end{gather}
    \begin{gather}
      u''(y_{j}^\theta(\xi)) \le C \\
      (u''(y_{j}^\theta(\xi)))' \le C \\
      (u''(y_{j}^\theta(\xi)))'' \le C
    \end{gather}
    \begin{gather}
      |\xi - y_{j}^\theta(\xi)|^{1-\alpha} \le C |x_N - y_j^\theta|^{1-\alpha}  \\
      (|\xi - y_{j}^\theta(\xi)|^{1-\alpha})' \le C |x_N - y_j^\theta|^{1-\alpha}  \\
      (|\xi - y_{j}^\theta(\xi)|^{1-\alpha})'' \le C |x_N - y_j^\theta|^{1-\alpha} + C(r-1)|x_N- y_j^\theta|^{-\alpha}
    \end{gather}

  \end{lemma}
  \begin{lemma} \label{lmm:estimate-VNj-j<N}
    There exists a constant \(C=C(T, \alpha, r, \|u\|_{\beta+\alpha}^{(-\alpha/2)})\) such that
    For \(N/2 \le j < N\),
    \begin{equation}
      V_{Nj} \le C h^2 \int_{x_{j-1}}^{x_{j}} |x_N-y|^{1-\alpha} + (r-1)|x_N-y|^{-\alpha} dy
    \end{equation}
  \end{lemma}
  Therefore,
  \begin{equation}
    \begin{aligned}
      I_3 & \le C h^2 \int_{x_{\lceil\frac{N}{2}\rceil}}^{x_{N-1}} |x_N-y|^{1-\alpha} + (r-1)|x_N-y|^{-\alpha} dy \\
          & \le C h^2 (|T-x_{N-1}|^{2-\alpha} + (r-1)|T-x_{N-1}|^{1-\alpha} )
    \end{aligned}
  \end{equation}
\end{proof}

For \(j=N\),
\begin{lemma} \label{lmm:VNN}
  \begin{equation}
    V_{N,N} = \frac{1}{h_N^2}\left(T_{N-1,N-1} - 2T_{N,N} +T_{N+1, N+1} \right) \le Ch^2 + C(r-1)h^2|T-x_{N-1}|^{1-\alpha}
  \end{equation}
\end{lemma}
\begin{proof}
  \begin{equation}
    \begin{aligned}
      V_{N,N}
       & = \int_{0}^{1} -\frac{\theta (1-\theta)^{2-\alpha}}{2}\frac{1}{h_{N}^2} \left( h_{N-1}^{4-\alpha}u''(y_{N-1}^\theta) - 2 h_N^{4-\alpha} u''(y_N^\theta) +  h_{N+1}^{4-\alpha} u''(y_{N+1}^\theta)    \right)   d\theta \\
       & \quad +  \int_{0}^1 \frac{\theta^3 (1-\theta)}{3!} \frac{1}{h_{N}}\left( \frac{{Q_{N\to N}^\theta}(x_{N+1}) u'''(\eta_{N+1,1}^\theta) - {Q_{N\to N}^\theta}(x_{i}) u'''(\eta_{N,1}^\theta)}{h_{N}}\right)  d\theta     \\
       & \quad -  \int_{0}^1 \frac{\theta^3 (1-\theta)}{3!} \frac{1}{h_{N}}\left( \frac{{Q_{N\to N}^\theta}(x_{N}) u'''(\eta_{N,1}^\theta) - {Q_{N\to N}^\theta}(x_{N-1}) u'''(\eta_{N-1,1}^\theta)}{h_{N}}\right)  d\theta     \\
       & \quad -  \int_{0}^1 \frac{\theta (1-\theta)^3}{3!} \frac{1}{h_{N}}\left( \frac{{Q_{N\to N}^\theta}(x_{N+1}) u'''(\eta_{N+1,2}^\theta) - {Q_{N\to N}^\theta}(x_{N}) u'''(\eta_{N,2}^\theta)}{h_{N}}\right)  d\theta     \\
       & \quad +  \int_{0}^1 \frac{\theta (1-\theta)^3}{3!} \frac{1}{h_{N}}\left( \frac{{Q_{N\to N}^\theta}(x_{N}) u'''(\eta_{N,2}^\theta) - {Q_{N\to N}^\theta}(x_{N-1}) u'''(\eta_{N-1,2}^\theta)}{h_{N}}\right)  d\theta     \\
    \end{aligned}
  \end{equation}
\end{proof}

So combine
\cref{lmm:Ri-I1},
\cref{thm:Ri-I6-ileN},
\cref{thm:RN-I3},
\cref{lmm:VNN}
We have
\begin{lemma}
  \begin{equation}
    R_N \le C(r-1) h^2 |T-x_{N-1}|^{1-\alpha}
    + \begin{cases}
      C h^2,             & \alpha/2-2/r+1 > 0 \\
      C h^2 \ln(N) ,     & \alpha/2-2/r+1 = 0 \\
      C h^{r\alpha/2+r}, & \alpha/2-2/r+1 < 0
    \end{cases}
  \end{equation}
\end{lemma}
and with
\cref{thm:estimate-Ri-i<N}
we prove the \cref{thm:Ri-N/2le-i-leN}



% \subsection{Truncation error}

% combine \cref{lmm:trunerror2}, \cref{thm:Ri-ilessN/2} and \cref{thm:Ri-N/2le-i-leN} we get
% For \(1\le i \le N\)
% \begin{equation}
%   \begin{aligned}
%     R_i \le C_2(r-1)h^2 |T-x_{i-1}|^{1-\alpha} + \begin{cases}
%                                                    C_1 h^2 x_i^{-\alpha/2-2/r},             & r\alpha/2+r-2 > 0 \\
%                                                    C_1 h^2 (x_i^{-1-\alpha}\ln(i)+ \ln(N)), & r\alpha/2+r-2 = 0 \\
%                                                    C_1 h^{r\alpha/2+r} x_i^{-1-\alpha/2},   & r\alpha/2+r-2 < 0
%                                                  \end{cases}
%   \end{aligned}
% \end{equation}

% But,
% \begin{gather}
%   h^2 x_i^{-\alpha/2-2/r}
%   \le T^{\alpha/2-2/r} \begin{cases}
%     h^2 x_i^{-\alpha},           & \text{if} \quad  r\alpha/2-2 \ge 0 \\
%     h^{r\alpha/2} x_i^{-\alpha}, & \text{if} \quad  r\alpha/2-2 \le 0
%   \end{cases}    \\
%   h^{r\alpha/2+r} x_i^{-1-\alpha} \le T^{-1} h^{r\alpha/2} x_i^{-\alpha}, \quad \text{if} \quad  r\alpha/2-2 \le 0    \\
% \end{gather}

% And when \(r\alpha/2 - 2 = -r < 0\),
% \begin{equation}
%   \begin{aligned}
%     h^2 x_i^{-1-\alpha} \ln(i) h^{-r\alpha/2} x_i^{\alpha}
%      & = h^r x_i^{-1} \ln(i)                   \\
%      & = T^{-1} \frac{\ln(i)}{i^r} \le C(T, r)
%   \end{aligned}
% \end{equation}
% and
% \begin{equation}
%   h^2 \ln(N) h^{-r\alpha/2} x_i^{\alpha} = h^r \ln(N) x_i^{\alpha} \le T^{\alpha} \frac{\ln(N)}{N^r} \le C(T, \alpha, r)
% \end{equation}

% So for \(1\le i \le N\),
% \begin{equation}
%   \begin{aligned}
%     R_i \le C_2(r-1)h^2 |T-x_{i-1}|^{1-\alpha} + C_1 h^{\min\{\frac{r\alpha}{2}, 2\}} x_i^{-\alpha}
%   \end{aligned}
% \end{equation}
% And for \(i\ge N\), it is symmetric for \(i\) and \(2N-i\).

% The proof of \cref{thm:truncation-error} completed.







\newpage
\section{Convergence analysis}
\label{sec:proof-convergence}


\subsection{Properties of some Matrices}

Review \cref{sec:numformat}, we have got \cref{def:aij}.


\begin{definition}
  We call one matrix an \(M\) matrix , which means its entries are positive on major diagonal and nonpositive on others, and strictly diagonally dominant in rows.
\end{definition}

Now we have
\begin{lemma} \label{lmm:AisM}
  Matrix \(A\) defined by \cref{eq:equation_matrix} where \cref{eq:aij} is an \(M\) matrix. 
  And there exists a constant \(C_A=C(T, \alpha, r)\) such that
  \begin{equation}
    \begin{aligned}
      S_i := & \sum_{j=1}^{2N-1} a_{ij}
      % =      & -\kappa_\alpha \sum_{j=1}^{2N-1}  \frac{2}{h_i + h_{i+1}} \left( \frac{1}{h_{i+1}} \tilde{a}_{i+1,j} - (\frac{1}{h_{i}}+\frac{1}{h_{i+1}})\tilde{a}_{i,j} + \frac{1}{h_{i}} \tilde{a}_{i-1,j}\right) \\
      \ge  C_A (x_i^{-\alpha} + (2T-x_i)^{-\alpha})
    \end{aligned}
  \end{equation}
\end{lemma}
\begin{proof}
  From \cref{eq:tildeaij}, we have
  \begin{equation}
    \begin{aligned}
      \sum_{j=1}^{2N-1} \tilde{a}_{ij}
        & =  \frac{1}{\Gamma(4-\alpha)} \left( \frac{|x_i-x_0|^{3-\alpha} - |x_i-x_1|^{3-\alpha}}{h_1} + \frac{|x_{2N}-x_i|^{3-\alpha} - |x_{2N-1}-x_i|^{3-\alpha}}{h_{2N}} \right)
    \end{aligned}
  \end{equation}
  Let
  \begin{equation}
    g(x) = g_{0}(x) + g_{2N}(x)
  \end{equation}
  where
  \begin{gather*}
    g_{0}(x) := \frac{-\kappa_\alpha}{\Gamma(4-\alpha)} \frac{|x-x_0|^{3-\alpha} - |x-x_1|^{3-\alpha}}{h_1}    \\
    g_{2N}(x) := \frac{-\kappa_\alpha}{\Gamma(4-\alpha)} \frac{|x_{2N}-x|^{3-\alpha} - |x_{2N-1}-x|^{3-\alpha}}{h_{2N}}
  \end{gather*}
  Thus
  \begin{equation*}
    -\kappa_\alpha \sum_{j=1}^{2N-1} \tilde{a}_{ij} = g(x_i)
  \end{equation*}

  Then
  % Then, for \(2\le i\le 2N-2\),
  \begin{equation}
    \begin{aligned}
      S_i := & \sum_{j=1}^{2N-1}a_{ij}          \\
      =      & \frac{2}{h_i + h_{i+1}} \left( \frac{1}{h_{i+1}} g(x_{i+1}) - (\frac{1}{h_{i}}+\frac{1}{h_{i+1}})g(x_{i}) + \frac{1}{h_{i}} g(x_{i-1}) \right) \\
      =      & D_h^2 g_0(x_i) + D_h^2 g_{2N}(x_i)
    \end{aligned}
  \end{equation}
  
  When \(i=1\)
  \begin{equation}
    \begin{aligned}
      D_h^2 g_0(x_1) &= \frac{2}{h_{1} + h_{2}} \left( \frac{1}{h_{2}} g_0(x_{2}) - (\frac{1}{h_{1}}+\frac{1}{h_{2}})g_0(x_{1}) + \frac{1}{h_{1}} g_0(x_{0}) \right)                                                       \\
      =                       & \frac{2\kappa_\alpha}{\Gamma(4-\alpha)} \frac{h_1^{3-\alpha}+h_2^{3-\alpha} + 2h_1^{2-\alpha}h_2 - (h_1+h_2)^{3-\alpha} }{(h_{1} + h_{2})h_1 h_2}                          \\
      =                       & \frac{2\kappa_\alpha}{\Gamma(4-\alpha)} \frac{h_1^{3-\alpha}+h_2^{3-\alpha} + 2h_1^{2-\alpha}h_2 - (h_1+h_2)^{3-\alpha} }{(h_{1} + h_{2})h_1^{1-\alpha} h_2} h_1^{-\alpha} \\
      =                       & \frac{2\kappa_\alpha}{\Gamma(4-\alpha)} \frac{1+(2^r-1)^{3-\alpha} + 2(2^r-1) - (2^r)^{3-\alpha} }{2^r (2^r-1)} h_1^{-\alpha}
    \end{aligned}
  \end{equation}
  but
  \begin{equation}
    1+(2^r-1)^{3-\alpha} + 2(2^r-1) - (2^r)^{3-\alpha} > 0
  \end{equation}

  While for \(i \ge 2\)
  \begin{equation}
    \begin{aligned}
      D_h^2 g_0(x_i) &=  g_0''(\xi) , \quad \xi \in (x_{i-1}, x_{i+1}) \\
                 & =  -\kappa_\alpha \frac{|\xi-x_0|^{1-\alpha} - |\xi-x_1|^{1-\alpha}}{\Gamma(2-\alpha)h_1}                                               \\
                 & = \frac{\kappa_\alpha}{-\Gamma(1-\alpha)}  |\xi-\eta|^{-\alpha} , \quad \eta\in [x_0, x_1]                                              \\
                 & \ge \frac{\kappa_\alpha}{-\Gamma(1-\alpha)} x_{i+1}^{-\alpha}  \ge \frac{\kappa_\alpha}{-\Gamma(1-\alpha)} 2^{-r\alpha} x_{i}^{-\alpha}
    \end{aligned}
  \end{equation}
  
  So
  \begin{equation}
    \frac{2}{h_i + h_{i+1}} \left( \frac{1}{h_{i+1}} g_0(x_{i+1}) - (\frac{1}{h_{i}}+\frac{1}{h_{i+1}})g_0(x_{i}) + \frac{1}{h_{i}} g_0(x_{i-1}) \right) \ge C x_i^{-\alpha}
  \end{equation}
  symmetricly,
  \begin{equation}
    \frac{2}{h_i + h_{i+1}} \left( \frac{1}{h_{i+1}} g_{2N}(x_{i+1}) - (\frac{1}{h_{i}}+\frac{1}{h_{i+1}})g_{2N}(x_{i}) + \frac{1}{h_{i}} g_{2N}(x_{i-1}) \right) \ge C(\alpha, r) (2T-x_i)^{-\alpha}
  \end{equation}

\end{proof}



Let
\begin{equation} \label{def:hat}
  g(x) = \begin{cases}
    x, & 0<x\le T \\ 2T-x , & T<x<2T
  \end{cases}
\end{equation}
And define
\begin{equation}
  G = \text{diag}(g(x_1), ..., g(x_{2N-1}))
\end{equation}
Then
\begin{lemma} \label{lmm:AGhasSingularity}
  The matrix \(B:= AG\) , the major diagnal is positive, and nonpositive on others.
  And there is a constant \(C_{AG}, C=C(\alpha, r)\) such that
  \begin{equation}
    M_i := \sum_{j=1}^{2N-1} b_{ij}
    \ge -C_{AG} (x_i^{1-\alpha} + (2T-x_i)^{1-\alpha}) + C \begin{cases}
      |T - x_{i-1}|^{1-\alpha} , & i\le N \\
      |x_{i+1} - T|^{1-\alpha} , & i\ge N
    \end{cases}
  \end{equation}
\end{lemma}
\begin{proof}
  \begin{equation*}
    b_{ij} = a_{ij} g(x_j) = -\kappa_\alpha  \frac{2}{h_i + h_{i+1}} \left( \frac{1}{h_{i+1}} \tilde{a}_{i+1,j} - (\frac{1}{h_{i}}+\frac{1}{h_{i+1}})\tilde{a}_{i,j} + \frac{1}{h_{i}} \tilde{a}_{i-1,j}\right) g(x_j)
  \end{equation*}
  Since
  \begin{equation}
    g(x) \equiv \Pi_h g(x)
  \end{equation}
  by \cref{eq:Ih}, we have
  \begin{equation}
    \begin{aligned}
      \tilde{M_i} & := \sum_{j=1}^{2N-1} \tilde{b}_{ij}
      = \sum_{j=1}^{2N-1} \tilde{a}_{ij} g(x_j)                                                                                        \\
                  & = \int_{0}^{2T} \frac{|x_i-y|^{1-\alpha}}{\Gamma(2-\alpha)} \Pi_h g(y) dy
      = \int_{0}^{2T} \frac{ |x_i-y|^{1-\alpha}}{\Gamma(2-\alpha)} g(y) dy                                                              \\
                  & = \frac{-2}{\Gamma(4-\alpha)}|T-x_i|^{3-\alpha} + \frac{1}{\Gamma(4-\alpha)}(x_i^{3-\alpha} + (2T-x_i)^{3-\alpha}) \\
                  & := w(x_i) = p(x_i) + q(x_i)
    \end{aligned}
  \end{equation}

  Thus,
  \begin{equation}
    \begin{aligned}
      M_i & := \sum_{j=1}^{2N-1} b_{ij} =\sum_{j=1}^{2N-1} a_{ij} g(x_j)      \\
        &= -\kappa_\alpha\frac{2}{h_i + h_{i+1}}\left(\frac{1}{h_{i+1}} \tilde{M}_{i+1} - (\frac{1}{h_{i}}+\frac{1}{h_{i+1}})\tilde{M}_{i} + \frac{1}{h_{i}} \tilde{M}_{i-1}\right)  \\
      %     & = -\kappa_\alpha\frac{2}{h_i + h_{i+1}}
      % \left( \frac{1}{h_{i+1}} w(x_{i+1})
      % - (\frac{1}{h_{i}}+\frac{1}{h_{i+1}}) w(x_{i})
      % +  \frac{1}{h_{i}} w(x_{i-1}) \right)
      & =  D_h^2 (-\kappa_\alpha p) (x_i) - \kappa_\alpha D_h^2 q(x_i)
    \end{aligned}
  \end{equation}
  for \(1\le i\le N-1\),by \cref{lmm:Dh2simd2}
  \begin{equation}
    \begin{aligned}
      D_h^2 (-\kappa_\alpha p)(x_i) & := -\kappa_\alpha\frac{2}{h_i + h_{i+1}}
      \left( \frac{1}{h_{i+1}} p(x_{i+1})
      - (\frac{1}{h_{i}}+\frac{1}{h_{i+1}}) p(x_{i})
      +  \frac{1}{h_{i}} p(x_{i-1}) \right)                                                                     \\
            & = \frac{2 \kappa_\alpha}{\Gamma(2-\alpha)} |T - \xi|^{1-\alpha}  \quad \xi \in (x_{i-1}, x_{i+1}) \\
            & \ge \frac{2 \kappa_\alpha}{\Gamma(2-\alpha)} |T - x_{i-1}|^{1-\alpha}
    \end{aligned}
  \end{equation}
  % and
  % \begin{equation}
  %   \begin{aligned}
  %     D_h^2 (-\kappa_\alpha p)(x_{N-1}) & := \frac{-2\kappa_\alpha}{h_{N-1} + h_{N}}
  %     \left( \frac{1}{h_{N}} p(x_{N})
  %     - (\frac{1}{h_{N-1}}+\frac{1}{h_{N}}) p(x_{N-1})
  %     +  \frac{1}{h_{N-1}} p(x_{N-2}) \right)                                                                                                                                   \\
  %             & = \frac{2 \kappa_\alpha}{\Gamma(4-\alpha)} \frac{2}{h_{N-1} + h_{N}}
  %     \left( - (\frac{1}{h_{N-1}}+\frac{1}{h_{N}}) h_N^{3-\alpha}
  %     +  \frac{1}{h_{N-1}} (h_{N-1}+h_{N})^{3-\alpha} \right)                                                                                                                   \\
  %             & = \frac{4 \kappa_\alpha}{\Gamma(4-\alpha) h_{N-1}}
  %     \left( - h_{N}^{2-\alpha}
  %     +  (h_{N-1}+h_{N})^{2-\alpha} \right)                                                                                                                                     \\
  %             & = \frac{4 \kappa_\alpha}{(3-\alpha)\Gamma(2-\alpha)} \xi^{1-\alpha}    \quad \xi \in [h_N, h_{N-1}+h_{N}]                                                       \\
  %             & \ge \frac{4 \kappa_\alpha}{(3-\alpha)\Gamma(2-\alpha)} (h_{N-1}+h_{N})^{1-\alpha} = \frac{4 \kappa_\alpha}{(3-\alpha)\Gamma(2-\alpha)} (T - x_{N-2})^{1-\alpha}
  %   \end{aligned}
  % \end{equation}
  \begin{equation}
    \begin{aligned}
      D_h^2(-\kappa_\alpha p)(x_N) & := -\kappa_\alpha \frac{2}{h_N + h_{N+1}}
      \left( \frac{1}{h_{N+1}} p(x_{N+1})
      - (\frac{1}{h_{N}}+\frac{1}{h_{N+1}}) p(x_{N})
      +  \frac{1}{h_{N}} p(x_{N-1}) \right)                                   \\
          & = \frac{4 \kappa_\alpha}{\Gamma(4-\alpha) h_N^2} h_N^{3-\alpha}   \\
          & = \frac{4 \kappa_\alpha}{\Gamma(4-\alpha)} (T-x_{N-1})^{1-\alpha}
    \end{aligned}
  \end{equation}

  Symmetricly for \(i\ge N\), we get
  \begin{equation} \label{eq:|T-xi-1|1-a}
    D_h^2(-\kappa_\alpha p)(x_i)\ge \frac{2\kappa_\alpha}{\Gamma(2-\alpha)} \begin{cases}
      |T - x_{i-1}|^{1-\alpha} , & i\le N \\
      |x_{i+1} - T|^{1-\alpha} , & i\ge N
    \end{cases}
  \end{equation}

  Similarly, we can get
  \begin{equation}
    \begin{aligned}
      D_h^2 q(x_i) & := \frac{2}{h_i + h_{i+1}}
      \left( \frac{1}{h_{i+1}} q(x_{i+1})
      - (\frac{1}{h_{i}}+\frac{1}{h_{i+1}}) q(x_{i})
      +  \frac{1}{h_{i}} q(x_{i-1}) \right)                                                                               \\
          & \le \frac{2^{r(\alpha-1)+1}}{\Gamma(2-\alpha)}  (x_{i}^{1-\alpha} + (2T-x_{i})^{1-\alpha}), \quad i=1,\cdots, 2N-1
    \end{aligned}
  \end{equation}
  So, we get the result.

\end{proof}



Notice that
\begin{equation}
  x_i^{-\alpha} \ge (2T)^{-1} x_i^{1-\alpha}
\end{equation}

We can get
\begin{theorem}\label{thm:ALGisM}
  There exists a real \(\lambda=\lambda(T, \alpha, r) > 0\) and \(C=C(T,\alpha,r) > 0\) such that \(B := A(\lambda I+G)\) is an \(M\) matrix. And
  \begin{equation}
    M_i := \sum_{j=1}^{2N-1} b_{ij} \ge C(x_i^{-\alpha} + (2T-x_i)^{-\alpha}) + C\begin{cases}
      |T - x_{i-1}|^{1-\alpha} , & i\le N \\
      |x_{i+1} - T|^{1-\alpha} , & i\ge N
    \end{cases}
  \end{equation}
\end{theorem}
\begin{proof}
  By \cref{lmm:AisM} with \(C_{A}\)  and \cref{lmm:AGhasSingularity} with \(C_{AG}\) , it's sufficient to take \(\lambda = (C + 2T C_{AG})/C_{A} \), then
  \begin{equation}
    \begin{aligned}
      M_i & \ge C \left( (x_i^{-\alpha} + (1-x_i)^{-\alpha})
      + \begin{cases}
          |T - x_{i-1}|^{1-\alpha} , & i\le N \\
          |x_{i+1} - T|^{1-\alpha} , & i\ge N
        \end{cases}   \right)
    \end{aligned}
  \end{equation}
\end{proof}


\subsection{Proof of \cref{thm:convergence}}

% Now, we can prove the convergency \cref{thm:convergence}.

For equation 
\begin{equation}
  A U = F \Leftrightarrow A(\lambda I+G)  (\lambda I+G)^{-1} U = F \quad \text{i.e.} \quad  B (\lambda I+G)^{-1} U = F
\end{equation}
which means
\begin{equation}
  \sum_{j=1}^{2N-1} b_{ij} \frac{\epsilon_j}{\lambda + g(x_j)} = -\tau_i
\end{equation}
where \(\epsilon_i = u(x_i) - u_i\).

And if 
\begin{equation}
  |\frac{\epsilon_{i_0}}{\lambda + g(x_{i_0})}| = \max_{1\le i\le 2N-1}|\frac{\epsilon_i}{\lambda + g(x_i)}| 
\end{equation}
Then, since \(B=A(\lambda I+G)\) is an \(M\) matrix, it is Strictly diagonally dominant. Thus,
\begin{equation}
  \begin{aligned}
    |\tau_{i_0}| &= |\sum_{j=1}^{2N-1} b_{i_0, j} \frac{\epsilon_j}{\lambda + g(x_j)}|  \\
    &\ge b_{i_0, i_0} |\frac{\epsilon_{i_0}}{\lambda + g(x_{i_0})}| - \sum_{j\ne i_0} |b_{i_0, j}| |\frac{\epsilon_j}{\lambda + g(x_j)}| \\
    &\ge b_{i_0, i_0} |\frac{\epsilon_{i_0}}{\lambda + g(x_{i_0})}| - \sum_{j\ne i_0} |b_{i_0, j}| |\frac{\epsilon_{i_0}}{\lambda + g(x_{i_0})}| \\
    &= \sum_{j=1}^{2N-1} b_{i_0, j} |\frac{\epsilon_{i_0}}{\lambda + g(x_{i_0})}|  \\
    &= M_{i_0} |\frac{\epsilon_{i_0}}{\lambda + g(x_{i_0})}|
  \end{aligned}
\end{equation}

By \cref{thm:truncation-error} and  \cref{thm:ALGisM}, 

We kown that there exists  constants \(C_1(T,\alpha,r, \|u\|_{\beta+\alpha}^{(-\alpha/2)}, \|f\|_{\beta}^{(\alpha/2)} )\), \\
and \(C_2(T,\alpha,r, \|u\|_{\beta+\alpha}^{(-\alpha/2)})\) such that
\begin{equation}
  |\frac{\epsilon_{i}}{\lambda + g(x_{i})}| \le |\frac{\epsilon_{i_0}}{\lambda + g(x_{i_0})}| \le C_1 h^{\min\{\frac{r\alpha}{2}, 2\}} + C_2 (r-1) h^2 
\end{equation}
as \(\lambda + g(x_{i}) \le \lambda + T\)

So, we can get
\begin{equation}
  |\epsilon_i| \le C(\lambda + T) h^{\min\{\frac{r\alpha}{2}, 2\}}
\end{equation}

The convergency has been proved.

Remarks:




\newpage
\section{Experimental results}
\label{sec:experiments}

\subsection{$f\equiv 1$}

\subsection{$f=x^\gamma, \gamma<0$}
% \textcolor{gray}{
%   \Cref{fig:testfig} shows some example results. Additional results are
%   available in the supplement in \cref{tab:foo}.
%   \begin{figure}[htbp]
%     \centering
%     \label{fig:a}\includegraphics{lexample_fig1}
%     \caption{Example figure using external image files.}
%     \label{fig:testfig}
%   \end{figure}
%   \Cref{tab:foo} shows additional
%   supporting evidence.
% }
% \textcolor{gray}{
%   \begin{table}[htbp]
%     \footnotesize
%     \caption{Example table.}\label{tab:foo}
%     \begin{center}
%       \begin{tabular}{|c|c|c|} \hline
%         Species & \bf Mean & \bf Std.~Dev. \\ \hline
%         1       & 3.4      & 1.2           \\
%         2       & 5.4      & 0.6           \\
%         3       & 7.4      & 2.4           \\
%         4       & 9.4      & 1.8           \\ \hline
%       \end{tabular}
%     \end{center}
%   \end{table}
% }

% \section{Discussion of \texorpdfstring{{\boldmath$Z=X \cup Y$}}{Z = X union Y}}


% \section{Remarks}
% \label{sec:remarks}

% some remarks.

% In \cref{thm:regularity}
% If \(f\in L^\infty(\Omega)\)  then \(u \in C_{\alpha/2}(\Omega)\), which is Proposition 1.1 in \cite{ROSOTON2014275}.

% When \(\|f\|_{\beta}^{(\gamma)} < \infty\), where \(\beta > 2-\alpha\) and \(\gamma \in [-\alpha, -\alpha/2]\), we observed convergent order \(\min\{r(\alpha+\gamma), 2\}\) in numerical experiments. 
% And we can prove that kind theorems with the techneque we used in this paper. 
















\appendix

\section{Approximate of difference quotients}

\begin{lemma} \label{lmm:Dh2simd2}
  If \(g(x)\in C^2(\Omega)\),
  there exists \(\xi\in (x_{i-1}, x_{i+1})\) such that
  \begin{equation} \label{eq:Dh2simd2}
    \begin{aligned}
      D_h^2 g(x_i) & = g''(\xi), \quad \xi \in (x_{i-1}, x_{i+1})
    \end{aligned}
  \end{equation}
  % \begin{equation} \label{eq:Dh2Cauchy2}
  %   \begin{aligned}
  %      & \frac{2}{h_i + h_{i+1}} \left( \frac{1}{h_{i+1}} g(x_{i+1}) - (\frac{1}{h_{i}}+\frac{1}{h_{i+1}})g(x_{i}) + \frac{1}{h_{i}} g(x_{i-1}) \right)                          \\
  %     % &= \frac{h_i}{h_i + h_{i+1}} g''(\xi_1) + \frac{h_{i+1}}{h_i + h_{i+1}} g''(\xi_2)  \\
  %      & \quad = \frac{2}{h_i + h_{i+1}} \left( \frac{1}{h_{i}}\int_{x_{i-1}}^{x_i} g''(y) (y-x_{i-1}) dy + \frac{1}{h_{i+1}} \int_{x_i}^{x_{i+1}} g''(y) (x_{i+1}-y) dy \right)
  %   \end{aligned}
  % \end{equation}
  And if \(g(x) \in C^4(\Omega)\), then
  \begin{equation} \label{eq:Dh2simd4}
    \begin{aligned}
      D_h^2 g(x_i) &= g''(x_{i}) + \frac{h_{i+1}-h_{i}}{3} g'''(x_{i}) \\
      &\quad  +  \frac{2}{h_i + h_{i+1}}\left( \frac{1}{h_{i}}\int_{x_{i-1}}^{x_i} g''''(y) \frac{(y-x_{i-1})^3}{3!} dy + \frac{1}{h_{i+1}} \int_{x_i}^{x_{i+1}} g''''(y) \frac{(x_{i+1}-y)^3}{3!} dy \right)  \\
    \end{aligned}
  \end{equation}
\end{lemma}
\begin{proof}
  \begin{gather*}
    g(x_{i-1}) = g(x_{i}) - (x_{i}-x_{i-1}) g'(x_{i}) + \frac{(x_{i}-x_{i-1})^2}{2} g''(\xi_1), \quad \xi_1 \in (x_{i-1}, x_{i})        \\
    g(x_{i+1}) = g(x_{i}) + (x_{i+1}-x_{i}) g'(x_{i}) + \frac{(x_{i+1}-x_{i})^2}{2} g''(\xi_2), \quad \xi_2 \in (x_{i}, x_{i+1})
  \end{gather*}
  Subsitute them in the left side of \eqref{eq:Dh2simd2}, we have
  \begin{equation*}
    \begin{aligned}
      D_h^2 g(x_i) & = \frac{2}{h_i + h_{i+1}} \left( \frac{1}{h_{i+1}} (g(x_{i+1})-g(x_i)  + \frac{1}{h_{i}} (g(x_{i-1})-g(x_i)) \right) \\
      &= \frac{h_i}{h_i + h_{i+1}} g''(\xi_1) + \frac{h_{i+1}}{h_i + h_{i+1}} g''(\xi_2)
    \end{aligned}
  \end{equation*}
  Now, using \textcolor{red}{intermediate value theorem}, there exists \(\xi \in [\xi_1, \xi_2]\) such that
  \begin{equation*}
    \frac{h_i}{h_i + h_{i+1}} g''(\xi_1) + \frac{h_{i+1}}{h_i + h_{i+1}} g''(\xi_2) = g''(\xi)
  \end{equation*}
  % For the second equation, similarly
  % \begin{gather*}
  %   g(x_{i-1}) = g(x_{i}) - (x_{i}-x_{i-1}) g'(x_{i}) + \int_{x_{i-1}}^{x_i} g''(y) (y-x_{i-1}) dy \\
  %   g(x_{i+1}) = g(x_{i}) + (x_{i+1}-x_{i}) g'(x_{i}) + \int_{x_i}^{x_{i+1}} g''(y) (x_{i+1}-y) dy
  % \end{gather*}
  And the last equation can be obtained by
  \begin{gather*}
    g(x_{i-1}) = g(x_{i}) - h_{i} g'(x_{i}) + \frac{h_{i}^2}{2} g''(x_{i}) - \frac{h_{i}^3}{3!} g'''(x_{i}) +  \int_{x_{i-1}}^{x_{i}} g''''(y) \frac{(y-x_{i-1})^3}{3!} dy        \\
    g(x_{i+1}) = g(x_{i}) + h_{i+1} g'(x_{i}) + \frac{h_{i+1}^2}{2} g''(x_{i}) + \frac{h_{i+1}^3}{3!} g'''(x_{i}) + \int_{x_{i}}^{x_{i+1}} g''''(y) \frac{(x_{i+1}-y)^3}{3!} dy
  \end{gather*}
  Expecially,
  \begin{equation} \label{eq:Dh2Cauchy4}
    \begin{aligned}
      \int_{x_{i-1}}^{x_{i}} g''''(y) \frac{(y-x_{i-1})^3}{3!} dy & = \frac{h_i^4}{4!}g''''(\eta_1) \\
      \int_{x_{i}}^{x_{i+1}} g''''(y) \frac{(x_{i+1}-y)^3}{3!} dy & = \frac{h_{i+1}^4}{4!}g''''(\eta_2)
    \end{aligned}
  \end{equation}
  where \(\eta_1 \in (x_{i-1}, x_{i}), \eta_2 \in (x_{i}, x_{i+1})\).
\end{proof}


\begin{lemma} \label{lmm:Dyj}
  Denote \(y_j^\theta = (1-\theta) x_{j-1} + \theta x_j, \theta\in (0,1)\),
  % then for \(2\le j \le 2N-1\)
  \begin{equation}
    \begin{aligned}
      u(y_j^\theta) - \Pi_hu(y_j^\theta) & = -\frac{\theta (1-\theta)}{2} h_j^2 u''(\xi), \quad \xi \in (x_{j-1}, x_j)
    \end{aligned}
  \end{equation}
  \begin{equation}
    \begin{aligned}
      u(y_j^\theta) - \Pi_hu(y_j^\theta) = & -\frac{\theta (1-\theta)}{2} h_j^2 u''(y_j^\theta)
      + \frac{\theta (1-\theta)}{3!} h_j^3 (\theta^2 u'''(\eta_1) - (1-\theta)^2 u'''(\eta_2))
    \end{aligned}
  \end{equation}
  where \(\eta_1 \in (x_{j-1}, y_j^\theta), \eta_2 \in (y_j^\theta, x_j)\).
\end{lemma}
\begin{proof}
  By Taylor expansion, we have
  \begin{gather*}
    u(x_{j-1}) = u(y_j^\theta) - \theta h_{j} u'(y_j^\theta) + \frac{\theta^2 h_{j}^2}{2!} u''(\xi_1), \quad \xi_1 \in (x_{j-1}, y_j^\theta) \\
    u(x_{j}) = u(y_j^\theta) + (1-\theta) h_{j} u'(y_j^\theta) + \frac{(1-\theta)^2 h_{j}^2}{2!} u''(\xi_2) , \quad \xi_2 \in (y_j^\theta, x_j)
  \end{gather*}
  Thus
  \begin{equation*}
    \begin{aligned}
      u(y_j^\theta) - \Pi_hu(y_j^\theta) 
      & = u(y_j^\theta) - (1-\theta) u(x_{j-1}) - \theta u(x_{j})      \\
      & = -\frac{\theta (1-\theta)}{2} h_j^2 ( \theta u''(\xi_1) + (1-\theta) u''(\xi_2) ) \\
      & = -\frac{\theta (1-\theta)}{2} h_j^2 u''(\xi), \quad \xi \in [\xi_1, \xi_2]
    \end{aligned}
  \end{equation*}
  The second equation is similar,
  \begin{gather*}
    u(x_{j-1}) = u(y_j^\theta) - \theta h_{j} u'(y_j^\theta) + \frac{ \theta^2h_{j}^2}{2!} u''(y_j^\theta) - \frac{\theta^3 h_{j}^3}{3!} u'''(\eta_1)  \\
    u(x_{j}) = u(y_j^\theta) + (1-\theta) h_{j} u'(y_j^\theta) + \frac{(1-\theta)^2 h_{j}^2}{2!} u''(y_j^\theta) + \frac{(1-\theta)^3 h_{j}^3}{3!} u'''(\eta_2)
  \end{gather*}
  where \(\eta_1 \in (x_{j-1}, y_j^\theta), \eta_2 \in (y_j^\theta, x_j)\).
  Thus
  \begin{equation*}
    \begin{aligned}
      u(y_j^\theta) - \Pi_hu(y_j^\theta) 
      & = u(y_j^\theta) - (1-\theta) u(x_{j-1}) -\theta u(x_{j})   \\
      & = -\frac{\theta (1-\theta)}{2} h_j^2 u''(y_j^\theta) + \frac{\theta (1-\theta)}{3!} h_j^3 ( \theta^2 u'''(\eta_1) - (1-\theta)^2 u'''(\eta_2))
    \end{aligned}
  \end{equation*}
\end{proof}

\begin{lemma} \label{lmm:Dyj1}
  For \(x\in [x_{j-1}, x_j]\)
  \begin{equation}
    \begin{aligned}
      |u(x) - \Pi_hu(x)| & = \left| \frac{x_{j}-x}{h_j} \int_{x_{j-1}}^x u'(y) dy - \frac{x-x_{j-1}}{h_j} \int_{x}^{x_{j}} u'(y) dy \right| \\
                      & \le \int_{x_{j-1}}^{x_{j}} |u'(y)| dy
    \end{aligned}
  \end{equation}
  If \(x\in [0, x_1]\), with \cref{cor:regularity-u}, we have
  \begin{equation}
    |u(x) - \Pi_hu(x)| \le \int_{0}^{x_1} |u'(y)| dy \le \int_{0}^{x_1} C y^{\alpha/2-1} dy  \le C\frac{2}{\alpha} x_1^{\alpha/2}
  \end{equation}
  Similarly, if \(x\in [x_{2N-1}, 1]\), we have
  \begin{equation}
    |u(x) - \Pi_hu(x)| \le C\frac{2}{\alpha} (2T-x_{2N-1})^{\alpha/2} = C\frac{2}{\alpha} x_1^{\alpha/2}
  \end{equation}
\end{lemma}


\begin{lemma} \label{ineq:a-b-theta}
  \begin{equation}
    b^{1-\theta}|a^{\theta}-b^{\theta}| \le |a-b| \; (\text{  also  }a^{1-\theta}|a^{\theta}-b^{\theta}| \le |a-b|), \quad a,b\ge 0,\; \theta \in [0,1]
  \end{equation}
\end{lemma}




\section{Proofs of some technical details}

Review that \(h=\frac{1}{N}\) and the defination of \(\simeq\) in \cref{sec:numformat}


\begin{lemma} \label{lmm:hilexi}
  \begin{equation}
    h_i \simeq \begin{cases}
      hx_i^{1-1/r}, & 1\le i \le N   \\
      h(2T-x_{i-1})^{1-1/r}, & N < i \le 2N
    \end{cases}
  \end{equation}
  Since \( i^r -(i-1)^r \simeq i^{r-1}, \text{  for  } i\ge 1 \).
  
  And
  \begin{equation}
    h_i \simeq h_{i+1}, \quad x_i \simeq x_{i+1},  \quad \text{  for  } 1\le i\le 2N-1
  \end{equation}
\end{lemma}



\textcolor{blue}{
  \begin{lemma} \label{lmm:hi1-hi}
    There is a constant \(C\) such that for \(i=1,2,\cdots,2N-1\)
    \begin{equation}
      |h_{i+1} - h_{i}| \le C h^2 \begin{cases}
        x_i^{1-2/r} ,     & 1\le i\le N-1 \\
        0,                & i=N           \\
        (2T-x_i)^{1-2/r}, & N<i\le 2N-1   \\
      \end{cases}
    \end{equation}
  \end{lemma}
  \begin{proof}
    By \cref{def:hj},
    \begin{equation}
      \begin{aligned}
        h_{i+1} - h_{i} =
        \begin{cases}
          T \left( \left(\frac{i+1}{N}\right)^r - 2\left(\frac{i}{N}\right)^r + \left(\frac{i-1}{N}\right)^r  \right) ,           & 1\le i\le N-1    \\
          0,    & i=N    \\
          -T \left( \left(\frac{2N-i-1}{N}\right)^r - 2\left(\frac{2N-i}{N}\right)^r + \left(\frac{2N-i+1}{N}\right)^r  \right) , & N+1\le i\le 2N-1 \\
        \end{cases}
      \end{aligned}
    \end{equation}
    Since
    \begin{equation}
      (i+1)^r - 2i^r + (i-1)^r \simeq r(r-1)i^{r-2}, \quad \text{for  } i\ge 1
    \end{equation}
    We get the result.
  \end{proof}
}




\begin{lemma} \label{lmm:trucerr2d2f}
  there is a constant \(C=C(T, \alpha, r, \|f\|_{\beta}^{\alpha/2})\) such that
    \begin{equation}
      \begin{aligned}
        & \frac{2}{h_i + h_{i+1}} \left| \frac{1}{h_i} \int_{x_{i-1}}^{x_{i}} f''(y) \frac{(y-x_{i-1})^3}{3!} dy + \frac{1}{h_{i+1}} \int_{x_{i}}^{x_{i+1}} f''(y) \frac{(y-x_{i+1})^3}{3!} dy \right| \\
         & \quad \le C h^2 \begin{cases}
                             x_i^{-\alpha/2-2/r} ,     & 1\le i\le N    \\
                             (2T-x_i)^{-\alpha/2-2/r}, & N\le i\le 2N-1
                           \end{cases}
      \end{aligned}
    \end{equation}
\end{lemma}
\begin{proof} \label{prf:trucerr2d2f}
  By \cref{thm:regularity-f}, we have for \(1 \le i \le N\)
  \begin{equation}
    \begin{aligned}
      \left|\int_{x_{i-1}}^{x_{i}} f''(y)\frac{(y-x_{i-1})^3}{3!} dy \right|  \le \frac{\|f\|_{\beta}^{(\alpha/2)}}{3!} \int_{x_{i-1}}^{x_{i}} y^{-\alpha/2-2} (y-x_{i-1})^3 dy
    \end{aligned}
  \end{equation}
  For \(i=1\),
  \begin{equation*}
    \int_{x_{i-1}}^{x_{i}} y^{-\alpha/2-2} (y-x_{i-1})^3 dy
    = \int_{0}^{x_{1}} y^{1-\alpha/2} dy
    = \frac{1}{2-\alpha/2} x_1^{2-\alpha/2} = \frac{1}{2-\alpha/2} x_1^{-\alpha/2-2} h_1^4
  \end{equation*}
  And for \(2\le i \le N\), since \(x_i\simeq x_{i-1}\le y\le x_i\), we have
  \begin{equation*}
    \begin{aligned}
      \int_{x_{i-1}}^{x_{i}} y^{-\alpha/2-2} (y-x_{i-1})^3 dy
      \simeq \int_{x_{i-1}}^{x_{i}} x_i^{-\alpha/2-2} (y-x_{i-1})^3 dy
      = \frac{1}{4!} x_i^{-\alpha/2-2} h_i^4
    \end{aligned}
  \end{equation*}
  So for  \(1\le i\le N\), we have 
  \begin{equation}
    \left|\int_{x_{i-1}}^{x_{i}} f''(y)\frac{(y-x_{i-1})^3}{3!} dy \right| \le C x_i^{-\alpha/2-2} h_i^4
  \end{equation}
  and similarly,
  \begin{equation}
    \left|\int_{x_{i}}^{x_{i+1}} f''(y)\frac{(x_{i+1}-y)^3}{3!} dy \right| \le C x_i^{-\alpha/2-2} h_{i+1}^4
  \end{equation}
  Thus for \(1\le i\le N\), with \cref{lmm:hilexi} we have
  \begin{equation}
    \begin{aligned}
      & \frac{2}{h_i + h_{i+1}} \left| \frac{1}{h_i} \int_{x_{i-1}}^{x_{i}} f''(y) \frac{(y-x_{i-1})^3}{3!} dy + \frac{1}{h_{i+1}} \int_{x_{i}}^{x_{i+1}} f''(y) \frac{(y-x_{i+1})^3}{3!} dy \right| \\
       & \quad \le C x_i^{-\alpha/2-2} \frac{2}{h_i + h_{i+1}} (h_i^3 + h_{i+1}^3)
       \simeq x_i^{-\alpha/2-2} h_i^2 \simeq x_i^{-\alpha/2-2} h^2 x_i^{2-2/r} \\
       &\quad = C h^2 x_i^{-\alpha/2-2/r}
    \end{aligned}
  \end{equation}
  It's symmetric for \(N < i \le 2N-1\).
\end{proof}





\begin{lemma} \label{lmm:Dyjleh2ya/2m2/r}
  By a standard error estimate for linear interpolation, and \cref{cor:regularity-u},
  There is a constant \(C=C(T, \alpha, r, \|u\|_{\beta+\alpha}^{(-\alpha/2)})\) for \(2\le j \le N\),
  \begin{equation}
    |u(y)-\Pi_hu(y)| \le C h^2 y^{\alpha/2-2/r}, \quad \text{for } y\in [x_{j-1}, x_{j}]
  \end{equation}
  symmetricly, for \(N <j \le 2N-1\), we have
  \begin{equation} \label{lmm:Dyjleh22T-y}
    |u(y) - \Pi_hu(y)| \le C h^2 (2T-y)^{\alpha/2-2/r}
  \end{equation}
\end{lemma}
% \begin{proof}
%   For  \(2\le j \le N\),  we have
%   \begin{equation*}
%     x_j \le 2^r y, \quad x_{j-1} \ge 2^{-r} y
%   \end{equation*}
%   And by \cref{lmm:Dyj}, \cref{lmm:hilexi} and \cref{cor:regularity-u}, we have
%   \begin{equation*}
%     \begin{aligned}
%       u(y)-\Pi_hu(y) & = -\frac{\theta (1-\theta)}{2} h_j^2 u''(\xi), \quad \xi \in [x_{j-1}, x_j]                         \\
%                   & \le \frac{\|u\|_{\beta+\alpha}^{(-\alpha/2)}}{4} r^2T^{2/r} \; h^2 x_j^{2-2/r} x_{j-1}^{\alpha/2-2} \\
%                   & \le C h^2 \; 2^{2r-2} y^{2-2/r} \; 2^{-r(\alpha/2-2)} y^{\alpha/2-2}                                \\
%                   & = C 2^{-r\alpha/2+4r-2} \; h^2 y^{\alpha/2-2/r}
%     \end{aligned}
%   \end{equation*}
%   

% \end{proof}


\begin{lemma} \label{lmm:Dh2ymx1malecy1ma}
  There is a constant \(C=C(\alpha, r)\) such that for all \(1 \le i < N/2\), \\
  \(\max\{2i+1, i+3\} \le j \le 2N\) , we have
  \begin{equation}
    D_h^2 K_y (x_i) \le C \frac{y^{-1-\alpha}}{\Gamma(-\alpha)}, \quad y\in [x_{j-1}, x_{j}]
  \end{equation}
\end{lemma}
\begin{proof}
  Sinec \(y\ge x_{j-1} > x_{i+1}\), by \cref{lmm:Dh2simd2}, if \(j-1>i+1\)
  \begin{equation*}
    \begin{aligned}
      D_h^2 K_y (x_i) & = K_y''(\xi)= \frac{|y-\xi|^{-1-\alpha}}{\Gamma(-\alpha)}, \quad \xi\in (x_{i-1}, x_{i+1}) \\
                      & \le \frac{(y-x_{i+1})^{-1-\alpha}}{\Gamma(-\alpha)}                             \\
                      & \le (1-(\frac{2}{3})^r)^{-1-\alpha}\frac{y^{-1-\alpha}}{\Gamma(-\alpha)}
    \end{aligned}
  \end{equation*}
\end{proof}


\begin{lemma} \label{lmm:Dh2xmy1malecx1ma}
  There is a constant \(C=C(\alpha, r)\) such that for all \(3 \le i \le N, k = \lceil \frac{i}{2} \rceil\),
  \(1 \le j \le k-1\) and \(y\in [x_{j-1}, x_{j}]\), we have
  \begin{equation}
    D_h^2 K_y (x_i) \le C \frac{x_i^{-1-\alpha}}{\Gamma(-\alpha)}
  \end{equation}
\end{lemma}
\begin{proof}
  Sinec \(y \le x_{j} < x_{i-1}\), by \cref{lmm:Dh2simd2},
  \begin{equation*}
    \begin{aligned}
      D_h^2 K_y (x_i) & = \frac{|\xi-y|^{-1-\alpha}}{\Gamma(-\alpha)}, \quad \xi\in (x_{i-1}, x_{i+1})                                    \\
                                                                & \le \frac{(x_{i-1}-x_{j})^{-1-\alpha}}{\Gamma(-\alpha)}  \le \frac{(x_{i-1}-x_{k-1})^{-1-\alpha}}{\Gamma(-\alpha)} \\
                                                                & \le ((\frac{2}{3})^r-(\frac{1}{2})^r)^{-1-\alpha}\frac{x_i^{-1-\alpha}}{\Gamma(-\alpha)}
    \end{aligned}
  \end{equation*}
\end{proof}

\textcolor{brown}{
  \begin{lemma} \label{lmm:Ti1}
    While \(0 \le i < N/2\),
    By \cref{lmm:Dyj1}
    \begin{equation}
      \begin{aligned}
        |T_{i1}| & \le C \int_{0}^{x_1} x_1^{\alpha/2} \frac{|x_i-y|^{1-\alpha}}{\Gamma(2-\alpha)} dy                                             \\
                 & = C \frac{1}{\Gamma(3-\alpha)} x_1^{\alpha/2}  \left| x_i^{2-\alpha} - |x_i-x_1|^{2-\alpha} \right|                            \\
                 & \le C \frac{1}{\Gamma(3-\alpha)} x_1^{\alpha/2 + 2-\alpha} =  C \frac{1}{\Gamma(3-\alpha)} x_1^{2-\alpha/2} \quad 0<2-\alpha<1
      \end{aligned}
    \end{equation}
    For \(2\le j \le N\), by \cref{lmm:Dyj} and \cref{cor:regularity-u}
    \begin{equation}
      \begin{aligned}
        |T_{ij}| & \le \frac{C}{4} \int_{x_{j-1}}^{x_j} h_j^2 x_{j-1}^{\alpha/2-2}  \frac{|y-x_i|^{1-\alpha}}{\Gamma(2-\alpha)} dy                \\
                 & \le \frac{C}{4 \Gamma(3-\alpha)}  h_j^2 x_{j-1}^{\alpha/2-2}  \left| |x_j-x_i|^{2-\alpha} - |x_{j-1}-x_{i}|^{2-\alpha} \right| \\
      \end{aligned}
    \end{equation}
  \end{lemma}
}


\begin{lemma} \label{lmm:sumSij13}
  There exists a constant \(C=C(T, \alpha, r, \|u\|_{\beta+\alpha}^{(-\alpha/2)})\) such that
  \begin{equation}
    \sum_{j=1}^{3} S_{1j} \le C h^2 x_1^{-\alpha/2-2/r}
  \end{equation}
  \textcolor{orange}{
    \begin{equation}
      \sum_{j=1}^{4} S_{2j} \le C h^2 x_2^{-\alpha/2-2/r}
    \end{equation}
  }
\end{lemma}
\begin{proof}
  \begin{equation*}
    \begin{aligned}
      S_{1j} = \frac{2}{x_2}\left( \frac{1}{x_1}T_{0j} - \left(\frac{1}{x_1} + \frac{1}{h_2}\right) T_{1j} + \frac{1}{h_2} T_{2j} \right) \\
    \end{aligned}
  \end{equation*}
  So, by \cref{lmm:Ti1}
  \begin{equation*}
    S_{11} \le \frac{2}{x_2 x_1} 4 \frac{C}{\Gamma(3-\alpha)} x_1^{2-\alpha/2} \le C x_1^{-\alpha/2}
  \end{equation*}
  \begin{equation*}
    \begin{aligned}
      S_{12} & \le  \frac{2}{x_2 x_1} \frac{C}{4\Gamma(3-\alpha)} h_2^2 x_{1}^{\alpha/2-2}
      \left(  x_2^{2-\alpha} + 2 h_2^{2-\alpha} + h_2^{2-\alpha}    \right)
      \le C x_1^{-\alpha/2}
    \end{aligned}
  \end{equation*}
  \begin{equation*}
    \begin{aligned}
      S_{13} & \le  \frac{2}{x_2 x_1} \frac{C}{4\Gamma(3-\alpha)} h_3^2 x_{2}^{\alpha/2-2}
      \left(  x_3^{2-\alpha} + 2 x_3^{2-\alpha} + h_3^{2-\alpha}    \right)
      \le C x_1^{-\alpha/2}
    \end{aligned}
  \end{equation*}
  But
  \begin{equation*}
    x_1^{-\alpha/2} = T^{2/r} h^2 x_1^{-\alpha/2-2/r}
  \end{equation*}
  \(i=2\) is similar.
\end{proof}


\textcolor{blue}{
  \begin{lemma} \label{lmm:dhj-i3le-Ch2xi1-2r}
    There exists a constant \(C=C(T, r, l)\) such that
    For \(3\le i \le N-1, \lceil\frac{i}{2}\rceil \le j \le \min\{2i, N\}\), \\
    when \(\xi\in (x_{i-1}, x_{i+1})\),
    \begin{equation}
      (h_{j-i}^3(\xi))' \le (r-1) C h^2  x_{i}^{1-2/r} h_{j}
    \end{equation}
    \begin{equation}
      (h_{j-i}^4(\xi))' \le (r-1) C h^2 x_{i}^{1-2/r} h_{j}^2
    \end{equation}
  \end{lemma}
  }
  \begin{proof}
    From \eqref{def:xi2yj-jlN}
    \begin{gather} \label{eq:yj-i-diff}
      y_{j-i}'(x) = y_{j-i}^{1-1/r}(x) x^{1/r-1}  \\
      y_{j-i}''(x) = \frac{1-r}{r} y_{j-i}^{1-2/r}(x) x^{1/r-2} Z_{j-i}
    \end{gather}
    For \(\xi\in (x_{i-1}, x_{i+1})\) and \(2\le k \le j \le \min\{2i-1, N-1\}\), using \cref{lmm:hilexi}
    \textcolor{purple}{
    \begin{equation*}
      \xi \simeq x_i \simeq x_j
    \end{equation*}
    }
    \begin{equation*}
      \begin{aligned}
        h_{j-i}(\xi) \simeq h_j \simeq h x_{j}^{1-1/r} \simeq  hx_{i}^{1-1/r} \\
      \end{aligned}
    \end{equation*}
    % \textcolor{purple}{
      % \begin{equation*}
      %   h_{j-i}(\xi) \simeq h x_i^{1-1/r}
      % \end{equation*}
      % }
      
      \begin{equation}
        \begin{aligned}
          h_{j-i}'(\xi) &= y_{j-i}'(\xi)-y_{j-i-1}'(\xi)  \\
          &= \xi^{1/r-1} (y_{j-i}^{1-1/r}(\xi)- y_{j-i-1}^{1-1/r}(\xi)) \\
        \end{aligned}
      \end{equation}
      Since
      \begin{equation}
        \begin{aligned}
          y_{j-i}^{1-1/r}(\xi)- y_{j-i-1}^{1-1/r}(\xi) 
          & \le x_{j+1}^{1-1/r} - x_{j-2}^{1-1/r} \\
          &= T^{1-1/r} N^{1-r}( (j+1)^{r-1} - (j-2)^{r-1} ) \\
          &\le C (r-1) j^{r-2} N^{1-r} \\
          &= C (r-1) h x_j^{1-2/r}
        \end{aligned}
      \end{equation}
      Therefore,
      \begin{equation} \label{eq:hj-i'}
        h_{j-i}'(\xi) \le C x_i^{1/r-1} (r-1) h x_j^{1-2/r} \simeq (r-1) h x_i^{-1/r}
      \end{equation}
      
    for \(l=3, 4\)
    \begin{equation}
      \begin{aligned}
        (h_{j-i}^l(\xi))' & = l h_{j-i}^{l-1}(\xi) h_{j-i}'(\xi) \\
        &\le C  h_{j-i}^{l-1}(\xi) (r-1) h x_i^{-1/r} \\
        &\simeq C h_{j}^{l-2} h x_j^{1-1/r} (r-1) h x_i^{-1/r} \\
        &\simeq C (r-1) h^2 x_i^{1-2/r} h_{j}^{l-2}
      \end{aligned}
    \end{equation}
  % \textcolor{pink}{
  % And
  % \begin{equation}
  %   2^{-r}x_i  \le x_{i-1} \le \xi \le x_{i+1} \le 2^r x_i
  % \end{equation}
  % We have
  % \begin{equation}
  %   \xi^{1/r-m} \le 2^{|mr-1|} x_i^{1/r-m}, \quad m=1,2
  % \end{equation}
  % }
  % but
  % \begin{equation}
  %   \begin{aligned}
  %     y_{j-i}^{1-1/r}(\xi)- y_{j-i-1}^{1-1/r}(\xi) & = (\xi^{1/r}+ Z_{j-i})^{r-1} - (\xi^{1/r}+ Z_{j-i-1})^{r-1}           \\
  %                                                  & = (r-1)Z_{1} (\xi^{1/r}+ Z_{j-i-\gamma})^{r-2}, \quad \gamma\in [0,1] \\
  %                                                  & = (r-1)T^{1/r} h y_{j-i-\gamma}^{1-2/r}(\xi)
  %   \end{aligned}
  % \end{equation}
  % \textcolor{purple}{
  %   \begin{equation*}
  %     x_j \simeq x_i, \quad \text{for} \; k-1 \le j \le \min\{2i-1, N-1\}
  %   \end{equation*}
  % }
  % Therefore,
  % \textcolor{purple}{
  %   \begin{equation}
  %     y_{j-i}^{1-1/r}(\xi)- y_{j-i-1}^{1-1/r}(\xi) \le C(r-1) h x_j^{1-2/r} \le C (r-1) h x_i^{1-2/r}
  %   \end{equation}
  % }

  % \begin{equation} \label{eq:xj-2-xjp1-xi}
  %   4^{-r} x_i \le x_{\lceil\frac{i}{2}\rceil-1} \le x_{j-2} = y_{j-i-1}(x_{i-1}) \le  y_{j-i-\gamma}(\xi) \le y_{j-i}(x_{i+1}) = x_{j+1} \le x_{2i} \le 2^r x_{i} \\
  % \end{equation}
  % \begin{equation}
  %   y_{j-i-\gamma}^{1-2/r}(\xi) \le 2^{2|r-2|} x_i^{1-2/r}
  % \end{equation}
  % \textcolor{pink}{
  % But expecially for \(i=3, j=2\), 
  % \begin{equation*}
  %   y_{-1}^{1-1/r}(\xi)- y_{-2}^{1-1/r}(\xi) 
  %   \le \max \begin{cases}
  %     x_3^{1-1/r} - x_2^{1-1/r} \\
  %     x_1^{1-1/r} - 0
  %   \end{cases} \le C (r-1) x_1^{1-1/r} \le C (r-1) h x_3^{1-2/r}
  % \end{equation*}
  % }
  % So we can get
  % \begin{equation} \label{eq:yj1-1/r-yj-1}
  %   y_{j-i}'(\xi)- y_{j-i-1}'(\xi) \le C (r-1) h x_i^{-1/r}
  % \end{equation}

  % We get
  % \begin{equation}
  %   (h_{j-i}^l(\xi))' \le C(r-1) \; h_{j}^{l-1}  h  x_i^{-1/r}  \\
  % \end{equation}
  % \textcolor{pink}{
  % And by \cref{lmm:hilexi},
  % \begin{equation}
  %   h_{j+1} \le rT h \left(\frac{j+1}{N}\right)^{r-1} \le rTh 2^{r-1} \left(\frac{j-1}{N}\right) = 2^{r-1} h_{j}
  % \end{equation}
  % \begin{equation}
  %   h_{j+1} \le rT^{1/r} h x_{j+1}^{1-1/r} \le rT^{1/r} h x_{2i}^{1-1/r} \le rT^{1/r}2^{r-1} h x_{i}^{1-1/r}
  % \end{equation}
  % }
  % \begin{equation}
  %   x_{i+1}^{1/r-1} \le (\frac{4}{3})^{1/r-1} x_{i}^{1/r-1}
  % \end{equation}
  % We can get
  % \begin{equation}
  %   \begin{aligned}
  %     (h_{j-i}^l(\xi))' & \le l(r-1)C \; h_j^{l-2}  h_{j}  h  x_i^{-1/r}        \\
  %                       & \le l (r-1)C h h_j^{l-2} ( h x_{i}^{1-1/r})  x_i^{-1/r} \\
  %                       & = (r-1)C \; h^2   x_i^{1-2/r} h_j^{l-2}
  %   \end{aligned}
  % \end{equation}
  Meanwhile, we can get
  \begin{gather} \label{eq:hj-i3leCh2x2-2rhj}
    h_{j-i}^3(\xi) \simeq h_{j}^3 \le C h^{2} x_{i}^{2-2/r}  h_j \\
    h_{j-i}^4(\xi) \simeq h_{j}^4 \le C h^{2} x_{i}^{2-2/r}  h_j^2
  \end{gather}
\end{proof}


\textcolor{blue}{
  \begin{lemma} \label{lmm:d2hj-i3leh2xi-2r}
    There exists a constant \(C=C(T, r, l)\) such that
    For \(3\le i \le N-1, \lceil\frac{i}{2}\rceil \le j \le \min\{2i, N\}\), \\
    when \(\xi \in (x_{i-1}, x_{i+1})\),
    \begin{equation}
      (h_{j-i}^3(\xi))'' \le C(r-1)  h^2  x_{i}^{-2/r} h_{j}
    \end{equation}
  \end{lemma}
}
\begin{proof}
  % From \eqref{eq:yj-i-diff}
  \begin{equation}
    \begin{aligned}
      (h_{j-i}^3(\xi))'' &= 6 h_{j-i}(\xi) (h_{j-i}'(\xi))^2 + 3 h_{j-i}^2(\xi) h_{j-i}''(\xi) \\
      % & = 6 h_{j-i}(\xi)(y_{j-i}'(\xi)-y_{j-i-1}'(\xi))^2 + 3 h_{j-i}^2(\xi)( y_{j-i}''(\xi)-y_{j-i-1}''(\xi))               \\
                        %  & = 6 h_{j-i}(\xi) (h_{j-i}'(\xi))^2  \\
                        %  & \quad + 3 \frac{1-r}{r} h_{j-i}^2(\xi)\xi^{1/r-2}( y_{j-i}^{1-2/r}(\xi)  Z_{j-i} - y_{j-i-1}^{1-2/r}(\xi) Z_{j-i-1}) \\
    \end{aligned}
  \end{equation}
  By \cref{eq:hj-i'}
  \begin{equation}
    \begin{aligned}
      h_{j-i}(\xi) (h_{j-i}'(\xi))^2  \le C h_j (r-1)^2 h^2 x_{i}^{-2/r}
    \end{aligned}
  \end{equation}
  For the second partial
  \begin{equation}
    \begin{aligned}
      h_{j-i}''(\xi) &= y_{j-i}''(\xi) - y_{j-i-1}''(\xi) \\
        &= \frac{1-r}{r}\xi^{1/r-2} ( y_{j-i}^{1-2/r}(\xi)  Z_{j-i} - y_{j-i-1}^{1-2/r}(\xi) Z_{j-i-1})                                        \\
        &=  \frac{1-r}{r}\xi^{1/r-2} ( (y_{j-i}^{1-2/r}(\xi)-y_{j-i-1}^{1-2/r}(\xi)) Z_{j-i} +  y_{j-i-1}^{1-2/r}(\xi) Z_{1})
    \end{aligned}
  \end{equation}
  but
  \begin{equation}
    \begin{aligned}
      |y_{j-i}^{1-2/r}(\xi)-y_{j-i-1}^{1-2/r}(\xi)|
       & \le |x_{j+1}^{1-2/r} - x_{j-2}^{1-2/r}| \\
       & = T^{1-2/r} N^{2-r}| (j+1)^{r-2} - (j-2)^{r-2} | \\
       & \le C|r-2| N^{2-r} j^{r-3} \\
       & = C|r-2| h x_j^{1-3/r}
    \end{aligned}
  \end{equation}
  % \begin{equation}
  %   \begin{aligned}
  %     y_{j-i}^{1-2/r}(\xi)-y_{j-i-1}^{1-2/r}(\xi)
  %      & = (\xi^{1/r} + Z_{j-i})^{r-2} - (\xi^{1/r} + Z_{j-i-1})^{r-2} \\
  %      & = (r-2) Z_1 (\xi^{1/r} + Z_{j-i-\gamma})^{r-3}                \\
  %      & = (r-2) T^{1/r} h y_{j-i-\gamma}^{1-3/r}(\xi)                  \\
  %      & \le C(r-2) h x_i^{1-3/r}
  %   \end{aligned}
  % \end{equation}
  So we can get
  \begin{equation} \label{eq:hj-i''}
    \begin{aligned}
      |h_{j-i}''(\xi)| &\le C(r-1)x_i^{1/r-2} (|r-2|h x_i^{1-3/r}x_i^{1/r}+x_i^{1-2/r}h) \\
      &\le C (r-1) h x_{i}^{-1-1/r}
    \end{aligned}
  \end{equation}
  Summarizes, we have
  \begin{equation}
    (h_{j-i}^3(\xi))'' \le C(r-1)  h^2  x_{i}^{-2/r} h_{j}
  \end{equation}
\end{proof}

\begin{proof}[proof of \cref{lmm:estimatesofu2yj-itheta}] \label{prf:estimatesofu2yj-itheta}
  From \eqref{def:xi2yj-jlN}
  \begin{gather}
    y_{j-i}'(x) = y_{j-i}^{1-1/r}(x) x^{1/r-1}  \\
    y_{j-i}''(x) = \frac{1-r}{r} y_{j-i}^{1-2/r}(x) x^{1/r-2} Z_{j-i}
  \end{gather}
  Since
  % \begin{equation*}
  %   x_{j-2} \le y_{j-i-1}(x_{i-1}) \le  y_{j-i}^\theta(\xi) \le y_{j-i}^\theta(x_{i+1}) \le x_{j+1}
  % \end{equation*}
  \begin{equation*}
    y_{j-i}^\theta(\xi)  \simeq x_j \simeq x_i
  \end{equation*}
  We have known
  \textcolor{purple}{
  \begin{equation}
    u''(y_{j-i}^{\theta}(\xi)) \le C ({y_{j-i}^{\theta}}(\xi))^{\alpha/2-2} \simeq  x_j^{\alpha/2-2} \simeq x_i^{\alpha/2-2}
  \end{equation}}
  \begin{equation}
    \begin{aligned}
      (u''(y_{j-i}^{\theta}(\xi)))' & = u'''(y_{j-i}^{\theta}(\xi)) ({y_{j-i}^{\theta}}(\xi))'              \\
                                    & \le C x_i^{\alpha/2-3} \xi^{1/r-1} y_{j-i}^{1-1/r}(\xi)             \\
                                    & \simeq  x_i^{\alpha/2-3} x_i^{1/r-1} x_i^{1-1/r} = C x_i^{\alpha/2-3}
    \end{aligned}
  \end{equation}
  \begin{equation}
    \begin{aligned}
      (u''(y_{j-i}^{\theta}(\xi)))'' & = u''''(y_{j-i}^{\theta}(\xi)) ({y_{j-i}^{\theta}}'(\xi))^2 + u'''(y_{j-i}^{\theta}(\xi)) {y_{j-i}^{\theta}}''(\xi) \\
                                     & \le C x_i^{\alpha/2-4}  + C x_i^{\alpha/2-3} \frac{r-1}{r} x_i^{1-2/r} x_i^{1/r-2} Z_{|j-i|+1}                       \\
                                     & \le C x_i^{\alpha/2-4} + C\frac{r-1}{r} x_i^{\alpha/2-3}  x_i^{-1-1/r} x_i^{1/r}                                       \\
                                     & = C x_i^{\alpha/2-4}
    \end{aligned}
  \end{equation}
\end{proof}

\begin{proof}[Proof of \cref{lmm:estimatesofy-xi1-a}] \label{prf:estimatesofy-xi1-a}
  \begin{equation} \label{eq:yj-it-x}
    \begin{aligned}
      |{y_{j-i}^\theta}(\xi)-\xi|
       & = |\theta(y_{j-i-1}(\xi)-\xi) + (1-\theta)(y_{j-i}(\xi)-\xi)| \\
       & = \theta |y_{j-i-1}(\xi)-\xi| + (1-\theta)|y_{j-i}(\xi)-\xi|
    \end{aligned}
  \end{equation}
  where \(y_{j-i-1}(\xi)-\xi\) and \(y_{j-i}(\xi)-\xi\) have the same sign (\(\ge 0\) or \(\le 0\)), independent with \(\xi\).

  Since \(|{y_{j-i}}(\xi)-\xi|=\text{sign}(j-i)({y_{j-i}}(\xi)-\xi)\) is increasing with \(\xi\),
  \begin{equation} \label{eq:yj-ixi-xi-sim-xj-xi}
    (\frac{i-1}{i})^r |x_{j} - x_{i}| \le |x_{j-1} - x_{i-1}| \le |{y_{j-i}}(\xi)-\xi| \le |x_{j+1} - x_{i+1}| \le (\frac{i+1}{i})^r |x_{j} - x_{i}|
  \end{equation}
  we have
  \begin{equation} \label{eq:yj-i-x-sim-xj-xi}
    |{y_{j-i}}(\xi)-\xi| \simeq |x_j - x_i|
  \end{equation}
  Similarly, \(|{y_{j-1-i}}(\xi)-\xi| \simeq |x_{j-1} - x_i|\).
  % Thus,
  % \begin{equation} 
  %   (\frac{2}{3})^r |y_j^\theta - x_i| \le |{y_{j-i}^\theta}(\xi)-\xi| \le (\frac{3}{4})^r (\theta |x_{j} - x_{i}| + (1-\theta) |x_{j-1} - x_{i}|) = (\frac{3}{4})^r |y_j^\theta - x_i|
  % \end{equation}
  % \begin{equation}
  %   |{y_{j-i}^\theta}(\xi)-\xi|^{1-\alpha} \le C  |y_j^\theta - x_i|^{1-\alpha}
  % \end{equation}
  \textcolor{purple}{
    Thus, with \eqref{eq:yj-it-x} , \eqref{eq:yj-i-x-sim-xj-xi} and \eqref{def:yjt}  we get
  \begin{equation} \label{eq:yj-it-x-sim-xjt-xi}
    |{y_{j-i}^\theta}(\xi)-\xi| \simeq |y_j^\theta - x_i|
  \end{equation}
  }
  Next, since \(|y_{j-i}^\theta(\xi)-\xi| = \text{sign}(j-i-1+\theta)(y_{j-i}^\theta(\xi)-\xi)\), so we can derivate it.
  \begin{equation}
    \begin{aligned}
      |(|{y_{j-i}^\theta}(\xi)-\xi|^{1-\alpha})'| 
      & = (\alpha-1)|{y_{j-i}^\theta}(\xi)-\xi|^{-\alpha}|({y_{j-i}^\theta}(\xi))'-1|
    \end{aligned}
  \end{equation}
  While, similar with \cref{eq:yj-it-x}, we have
  \begin{equation} 
    |({y_{j-i}^\theta}(\xi))'-1| = (1-\theta)|y_{j-i-1}'(\xi) - 1| + \theta |y_{j-i}'(\xi) - 1|
  \end{equation}
  By \cref{ineq:a-b-theta} and \cref{eq:yj-i-x-sim-xj-xi}, we have
  \begin{equation} 
    \begin{aligned}
      |y_{j-i}'(\xi) - 1|
      &= \xi^{1/r-1}|y_{j-i}^{1-1/r}(\xi) - \xi^{1-1/r}| \\
      & \le \xi^{-1} | y_{j-i}(\xi)-\xi| \\
      & \simeq x_i^{-1} |x_{j}-x_{i}|
    \end{aligned}
  \end{equation}
  So similar with \cref{eq:yj-it-x-sim-xjt-xi}, we can get
  \begin{equation} \label{eq:yj-it'-1}
    \begin{aligned}
      |({y_{j-i}^\theta}(\xi))'-1| \le C x_i^{-1} |y_{j}^\theta-x_{i}|
    \end{aligned}
  \end{equation}
  Combine with \cref{eq:yj-it-x-sim-xjt-xi}, we get
  \begin{equation}
    \begin{aligned}
      |(|{y_{j-i}^\theta}(\xi)-\xi|^{1-\alpha})'| 
       \le C |y_j^\theta - x_i|^{-\alpha} x_i^{-1}|y_{j}^\theta - x_i|
       = C |y_j^\theta - x_i|^{1-\alpha} x_i^{-1}
    \end{aligned}
  \end{equation}
  Finally, we have
  \begin{equation}
    \begin{aligned}
      (|{y_{j-i}^\theta}(\xi)-\xi|^{1-\alpha})'' 
      & = \alpha(\alpha-1)|{y_{j-i}^\theta}(\xi)-\xi|^{-\alpha-1}((y_{j-i}^\theta(\xi))'-1)^2                       \\
      & \quad + \text{sign}(j-i-1+\theta)(1-\alpha)|{y_{j-i}^\theta}(\xi)-\xi|^{-\alpha} (y_{j-i}^\theta(\xi))'' \\
    \end{aligned}
  \end{equation}
  % Using the inequalities above ,we have
  % \begin{equation}
  %   \begin{aligned}
  %      & |{y_{j-i}^\theta}(\xi)-\xi|^{-\alpha-1}(\xi^{1/r-1}(\theta y_{j-i-1}^{1-1/r}(\xi) + (1-\theta) y_{j-i}^{1-1/r}(\xi))-1)^2 \\
  %      & \quad \le C |y_j^\theta - x_i|^{-\alpha-1} (x_i^{-1}|y_{j}^\theta - x_i|)^2                                               \\
  %      & \quad = C |y_j^\theta - x_i|^{1-\alpha} x_i^{-2}                                                                          \\
  %   \end{aligned}
  % \end{equation}
  For
  \begin{equation}
    \begin{aligned}
      (y_{j-i}^\theta(\xi))'' &= (1-\theta) y_{j-i-1}''(\xi) + \theta y_{j-i}''(\xi)
    \end{aligned}
  \end{equation}
  and
  \begin{equation}
    \begin{aligned}
      y_{j-i}''(\xi) &= \frac{1-r}{r} y_{j-i}^{1-2/r}(x) x^{1/r-2} Z_{j-i} \\
      & \simeq \frac{1-r}{r} x_j^{1-2/r} x_i^{1/r-2} Z_{j-i}
    \end{aligned}
  \end{equation}
  while by \cref{ineq:a-b-theta}
  \begin{equation}
    |Z_{j-i}| = |x_{j}^{1/r}-x_{i}^{1/r}| \le |x_{j} - x_{i}| x_{i}^{1/r-1}
  \end{equation}
  we have
  \begin{equation}
    |y_{j-i}''(\xi)| \le C(r-1) x_i^{-2} |x_j-x_i|
  \end{equation}
  Therefore
  \begin{equation}
    |(y_{j-i}^\theta(\xi))''| \le C (r-1) x_i^{-2} |y_j^\theta - x_i|
  \end{equation}
  Then, combine with \cref{eq:yj-it'-1},
  \begin{equation}
    |(|{y_{j-i}^\theta}(\xi)-\xi|^{1-\alpha})'' | \le C |y_j^\theta - x_i|^{1-\alpha} x_i^{-2}
  \end{equation}
  % \begin{equation}
  %   \begin{aligned}
  %      & |{y_{j-i}^\theta}(\xi)-\xi|^{-\alpha} \xi^{1/r-2}|\theta y_{j-i-1}^{1-2/r}(\xi)Z_{j-i-1} + (1-\theta) y_{j-i}^{1-2/r}(\xi)Z_{j-i}| \\
  %      & \le C |y_{j}^\theta - x_i|^{-\alpha} x_i^{1/r-2} x_i^{1-2/r} |\theta Z_{j-i-1} + (1-\theta) Z_{j-i}|                               \\
  %      & \le C |y_{j}^\theta - x_i|^{-\alpha} x_i^{-2} |y_j^\theta - x_i|                                                                   \\
  %      & = C |y_{j}^\theta - x_i|^{1-\alpha} x_i^{-2}                                                                                       \\
  %   \end{aligned}
  % \end{equation}
\end{proof}

\begin{proof}[proof of \cref{lmm:dQj-itle}] \label{prf:dQj-itle}
  For \(\lceil\frac{i}{2}\rceil \le j \le \min\{2i-1, N-1\}\)
  \begin{equation}
    \begin{aligned}
       & \frac{{Q_{j-i}^\theta}(x_{i+1}) u'''(\eta_{j+1}^\theta) - {Q_{j-i}^\theta}(x_{i}) u'''(\eta_{j}^\theta)}{h_{i+1}}  \\
       & \quad = \frac{Q_{j-i}^\theta(x_{i+1}) - Q_{j-i}^\theta(x_{i})}{h_{i+1}} u'''(\eta_{j+1}^\theta)
      + Q_{j-i}^\theta(x_i) \frac{u'''(\eta_{j+1}^\theta)-u'''(\eta_{j}^\theta)}{h_{i+1}}                                   \\
      %  & \quad \le {Q_{j-i}^\theta}'(\xi) C x_j^{\alpha/2-3} + Q_{j-i}^\theta(x_i) C u''''(\eta)\frac{h_i+h_{i+1}}{h_{i+1}}
    \end{aligned}
  \end{equation}
  Since mean value theorem
  \begin{equation}
    \frac{Q_{j-i}^\theta(x_{i+1}) - Q_{j-i}^\theta(x_{i})}{h_{i+1}} = {Q_{j-i}^\theta}'(\xi), \quad \xi \in (x_{i}, x_{i+1})
  \end{equation}
  From \eqref{def:Qj-itheta-jlN} and Leibniz rule, by \cref{lmm:dhj-i3le-Ch2xi1-2r} and \cref{lmm:estimatesofy-xi1-a}, we have
  \begin{equation}
    \begin{aligned}
      |{Q_{j-i}^\theta}'(\xi)| \le C h^2 \frac{|y_{j}^\theta - x_{i}|^{1-\alpha}}{\Gamma(2-\alpha)} x_{i}^{1-2/r} h_{j}^2 
    \end{aligned}
  \end{equation}
  And by \cref{def:yj-i}  and \cref{lmm:hilexi}
  \begin{equation}
    Q_{j-i}^\theta(x_i) = h_{j}^4 \frac{|y_j^\theta-x_i|^{1-\alpha}}{\Gamma(2-\alpha)} \simeq C h^2 x_i^{2-2/r}\frac{|y_j^\theta-x_i|^{1-\alpha}}{\Gamma(2-\alpha)} h_j^2
  \end{equation}
  With \(\eta_{j}^\theta \in (x_{j-1}, x_{j})\)
  \begin{equation*}
    u'''(\eta_{j+1}^\theta) \le C (\eta_{j+1}^\theta)^{\alpha/2-3} \simeq x_j^{\alpha/2-3} \simeq x_i^{\alpha/2-3}
  \end{equation*}
  and
  \begin{equation*}
    \begin{aligned}
      \frac{u'''(\eta_{j+1}^\theta)-u'''(\eta_{j}^\theta)}{h_{i+1}}
      &= u''''(\eta) \frac{\eta_{j+1}^\theta - \eta_{j}^\theta}{h_{i+1}}  \\
      & \le C \eta^{\alpha/2-4} \frac{x_{j+1}-x_{j-1}}{h_{i+1}}
      = C \eta^{\alpha/2-4} \frac{h_{j+1}+h_{j}}{h_{i+1}} \\
      & \simeq x_j^{\alpha/2-4} \simeq x_i^{\alpha/2-4}
    \end{aligned}
  \end{equation*}
  So we have
  \begin{equation}
    \begin{aligned}
       & \frac{{Q_{j-i}^\theta}(x_{i+1}) u'''(\eta_{j+1}^\theta) - {Q_{j-i}^\theta}(x_{i}) u'''(\eta_{j}^\theta)}{h_{i+1}} \\
       & \quad \le C h^2 \frac{|y_j^\theta - x_i|^{1-\alpha}}{\Gamma(2-\alpha)} x_i^{1-2/r} h_j^2 \; x_i^{\alpha/2-3}
      + C h^2 x_i^{2-2/r}\frac{|y_j^\theta-x_i|^{1-\alpha}}{\Gamma(2-\alpha)} h_j^2x_{j-1}^{\alpha/2-4}                    \\
       & \quad = C h^2 \frac{|y_j^\theta-x_i|^{1-\alpha}}{\Gamma(2-\alpha)} x_i^{\alpha/2-2-2/r} h_j^2
    \end{aligned}
  \end{equation}
  while \(h_j \simeq h_{i}\),
  subsitute into the inequality above, we get the goal
  \begin{equation}
    \begin{aligned}
      \frac{2}{h_{i} + h_{i+1}} & \left( \frac{{Q_{j-i}^\theta}(x_{i+1}) u'''(\eta_{j+1}^\theta) - {Q_{j-i}^\theta}(x_{i}) u'''(\eta_{j}^\theta)}{h_{i+1}}\right) \\
                                & \le \frac{1}{h_{i}} C h^2 \frac{|y_j^\theta - x_i|^{1-\alpha}}{\Gamma(2-\alpha)} x_i^{\alpha/2-2-2/r} h_j  h_i               \\
                                & = C h^2 \frac{|y_j^\theta - x_i|^{1-\alpha}}{\Gamma(2-\alpha)} x_i^{\alpha/2-2-2/r} h_j
    \end{aligned}
  \end{equation}
  While, the later is similar.
\end{proof}









\textcolor{blue}{
  \begin{lemma} \label{lmm:estimate-hj-i-j>N}
    There exists a constant \(C=C(T, r)\) such that
    For \(N/2\le i \le N-1\), \(N+2 \le j \le 2N-\lceil\frac{N}{2}\rceil+1\), \(l=3, 4\) , \(\xi\in (x_{i-1}, x_{i+1})\),  we have
    \begin{gather}
      h_{j-i}^l(\xi) \le C h_j^l \le  C h^2  h_j^{l-2}     \\
      (h_{j-i-1}^l(\xi))' \le C(r-1) h^2 h_j^{l-2}    \\
      (h_{j-i}^3(\xi))'' \le C(r-1) h^2 h_j
    \end{gather}
  \end{lemma}
}
\begin{proof}
  \begin{equation}
    \begin{aligned}
      (h_{j-i}(\xi))' & = {y_{j-i}}'(\xi) - {y_{j-i-1}}'(\xi)                                              \\
                      & = \xi^{1/r-1} ( (2T-y_{j-i}(\xi))^{1-1/r}  - (2T-y_{j-i-1}(\xi))^{1-1/r} )  \le  0
    \end{aligned}
  \end{equation}
  Thus,
  \begin{equation}
    C h_{j}  \le h_{j+1} \le  h_{j-i}(\xi) \le h_{j-i}(x_{i-1}) = h_{j-1} \le C h_{j}
  \end{equation}
  So as \(4^{-r}T \le 2T-x_j \le T, 2^{-r}T \le x_i \le T \), we have
  \begin{equation}
    h_{j-i}^l(\xi) \le C h_j^l \le  C h^2 (2T-x_{j})^{2-2/r} h_j^{l-2} \le C h^2 h_j^{l-2}
  \end{equation}

  Since
  \begin{equation}
    \begin{aligned}
       & |(2T-y_{j-i}(\xi))^{1-1/r}  - (2T-y_{j-i-1}(\xi))^{1-1/r}|                       \\
       & \quad = |(Z_{2N-(j-i)} - \xi^{1/r})^{r-1} - (Z_{2N-(j-1-i)} - \xi^{1/r})^{r-1} | \\
       & \quad = (r-1) Z_1 (Z_{2N-(j-i-\gamma)} - \xi^{1/r})^{r-2} \quad \gamma \in [0,1] \\
       & \quad \le C (r-1) h (2T-x_j)^{1-2/r}
    \end{aligned}
  \end{equation}
  we have
  \begin{equation}
    |(h_{j-i}(\xi))'| \le C(r-1)h (2T-x_j)^{1-2/r} x_i^{1/r-1}
  \end{equation}
  And
  \begin{equation}
    \begin{aligned}
      (h_{j-i}^l(\xi))' & = lh_{j-i}^{l-1}(\xi){h_{j-i}}'(\xi)                   \\
                        & \le C(r-1) h_j^{l-1} \; h (2T-x_j)^{1-2/r} x_i^{1/r-1} \\
                        & \le C(r-1) h^2 h_j^{l-2} (2T-x_j)^{2-3/r} x_i^{1-1/r}  \\
                        & \le C(r-1) h^2 h_j^{l-2}
    \end{aligned}
  \end{equation}

  \begin{equation}
    \begin{aligned}
      (h_{j-i}^3(\xi))'' & = 6 h_{j-i}(\xi) ({y_{j-i}}'(\xi) - {y_{j-i-1}}'(\xi))^2 + 3 h_{j-i}^2(\xi) ({y_{j-i}}''(\xi) - {y_{j-i-1}}''(\xi))                         \\
                         & \le C(r-1) h_j h^2 + C h_j^2 \frac{1-r}{r}\xi^{1/r-2}( (2T-y_{j-i}(\xi))^{1-2/r}Z_{2N-(j-i)} - (2T-y_{j-i-1}(\xi))^{1-2/r} Z_{2N-(j-1-i)} ) \\
                         & \le C(r-1) h_j h^2 + C (r-1) h_j^2 (C(r-2)h(2T-x_j)^{1-3/r}Z_{2N-(j-i)} + Z_1(2T-x_{j-1})^{1-2/r} )                                         \\
                         & \le C(r-1) h_j h^2 + C (r-1) h_j^2 h  = C h^2 h_j
    \end{aligned}
  \end{equation}

\end{proof}

\textcolor{blue}{
  \begin{lemma} \label{lmm:estimate-d2uj-i-j>N}
    There exists a constant \(C=C(T, \alpha, r, \|u\|_{\beta+\alpha}^{(-\alpha/2)})\) such that
    For \(N/2\le i \le N-1\), \(N+2 \le j \le 2N-\lceil\frac{N}{2}\rceil+1\) , \(\xi\in (x_{i-1}, x_{i+1})\),  we have
    \begin{gather}
      u''(y_{j-i}^\theta(\xi)) \le C \\
      (u''(y_{j-i}^\theta(\xi)))' \le C \\
      (u''(y_{j-i}^\theta(\xi)))'' \le C
    \end{gather}
  \end{lemma}
}
\begin{proof}
  \begin{equation}
    x_{j-2} \le y_{j-i}^\theta(\xi) \le x_{j+1} \Rightarrow 4^{-r}T \le 2T-y_{j-i}^\theta(\xi) \le T
  \end{equation}
  Thus, for \(l=2,3,4\),
  \begin{equation}
    u^{(l)}(y_{j-i}^\theta(\xi)) \le C (2T - y_{j-i}^\theta(\xi))^{\alpha/2-l} \le C
  \end{equation}
  and
  \begin{equation}
    \begin{aligned}
      (y_{j-i}^\theta(\xi))' & = \theta {y_{j-1-i}}'(\xi) + (1-\theta) {y_{j-i-1}}'(\xi)                                   \\
                             & = \xi^{1/r-1} (\theta (2T-y_{j-1-i}(\xi))^{1-1/r} + (1-\theta) (2T-y_{j-i-1}(\xi))^{1-1/r}) \\
                             & \le C (2T-x_{j-2})^{1-1/r} \le C
    \end{aligned}
  \end{equation}
  With
  \begin{equation}
    Z_{2N-{j-i}} \le 2T^{1/r}
  \end{equation}
  \begin{equation}
    \begin{aligned}
      (y_{j-i}^\theta(\xi))'' & = \theta {y_{j-1-i}}''(\xi) + (1-\theta) {y_{j-i-1}}''(\xi)                                                                             \\
                              & = \frac{1-r}{r} \xi^{1/r-2} (\theta  (2T-y_{j-i-1}(\xi))^{1-2/r} Z_{2N-(j-i-1)}  + (1-\theta)  (2T-y_{j-i}(\xi))^{1-2/r} Z_{2N-(j-i)} ) \\
                              & \le C (r-1)
    \end{aligned}
  \end{equation}
  Therefore,
  \begin{equation}
    \begin{aligned}
      (u''(y_{j-i}^\theta(\xi)))' & = u'''(y_{j-i}^\theta(\xi)) (y_{j-i}^\theta(\xi))' \\
                                  & \le C
    \end{aligned}
  \end{equation}
  \begin{equation}
    \begin{aligned}
      (u''(y_{j-i}^\theta(\xi)))'' & = u'''(y_{j-i}^\theta(\xi)) ({y_{j-i}^\theta}'(\xi))^2 + u''''(y_{j-i}^\theta(\xi)) {y_{j-i}^\theta}''(\xi) \\
                                   & \le C + C(r-1) = C
    \end{aligned}
  \end{equation}
\end{proof}


\textcolor{blue}{
  \begin{lemma}
    \label{lmm:estimate-|yj-i-xi|^1-a-j>N}
    There exists a constant \(C=C(T, \alpha, r)\) such that
    For \(N/2\le i \le N-1\), \(N+2 \le j \le 2N-\lceil\frac{N}{2}\rceil+1\) , \(\xi\in (x_{i-1}, x_{i+1})\)
    \begin{gather}
      |y_{j-i}^\theta(\xi) - \xi|^{1-\alpha} \le C |y_j^\theta - x_i|^{1-\alpha} \\
      \left| (|y_{j-i}^\theta(\xi) - \xi)^{1-\alpha}|'\right| \le C |y_j^\theta - x_i|^{-\alpha} (|2T - x_i - y_j^\theta| + h_N) \\
      \left|(|y_{j-i}^\theta(\xi) - \xi)^{1-\alpha}|''\right| \le C (r-1) |y_j^\theta - x_i|^{-\alpha} + C |y_j^\theta - x_i|^{-1-\alpha}(|2T - x_i - y_j^\theta| + h_{N})^2
    \end{gather}
  \end{lemma}
}
\begin{proof}
  Since \(y_{j-i-1}(\xi) > x_{j-2} \ge x_N > \xi\)
  \begin{equation} \label{yj-it-xiR}
    y_{j-i}^\theta(\xi) - \xi = (1-\theta) ({y_{j-1-i}}(\xi) - \xi) + \theta ({y_{j-i}}(\xi) - \xi) > 0
  \end{equation}
  \begin{equation}
    \begin{aligned}
      (y_{j-i}(\xi) - \xi)'' & = {y_{j-i}}''(\xi)   \\
      & = \frac{1-r}{r} \xi^{1/r-2}  (2T-y_{j-i}(\xi))^{1-2/r} Z_{2N-(j-i)} \le 0 \\
    \end{aligned}
  \end{equation}
  It's concave, so
  \begin{equation}
    y_{j-i}(\xi) - \xi \ge\min_{\xi\in\{x_{i-1}, x_{i+1}\}} y_{j-i}(\xi) - \xi =\min\{x_{j+1}-x_{i+1}, x_{j-1}-x_{i-1}\} \ge C (x_j - x_i)
  \end{equation}
  With \eqref{yj-it-xiR}, we have
  \begin{equation}
    |y_{j-i}^\theta(\xi) - \xi|^{1-\alpha} \le C |y_j^\theta - x_i|^{1-\alpha}
  \end{equation}
  By \cref{ineq:a-b-theta}
  \begin{equation}
    \begin{aligned}
      |{y_{j-i}}'(\xi) - 1| & = \xi^{1/r-1} | (2T-y_{j-i}(\xi))^{1-1/r} - \xi^{1-1/r} | \\
                            & \le \xi^{-1} | 2T -y_{j-i}(\xi)- \xi | 
    \end{aligned}
  \end{equation}
  \begin{equation}
    \begin{aligned}
      |2T - \xi - y_{j-i}(\xi) | 
      & \le |2T - x_i - x_j| + |x_i - \xi| + |x_j - y_{j-i}(\xi)|   \\
      & \le |2T-x_i-x_j| + h_{i+1} + h_j  \\
      & \le C \left( |2T-x_i-x_j| + h_{N}  \right)
    \end{aligned}
  \end{equation}
  With \(\xi\simeq x_i \simeq 1\),
  \begin{equation}
    |{y_{j-i}}'(\xi) - 1| \le C \left( |2T-x_i-x_j| + h_{N}  \right)
  \end{equation}
  Thus,
  \begin{equation}
    \begin{aligned}
      |({y_{j-i}^\theta}(\xi))' - 1| 
      &\le (1-\theta)|{y_{j-i-1}}'(\xi) - 1| + \theta |{y_{j-i}}'(\xi) - 1| \\
      &\le C \left( (1-\theta)|2T-x_i-x_{j-1}| + \theta |2T-x_i-x_j| +  h_{N}  \right) \\
      &= C \left( |2T-x_i-y_j^\theta| + h_{N}  \right)
    \end{aligned}
  \end{equation}
  So
  \begin{equation}
    \begin{aligned}
      \left|(|y_{j-i}^\theta(\xi) - \xi|^{1-\alpha})'\right| & = |1-\alpha||y_{j-i}^\theta(\xi) - \xi|^{-\alpha} |({y_{j-i}^\theta}(\xi))' - 1|   \\
      & \le C |y_j^\theta - x_i|^{-\alpha} (|2T - x_i - y_j^\theta| + h_{N} )
    \end{aligned}
  \end{equation}
  \begin{equation}
    \begin{aligned}
      \left|(|y_{j-i}^\theta(\xi) - \xi|^{1-\alpha})''\right| 
      & \le |1-\alpha||y_{j-i}^\theta(\xi) - \xi|^{-\alpha} |(y_{j-i}^\theta(\xi) - \xi)''|  + \alpha(\alpha-1) |y_{j-i}^\theta(\xi) - \xi|^{-1-\alpha} ({y_{j-i}^\theta}'(\xi) - 1)^2 \\
      & \le C (r-1) |y_j^\theta - x_i|^{-\alpha} + C |y_j^\theta - x_i|^{-1-\alpha}(|2T - x_i - y_j^\theta| + h_{N})^2
    \end{aligned}
  \end{equation}
\end{proof}








\section*{Acknowledgments}
\textcolor{gray}{
  We would like to acknowledge the assistance of volunteers in putting
  together this example manuscript and supplement.
}
\bibliographystyle{siamplain}
\bibliography{references}
\end{document}
