\documentclass{ctexart}
\usepackage{amsmath, amsfonts, amssymb, amscd, amstext, amsthm} 
\usepackage{graphicx}

\usepackage{setspace}
\usepackage{xcolor}
\usepackage{listings}
\usepackage[CJKbookmarks=true]{hyperref}	% 添加pdf标签
\usepackage{float}
\usepackage{mathrsfs}



\newtheorem{theorem}{Theorem}[section]
\newtheorem{lemma}[theorem]{Lemma}
\newtheorem{proposition}[theorem]{Proposition}
\newtheorem{corollary}[theorem]{Corollary}

\theoremstyle{definition}
\newtheorem{definition}[theorem]{Definition}
\newtheorem{example}[theorem]{Example}
\newtheorem{xca}[theorem]{Exercise}
\newtheorem{question}[theorem]{Question}

\theoremstyle{remark}
\newtheorem{remark}[theorem]{Remark}

\numberwithin{equation}{section}


\begin{document}

\section{问题}

对于 \(\Omega=(0,1)\), \(1<\alpha<2\), 假设 \(f\in C^2(\Omega)\)

\begin{equation}
    \begin{cases}
        (-\Delta)^{\frac{\alpha}{2}} u(x) = f(x), & x \in \Omega                      \\
        u(x) = 0,                                 & x \in \mathbb{R} \setminus \Omega
    \end{cases}
\end{equation}

其中

\begin{equation}
    (-\Delta)^{\frac{\alpha}{2}} u(x) = -\frac{\partial^\alpha u}{\partial |x|^\alpha}
    = C_R \frac{d^2}{dx^2} \int_\Omega \frac{u(y)}{|x-y|^{\alpha-1}} dy
\end{equation}
\begin{equation}
    C_R = \frac{1}{2\cos(\alpha\pi/2)\Gamma(2-\alpha)} < 0
\end{equation}

\section{数值格式}

用线性插值代替原函数,中心差分代替二阶导数,记 \(u_h(x)\) 为 \(u(x)\) 在网格点上的线性插值。

我们解这样的数值解

\begin{equation}
    \begin{aligned}
        C_R ( & \frac{2}{h_{i+1} (h_i + h_{i+1})}\int_\Omega \frac{u_h(x)}{ |x_{i+1}-y|^{\alpha-1} } dy \\
              & -\frac{2}{h_i h_{i+1}}\int_\Omega \frac{u_h(x)}{ |x_i-y|^{\alpha-1} } dy                \\
              & + \frac{2}{h_i (h_i+ h_{i+1})}\int_\Omega \frac{u_h(x)}{ |x_{i-1}-y|^{\alpha-1} } dy
        )                                                                                               \\
        = F_i
    \end{aligned}
\end{equation}


矩阵 \(A\) 是 \(M\) 矩阵,主队角正,其他负,严格对角占优。


\section{一致网格}

当 \(r=1\) , 网格成为一致网格, \(x_i = ih, h=\frac{1}{2N}, i=0, ..., 2N\).

\(A\) 等于

\begin{equation}
    \begin{aligned}
        a_{ij} & = \frac{C_R}{(2-\alpha)(3-\alpha)} h^{-\alpha}                                                                             \\
               & \left( |i-j-2|^{3-\alpha} - 4 |i-j-1|^{3-\alpha} + 6 |i-j|^{3-\alpha} - 4 |i-j+1|^{3-\alpha} +  |i-j+2|^{3-\alpha} \right)
    \end{aligned}
\end{equation}

矩阵行和

\begin{equation}
    \begin{aligned}
        S_i = \sum_{j=1}^{2N-1} a_{ij} & = \frac{C_R}{(2-\alpha)(3-\alpha)} h^{-\alpha}
        ( |i+1|^{3-\alpha} - 3 |i|^{3-\alpha} + 3 |i-1|^{3-\alpha} - |i-2|^{3-\alpha}  + ... 2N )
    \end{aligned}
\end{equation}


我们得到

\begin{equation}
    S_i \ge  C (x_i^{-\alpha} + (1-x_i)^{-\alpha})
\end{equation}


下面估计截断误差 \(R_i\).

\begin{equation}
    \begin{aligned}
        R_i = & \int_0^1 D(y) \frac{ |y-x_{i-1}|^{1-\alpha} - 2|y-x_i |^{1-\alpha} + |y-x_{i+1}|^{1-\alpha} }{h^2} dy \\
        % &+ \underbrace{( I(x_{i-1}) -2I(x_{i}) + I({x_{i+1}}) ) / h^2 - f(x_i)}_{O(h^2)}
    \end{aligned}
\end{equation}



目标是

\begin{equation}
    R_i \le C h^{\alpha/2} S_i
\end{equation}

这样我们就有

\begin{equation}
    \epsilon \le \max_i \frac{R_i}{S_i} \le Ch^{\alpha/2}
\end{equation}

考虑 \(R_1\)

\begin{equation}
    \begin{aligned}
        R_1 = & \int_\Omega (u(y) - u_h(y)) \frac{ |y|^{1-\alpha} - 2|y-h|^{1-\alpha} + |y-2h|^{1-\alpha} }{h^2} dy \\
    \end{aligned}
\end{equation}


我们有

\begin{equation}
    \begin{aligned}
        R_1 = \int_0^{3h} + \int_{3h}^{1/2}
    \end{aligned}
\end{equation}

当 \(y>3h\),

\begin{equation}
    \frac{ |y|^{1-\alpha} - 2|y-h|^{1-\alpha} + |y-2h|^{1-\alpha} }{h^2} \le C |y|^{-1-\alpha}
\end{equation}

那么

\begin{equation}
    \begin{aligned}
        I_2 & \le  C \int_{3h}^{1/2} |y|^{-1-\alpha} u''(y) h^2 dy       \\
            & \le  C \int_{3h}^{1} |y|^{-1-\alpha} y^{\alpha/2-2} h^2 dy \\
            & \le C h^{2}  \int_{3h}^{1} y^{-3-\alpha/2} dy              \\
            & \le C h^{2} h^{-2-\alpha/2} = C h^{-\alpha/2}              \\
            & \le C h^{\alpha/2} x_1^{-\alpha} \le C h^{\alpha/2} S_1
    \end{aligned}
\end{equation}

在考虑

\begin{equation}
    \begin{aligned}
        I_1 & = \int_0^{3h}  \frac{ u(y) - u_h(y) }{h^2} (|y|^{1-\alpha} - 2|y-h|^{1-\alpha} + |y-2h|^{1-\alpha}) dy \\
            & = \int_0^h + \int_h^{3h} = J_1 + J_2
    \end{aligned}
\end{equation}

\begin{equation}
    J_2 \le C u''(\eta) h^{2-\alpha} \le C h^{\alpha/2-2} h^{2-\alpha} \le C h^{-\alpha/2}
\end{equation}

因为

\begin{equation}
    \begin{aligned}
        |u(x) - u_h(x)| & \le \int_0^{x_1} |u'(y)| dy                 \\
                        & \le C \int_0^{x_1} y^{\alpha/2-1} dy        \\
                        & \le C x_1^{\alpha/2}    \quad , x\in (0, h)
    \end{aligned}
\end{equation}


\begin{equation}
    \begin{aligned}
        J_1 & = \int_0^h \frac{ u(y) - u_h(y) }{h^2} (|y|^{1-\alpha} - 2|y-h|^{1-\alpha} + |y-2h|^{1-\alpha}) dy \\
            & \le C h^{\alpha/2-2} h^{2-\alpha} = C h^{-\alpha/2}
    \end{aligned}
\end{equation}

所以有

\begin{equation}
    R_1 \le C h^{-\alpha/2} \le C h^{\alpha/2} h^{-\alpha} \le C h^{\alpha/2} S_1, \quad (S_1\ge C x_1^{-\alpha})
\end{equation}


\(R_1, R_2, R_3\) 全部类似。

\subsection{猜想}

\begin{equation}
    R_i \le C h^{\alpha/2+1} (x_i^{-\alpha-1} + (1-x_i)^{-\alpha-1}) \quad (then \le C h^{\alpha/2} S_i)
\end{equation}

为了简便,我们记 \(D(y) := u(y) - u_h(y)\).

当 \(3<i\le N\) 时,

\begin{equation}
    \begin{aligned}
        R_i = & \int_0^1 D(y) \frac{ |y-x_{i-1}|^{1-\alpha} - 2|y-x_i |^{1-\alpha} + |y-x_{i+1}|^{1-\alpha} }{h^2} dy                                                                                                     \\
        =     & \int_0^{x_1} D(y) \frac{ |y-x_{i-1}|^{1-\alpha} - 2|y-x_i |^{1-\alpha} + |y-x_{i+1}|^{1-\alpha} }{h^2} dy                                                                                                 \\
              & + \int_{x_1}^{x_{\lceil \frac{i}{2}\rceil}} D(y) \frac{ |y-x_{i-1}|^{1-\alpha} - 2|y-x_i |^{1-\alpha} + |y-x_{i+1}|^{1-\alpha} }{h^2} dy                                                                  \\
              & + \int_{x_{\lceil \frac{i}{2}\rceil }}^{x_{\lceil \frac{i}{2}\rceil +1 }} \frac{  D(y+h) - D(y) }{h^2} |y-x_i|^{1-\alpha}  + D(y)\frac{ |y-x_{i+1}|^{1-\alpha} - |y-x_{i}|^{1-\alpha} }{h^2} dy           \\
              & + \int_{x_{\lceil \frac{i}{2}\rceil +1}}^{x_{i}} \frac{  D(y-h) -2D(y) +  D(y+h) }{h^2} |y-x_i|^{1-\alpha} dy                                                                                             \\
        % & + \cdots (2N-i)     \\
              & + \int_{x_{i}}^{x_{N + \lfloor \frac{i}{2}\rfloor -1}} \frac{  D(y-h) -2D(y) +  D(y+h) }{h^2} |y-x_i|^{1-\alpha} dy                                                                                       \\
              & + \int_{x_{N + \lfloor \frac{i}{2}\rfloor -1}}^{x_{N + \lfloor \frac{i}{2}\rfloor}} \frac{  D(y-h) - D(y) }{h^2} |y-x_i|^{1-\alpha}  + D(y)\frac{ |y-x_{i-1}|^{1-\alpha} - |y-x_{i}|^{1-\alpha} }{h^2} dy \\
              & +  \int_{x_{N + \lfloor \frac{i}{2}\rfloor}}^{x_{2N-1}} + \int_{x_{2N-1}}^1 D(y) \frac{ |y-x_{i-1}|^{1-\alpha} - 2|y-x_i |^{1-\alpha} + |y-x_{i+1}|^{1-\alpha} }{h^2} dy                                  \\
        =     & I_1 + I_2 + I_3 + I_4 + \cdots
    \end{aligned}
\end{equation}

1.

\begin{equation}
    \begin{aligned}
        I_1 & = \int_0^{x_1} (u(y) - u_h(y)) \frac{ |y-x_{i-1}|^{1-\alpha} - 2|y-x_i |^{1-\alpha} + |y-x_{i+1}|^{1-\alpha} }{h^2} dy \\
            & \le C h^{\alpha/2} \int_0^{h} |y-x_i|^{-1-\alpha} dy                                                                   \\
            & \le C h^{\alpha/2 + 1} x_{i}^{-1-\alpha}                                                                               \\
    \end{aligned}
\end{equation}

% 显然
% \begin{equation}
%     I_1 \le C h^{\alpha/2+1} h^{-1} x_i^{-\alpha} \le  C h^{\alpha/2} S_i
% \end{equation}

% 且当 \(i\) 与 \(N\) 相当时,有 \(h^{\alpha/2+1}\).
% \newline


2.

\begin{equation}
    \begin{aligned}
        I_2 & = \int_{x_1}^{x_{\lceil \frac{i}{2}\rceil}} D(y) \frac{ |y-x_{i-1}|^{1-\alpha} - 2|y-x_i |^{1-\alpha} + |y-x_{i+1}|^{1-\alpha} }{h^2} dy \\
            & \le C \int_{x_1}^{x_{\lceil \frac{i}{2}\rceil}} y^{\alpha/2 -2} h^2 |x_i - y |^{-1-\alpha} dy                                            \\
            & \le C h^{\alpha/2-1} h^2 x_i^{-1-\alpha} \le C h^{\alpha/2+1} x_i^{-1-\alpha}
    \end{aligned}
\end{equation}


3.


\begin{equation}
    \begin{aligned}
        I_3 & = \int_{x_{\lceil \frac{i}{2}\rceil }}^{x_{\lceil \frac{i}{2}\rceil +1 }} \frac{  D(y+h) - D(y) }{h^2} |y-x_i|^{1-\alpha} + D(y)\frac{ |y-x_{i+1}|^{1-\alpha} - |y-x_{i}|^{1-\alpha} }{h^2} dy \\
            & \le \int_{x_{\lceil \frac{i}{2}\rceil }}^{x_{\lceil \frac{i}{2}\rceil +1 }} u'''(\eta_1) h |x_i-y|^{1-\alpha} + u''(\eta_2) h |x_i-y|^{-\alpha}  dy                                            \\
            & \le C h^2 x_i^{-2-\alpha/2} \le C h^{1+\alpha/2} x_i^{-1-\alpha}
    \end{aligned}
\end{equation}





4.

\begin{equation}
    \begin{aligned}
        I_4 & = \int_{x_{\lceil \frac{i}{2}\rceil +1}}^{x_{i}} \frac{  D(y-h) -2D(y) +  D(y+h) }{h^2} |y-x_i|^{1-\alpha} dy \\
            & \le \int_{x_{\lceil \frac{i}{2}\rceil +1}}^{x_{i}} u''''(\eta) h^2 |x_i - y|^{1-\alpha} dy                    \\
            & \le C x_{i}^{\alpha/2 -4} h^2 x_i^{2-\alpha}                                                                  \\
            & \le C h^2 x_i^{-2-\alpha/2} \le C h^{1+\alpha/2} x_i^{-1-\alpha}
    \end{aligned}
\end{equation}


猜想证毕,一致网格证完。


\section{非一致}

\(r > 1\),

\begin{equation}
    \begin{cases}
        x_i = \frac{1}{2} \left(\frac{i}{N}\right)^r   ,     & 0 \le i \le N  \\
        x_i = 1 - \frac{1}{2} \left(\frac{2N-i}{N}\right)^r, & N \le i \le 2N
    \end{cases}
\end{equation}

令 \(h=\frac{1}{2N}\) , 那么

当 \(i \le N, x_i < \frac{1}{2}\) 时

\begin{equation}
    h_i = \frac{1}{2} \left(\left(\frac{i}{N}\right)^r - \left(\frac{i-1}{N}\right)^r\right)
    \le \frac{r}{2} \frac{1}{N} \left(\frac{i}{N}\right)^{r-1}  = C h x_i^{(r-1)/r}
\end{equation}

当 \(i > N, x_i \ge \frac{1}{2}\) 时

\begin{equation}
    h_i = \frac{1}{2} \left(\left(\frac{2N-i+1}{N}\right)^r - \left(\frac{2N-i}{N}\right)^r\right)
    \le   \frac{r}{2} \left(\frac{2N-i+1}{N}\right)^{r-1} \frac{1}{N} = C h (1-x_{i-1})^{(r-1)/r}
\end{equation}



\begin{equation}
    R_i = \int_0^1 D(y) \frac{2}{h_i + h_{i+1}}
    ( \frac{1}{h_{i+1}} |x_{i+1}-y|^{1-\alpha}
    - (\frac{1}{h_{i}}+\frac{1}{h_{i+1}}) |x_{i}-y|^{1-\alpha}
    +  \frac{1}{h_{i}}|x_{i-1}-y|^{1-\alpha} )  dy
\end{equation}


我们声明下面的命题并在这一节中证明
\begin{theorem}    \label{thm:un-uniform-truncation-error}
    \begin{equation}
        R_i \le C(h^{r\alpha/2+r}x_i^{-1-\alpha} + h^2 x_i^{-\alpha/2-2/r} + h^2
        \begin{cases}
            |\frac{1}{2}-x_{i-1}|^{1-\alpha}, \quad i\le N \\
            |\frac{1}{2}-x_{i+1}|^{1-\alpha} , \quad N<i\le 2N
        \end{cases})
    \end{equation}
\end{theorem}


为了简单,我们令
\begin{equation}
    D(x) := u(x) - u_h(x)
\end{equation}
\begin{equation}
    T_{ij} = \int_{x_{j-1}}^{x_{j}} D(y) |x_i - y|^{1-\alpha} dy
\end{equation}




\begin{lemma} \label{lmm:Dx1}
    当 \(x\in [0, x_1]\) 时,
    \begin{equation}
        \begin{aligned}
            |D(x)| & = |u(x) - u_h(x)|  \le \int_0^{x_1} |u'(y)| dy \\
            & \le C \int_0^{x_1} y^{\alpha/2-1} dy           \\
            & \le C x_1^{\alpha/2}
        \end{aligned}
    \end{equation}
    同理当 \(x\in [x_{2N-1}, x_{2N}(=1)]\) 时,
    \begin{equation}
        \begin{aligned}
            |D(x)| & = |u(x) - u_h(x)|  \le \int_{x_{2N-1}}^{1} |u'(y)| dy \\
            & \le C \int_{x_{2N-1}}^{1} (1-y)^{\alpha/2-1} dy           \\
            & \le C (1-x_{2N-1})^{\alpha/2} = C x_1^{\alpha/2}
        \end{aligned}
    \end{equation}
\end{lemma}


\begin{lemma} \label{lmm:Dyj}
    % 在此之前我们做一些准备工作。    \\
    对于 \(y\in [x_{j-1}, x_j]\), 我们记 \(y_j^\theta = \theta x_{j-1} + (1-\theta) x_j\),
    则  \\
    1. 当 \(2\le j \le N\) 时 
    \begin{equation}
        \begin{aligned}
            D(y_j^\theta) &= -\frac{\theta (1-\theta)}{2} h_j^2 u''(\eta), \quad \eta\in [x_{j-1}, x_j] \\
            & \le \frac{\theta (1-\theta)}{2} h_j^2 C  (\eta(1-\eta))^{\alpha/2-2} \\
            & \le C h_j^2 \eta^{\alpha/2-2} \\
            & \le C 2^{-r(\alpha/2-2)}h_j^2 (y_j^\theta)^{\alpha/2-2}
            \le C h_j^2 (y_j^\theta)^{\alpha/2-2}
        \end{aligned}
    \end{equation}
    所以存在 \(C=C(\alpha, r)\) 使得
    \begin{equation}
        \frac{D(y_j^\theta)}{h_j^2} \le C (y_j^\theta)^{\alpha/2-2}, \quad 2 \le j \le N
    \end{equation}
    同理
    \begin{equation}
        \frac{D(y_j^\theta)}{h_j^2} \le C (1-y_j^\theta)^{\alpha/2-2}, \quad N \le j \le 2N-1
    \end{equation}
    2. 当 \(2\le j \le N\) 时
    \begin{equation}
        \begin{aligned}
            h_j &\le \frac{r}{2} h x_j^{(r-1)/r} \\
            & \le \frac{r}{2} h 2^{r-1} (y_j^\theta)^{(r-1)/r}
        \end{aligned}
    \end{equation}
    因此我们有
    \begin{equation}
        D(y_j^\theta) \le C h^2 (y_j^\theta)^{\alpha/2-2/r}, \quad 2 \le j \le N
    \end{equation}
    同理
    \begin{equation}
        D(y_j^\theta) \le C h^2 (1-y_j^\theta)^{\alpha/2-2/r}, \quad N \le j \le 2N-1
    \end{equation}
\end{lemma}




\begin{lemma} \label{lmm:Dyj3}
    \begin{equation}
        \begin{aligned}
            D(y_j^\theta) = & -\frac{\theta (1-\theta)}{2} h_j^2 u''(y_j^\theta)
            + \frac{\theta (1-\theta)}{3!} h_j^3 (\theta^2 u'''(\eta_1) - (1-\theta)^2 u'''(\eta_2))
        \end{aligned}
    \end{equation}
    其中 \(\eta_1 \in [x_{j-1}, y_j^\theta], \eta_2 \in [y_j^\theta, x_j], \eta_1 < \eta_2\).
\end{lemma}






\subsection{i=1}


下面讨论 \(R_1\),令
\begin{equation}
    \begin{aligned}
        S_{ij} &= \frac{2}{h_i + h_{i+1}}
        \left( \frac{1}{h_{i+1}} T_{i+1, j}
        - (\frac{1}{h_{i}}+\frac{1}{h_{i+1}}) T_{i,j}
        +  \frac{1}{h_{i}} T_{i-1, j} \right)                     \\
        &= \int_{x_{j-1}}^{x_{j}}
        D(y) \frac{2}{h_1 + h_{2}}
        ( \frac{1}{h_{2}} |x_{2}-y|^{1-\alpha}
        - (\frac{1}{h_{1}}+\frac{1}{h_{2}}) |x_{1}-y|^{1-\alpha}
        +  \frac{1}{h_{1}}|y|^{1-\alpha} )  dy
    \end{aligned}
\end{equation}

\begin{equation}
    \begin{aligned}
        R_1 = \sum_{j=1}^{2N} S_{1j}
    \end{aligned}
\end{equation}

与一致网格时相似,\\
1.

因为 \(1-\alpha > -1\),应用引理 \ref{lmm:Dx1},
\begin{equation}
    \begin{aligned}
        S_{11} & \le 2^{1-r} \int_0^{x_1} \frac{D(y)}{h_1^2}
        ( |x_{2}-y|^{1-\alpha}
        + 2|x_{1}-y|^{1-\alpha}
        +|y|^{1-\alpha} )  dy                                                                                                         \\
            & \le  C x_1^{\alpha/2-2} (x_1h_2^{1-\alpha} + \frac{3}{2-\alpha}x_1^{2-\alpha}) \le C x_1^{-\alpha/2} = C h^{-r\alpha/2}
    \end{aligned}
\end{equation}

2.

\begin{equation}
    \begin{aligned}
        S_{12} & \le 2^{1-r} \int_{x_1}^{x_2} \frac{D(y)}{h_1^2}
        ( |x_{2}-y|^{1-\alpha}
        +2 |x_{1}-y|^{1-\alpha}
        +|y|^{1-\alpha} )  dy                                                                           \\
            & \le C x_1^{\alpha/2-2} h_2^{2-\alpha} \le C x_1^{\alpha/2} x_2^{-\alpha} \le C h^{-r\alpha/2}
    \end{aligned}
\end{equation}

\(S_{1,3}\) 同理。

3.

\begin{equation}
    \begin{aligned}
        I_3 &= \sum_{j=4}^{N} S_{1j}                   \\
        & = \int_{x_3}^{1/2}D(y) \frac{2}{h_1 + h_{2}}
        ( \frac{1}{h_{2}} |y-x_{2}|^{1-\alpha}
        - (\frac{1}{h_{1}}+\frac{1}{h_{2}}) |y-x_{1}|^{1-\alpha}
        +  \frac{1}{h_{1}}|y|^{1-\alpha} )  dy                                         \\
            & \le C \int_{x_3}^{1/2} h^2 y^{\alpha/2-2/r} y^{-1-\alpha} dy \quad by \; \ref{lmm:Dyj} \; \ref{lmm:D2simd2}\\
            & \le C h^2 \int_{x_3}^{1/2} y^{\alpha/2 - 2/r -1 - \alpha} dy             \\
            & \le C h^2 (h^r)^{-2/r-\alpha/2} = C h^{-r\alpha/2}
    \end{aligned}
\end{equation}


4.

\begin{equation}
    \begin{aligned}
        I_4 & = \int_{1/2}^{x_{2N-1}}D(y) \frac{2}{h_1 + h_{2}}
        ( \frac{1}{h_{2}} |x_{2}-y|^{1-\alpha}
        - (\frac{1}{h_{1}}+\frac{1}{h_{2}}) |x_{1}-y|^{1-\alpha}
        +  \frac{1}{h_{1}}|y|^{1-\alpha} )  dy                                                      \\
            & \le C \int_{1/2}^{x_{2N-1}} (1-y)^{\alpha/2-2} (h (1-y)^{(r-1)/r})^2 y^{-1-\alpha} dy \\
            & \le C 2^{1+\alpha} h^2 \int_{1/2}^{x_{2N-1}} (1-y)^{\alpha/2-2 + 2 - 2/r} dy                         \\
            & \le C h^2 (C + h_{2N}^{\alpha/2-2/r+1})                                               \\
            & = C h^2 (C + h^{r\alpha/2-2+r}) \le C h^{\min\{2, r\alpha/2+r\}}
    \end{aligned}
\end{equation}

5.

\begin{equation}
    I_5 \le C h_{2N}^{\alpha/2+1} \le C h^{r\alpha/2+r}
\end{equation}

综合有

\begin{equation}
    R_1 \le C h^{-r\alpha/2}
\end{equation}

\(R_1, R_2, R_3\) 一样。





\newpage

\(R_i, 3<i< N\) 比较困难。





\subsection{i<N/2}

当 \(3<i<N/2\) , 即 \(x_i<(\frac{1}{4})^r\) 时。


\begin{equation}
    \begin{aligned}
        R_i
        = & \sum_{j=1}^{2N} \frac{2}{h_i + h_{i+1}}
        \left( \frac{1}{h_{i+1}} T_{i+1, j}
        - (\frac{1}{h_{i}}+\frac{1}{h_{i+1}}) T_{i,j}
        +  \frac{1}{h_{i}} T_{i-1, j} \right)                     \\
        = & \sum_{j=1}^{i/2} \frac{2}{h_i + h_{i+1}}
        \left( \frac{1}{h_{i+1}} T_{i+1, j}
        - (\frac{1}{h_{i}}+\frac{1}{h_{i+1}}) T_{i,j}
        +  \frac{1}{h_{i}} T_{i-1, j} \right)                     \\
          & + \frac{2}{h_i + h_{i+1}}
        \left( \frac{1}{h_{i+1}} (T_{i+1, i/2+1} +  T_{i+1, i/2+2})
        - (\frac{1}{h_{i}}+\frac{1}{h_{i+1}}) T_{i,i/2+1} \right) \\
          & + \sum_{j=i/2+2}^{2i-1} \frac{2}{h_i + h_{i+1}}
        \left( \frac{1}{h_{i+1}} T_{i+1, j+1}
        - (\frac{1}{h_{i}}+\frac{1}{h_{i+1}}) T_{i,j}
        +  \frac{1}{h_{i}} T_{i-1, j-1} \right)                   \\
        %   & + \sum_{j=i+1}^{N-1} + \sum_{j=N}^{N+1} + \sum_{N+2}^{N+i/2-1}\frac{2}{h_i + h_{i+1}} 
        %   \left( \frac{1}{h_{i+1}} T_{i+1, j+1} 
        %   - (\frac{1}{h_{i}}+\frac{1}{h_{i+1}}) T_{i,j}
        %   +  \frac{1}{h_{i}} T_{i-1, j-1} \right)   \\
        %   & + \frac{2}{h_i + h_{i+1}} 
        %   \left( \frac{1}{h_{i+1}} T_{i+1, N+1} 
        %   - (\frac{1}{h_{i}}+\frac{1}{h_{i+1}}) T_{i,N}
        %   +  \frac{1}{h_{i}} T_{i-1, N-1} \right)   \\
        %   & + \sum_{j=N+1}^{N+i/2-1} \frac{2}{h_i + h_{i+1}} 
        %   \left( \frac{1}{h_{i+1}} T_{i+1, j+1} 
        %   - (\frac{1}{h_{i}}+\frac{1}{h_{i+1}}) T_{i,j}
        %   +  \frac{1}{h_{i}} T_{i-1, j-1} \right)   \\
          & + \frac{2}{h_i + h_{i+1}}
        \left( \frac{1}{h_{i-1}} (T_{i-1, 2i} +  T_{i-1, 2i-1})
        - (\frac{1}{h_{i}}+\frac{1}{h_{i+1}}) T_{i,2i} \right)    \\
          & + \sum_{j=2i+1}^{2N} \frac{2}{h_i + h_{i+1}}
        \left( \frac{1}{h_{i+1}} T_{i+1, j}
        - (\frac{1}{h_{i}}+\frac{1}{h_{i+1}}) T_{i,j}
        +  \frac{1}{h_{i}} T_{i-1, j} \right)                     \\
        = & I_1 + I_2 + I_3 + I_4 + I_5
    \end{aligned}
\end{equation}


\begin{equation}
    \begin{aligned}
        I_1 = & \int_0^{x_1} +\int_{x_1}^{x_{\lceil \frac{i}{2}\rceil}} \\
              & D(y) \frac{2}{h_i + h_{i+1}}
        ( \frac{1}{h_{i+1}} |x_{i+1}-y|^{1-\alpha}
        - (\frac{1}{h_{i}}+\frac{1}{h_{i+1}}) |x_{i}-y|^{1-\alpha}
        +  \frac{1}{h_{i}}|x_{i-1}-y|^{1-\alpha} )  dy
    \end{aligned}
\end{equation}


1.

\begin{equation}
    \begin{aligned}
        J_1 \le C x_1^{\alpha/2+1} x_i^{-1-\alpha} \le C h^{r\alpha/2 +r} x_i^{-1-\alpha}
    \end{aligned}
\end{equation}


2.

\begin{equation}
    \begin{aligned}
        J_2 & \le C \int_{x_1}^{x_{\lceil \frac{i}{2}\rceil}} y^{\alpha/2-2} (h y^{(r-1)/r})^2 |x_i - y|^{-1-\alpha} dy \\
            & \le C h^2 x_i^{-1-\alpha} \int_{x_1}^{x_{\lceil \frac{i}{2}\rceil}} y^{\alpha/2-2/r} dy                   \\
            & \le C h^2 x_i^{-1-\alpha} (h^{r\alpha/2-2+r} + x_i^{\alpha/2-2/r+1})                                      \\
            & = C (h^{r\alpha/2+r} x_i^{-1-\alpha} + h^2x_i^{-\alpha/2-2/r})
    \end{aligned}
\end{equation}
\\

我们先研究 \(I_3\),考虑

\begin{equation}
    \frac{2}{h_i + h_{i+1}}
    \left( \frac{1}{h_{i+1}} T_{i+1, j+1}
    - (\frac{1}{h_{i}}+\frac{1}{h_{i+1}}) T_{i,j}
    +  \frac{1}{h_{i}} T_{i-1, j-1} \right)
\end{equation}





\begin{equation}
    \begin{aligned}
        T_{ij} = & \int_{x_{j-1}}^{x_{j}} D(y) |x_i - y|^{1-\alpha} dy                                                                                                               \\
        =        & \int_0^1 \frac{\theta (1-\theta)}{2} h_j^{3} u''(y_j^\theta) |x_i - y_j^\theta|^{1-\alpha} d\theta                                                                \\
                 & + \int_0^1 \frac{\theta (1-\theta)}{3!} h_j^{4}  |x_i - y_j^\theta|^{1-\alpha} (\theta^2 u'''(\eta_{1,j}^\theta) -  (1-\theta)^2 u'''(\eta_{2,j}^\theta)) d\theta
    \end{aligned}
\end{equation}

现在回到原来的问题,我们要研究

\begin{equation}
    \begin{aligned}
        \frac{2}{h_i + h_{i+1}}
                             & ( \frac{1}{h_{i+1}}  h_{j+1}^{3} u''(y_{j+1}^\theta) |x_{i+1} - y_{j+1}^\theta|^{1-\alpha} \\
        - (\frac{1}{h_{i}} + & \frac{1}{h_{i+1}}) h_j^{3} \; u''(y_j^\theta) |x_i - y_j^\theta|^{1-\alpha}                \\
                             & +  \frac{1}{h_{i}} h_{j-1}^{3} u''(y_{j-1}^\theta) |x_{i-1} - y_{j-1}^\theta|^{1-\alpha} )
    \end{aligned}
\end{equation}

我们希望把他看成一个函数的二阶导,注意到当 \(i/2 \le j \le 2i\) 时

\begin{equation}
    x_i^{1/r} - x_j^{1/r} = x_{i+1}^{1/r} - x_{j+1}^{1/r} = 2^{-1/r}\frac{i-j}{N}
\end{equation}

那么我们将其他的相都表示成 \(x_i\) 的函数。

\begin{equation}
    y_L(x) = (x^{1/r} + z_L)^r , \quad y_R(x) =  (x^{1/r} + z_R)^r
\end{equation}

其中 \(z_L = 2^{-1/r}\frac{j-i-1}{N}, z_R = 2^{-1/r}\frac{j-i}{N}\).

\begin{gather}
    y_R(x_i)=x_j, \quad y_R(x_{i+1}) = x_{j+1}, \quad y_R(x_{i-1}) = x_{j-1} \\
    y_L(x_i) = x_{j-1} ,\quad y_L(x_{i+1}) = x_{j}, \quad y_L(x_{i-1}) = x_{j-2}
\end{gather}

\begin{gather}
    y_\theta(x) = \theta y_L(x) + (1-\theta) y_R(x)      \\
    h_J(x) = y_R(x) - y_L(x)
\end{gather}

那么我么要研究的就是函数

\begin{equation}
    K_1(x) = h_J^3(x) |x - y_\theta(x)|^{1-\alpha} u''(y_\theta(x))
\end{equation}

在网格 \(x_{i-1}, x_i , x_{i+1}\) 的数值二阶差商。
\begin{equation}
    \frac{2}{h_i + h_{i+1}}
    ( \frac{1}{h_{i+1}}  K_1(x_{i+1}) - (\frac{1}{h_{i}} + \frac{1}{h_{i+1}}) K_1(x_{i}) + \frac{1}{h_{i}} K_1(x_{i-1})) = K_1''(\xi), \;\xi\in [x_{i-1}, x_{i+1}]
\end{equation}

由 Leibniz 公式
\begin{equation}
    (uvw)'' = u''vw + uv''w + uvw'' + 2u'v'w + 2uv'w' + 2u'vw'
\end{equation}

由 \(y_R^{1/r} = x^{1/r} + z_R\), 我们得到

\begin{gather}
    \frac{d y_R}{dx} = x^{1/r-1} y_R^{1-1/r}    \\
    \frac{d^2 y_R}{dx^2} = \frac{1-r}{r} x^{1/r-2} y_R^{1-2/r}z_R
\end{gather}

因此

1.

\begin{equation}
    h_J^3 \sim h^3 y_R^{3-3/r} \sim h^3 x^{3-3/r}
\end{equation}
\begin{equation}
    \begin{aligned}
        (h_J^3)' & = 3 h_J^2 (y_R' - y_L ')                       \\
                 & = 3 h_J^2 x^{1/r-1}(y_R^{1-1/r} - y_L^{1-1/r}) \\
                 & \sim h^3 y_R^{2-2/r} x^{1/r-1} y_R^{1-2/r}     \\
                 & \sim h^3 x^{2-3/r}
    \end{aligned}
\end{equation}


\begin{equation}
    \begin{aligned}
        (h_J^3)'' & = 6 h_J x^{2/r-2}(y_R^{1-1/r}-y_L^{1-1/r})^2 + 3 h_J^2 \frac{1-r}{r} x^{1/r-2}(y_R^{1-2/r}z_R-y_L^{1-2/r}z_L)             \\
                  & \sim h y_R^{1-1/r} x^{2/r-2} (hy_R^{1-2/r})^2 + \frac{1-r}{r} h^2 y_R^{2-2/r} x^{1/r-2}(z_R h y_R^{1-3/r} + hy_L^{1-2/r}) \\
                  & \sim h^3 y_R^{3-5/r} x^{2/r-2} + h^3 (y_R^{3-5/r}x^{1/r-2}z_R + y_R^{3-4/r} x^{1/r-2})                                    \\
                  & \sim h^3 ( y_R^{3-5/r} x^{2/r-2} + y_R^{3-4/r} x^{1/r-2} + y_R^{3-5/r}x^{1/r-2}z_R)                                       \\
                  & \sim  h^3 x^{1-3/r}
    \end{aligned}
\end{equation}


2.

由于
\begin{equation}
    x-y_L = (x^{1/r})^{r} - (x^{1/r} + z_L)^r = -z_L \xi^{r-1} \sim z_L x^{(r-1)/r}
\end{equation}

\begin{equation}
    \begin{aligned}
        |x-y_\theta|^{1-\alpha} & = |x-\theta y_L - (1-\theta) y_R |^{1-\alpha} \sim | (\theta z_L + (1-\theta) z_R) \xi^{1-1/r} |^{1-\alpha} \\
                                & \sim z_\theta^{1-\alpha} \xi^{1-\alpha + (\alpha-1)/r}, \quad \xi\in [y_L, x]
    \end{aligned}
\end{equation}

\begin{equation}
    \begin{aligned}
        (|x-y_\theta|^{1-\alpha})' & = \text{sign}(x-y_\theta)(1-\alpha)|x-y_\theta|^{-\alpha}(1-x^{1/r-1}(\theta y_L^{1-1/r} + (1-\theta)y_R^{1-1/r}))  \\
                                   & = (1-\alpha)|x-y_\theta|^{-\alpha}x^{1/r-1} (x^{1-1/r} - (\theta y_L^{1-1/r} + (1-\theta)y_R^{1-1/r}))              \\
                                   & \sim |x-y_\theta|^{-\alpha} x^{1/r-1} (\theta z_L + (1-\theta) z_R) \xi_2^{1-2/r}                                   \\
                                   & \sim | (\theta z_L + (1-\theta) z_R) \xi_1^{1-1/r} |^{-\alpha} x^{1/r-1} (\theta z_L + (1-\theta) z_R)\xi_2^{1-2/r} \\
                                   & \sim z_\theta^{1-\alpha} x^{-\alpha + (\alpha-1)/r}
    \end{aligned}
\end{equation}

\begin{equation}
    \begin{aligned}
        (|x-y_\theta|^{1-\alpha})''
        =    & \alpha(\alpha-1)|x-y_\theta|^{-1-\alpha} (1-x^{1/r-1}(\theta y_L^{1-1/r} + (1-\theta)y_R^{1-1/r}))^2                                     \\
             & - \text{sign}(x-y_\theta)(1-\alpha)|x-y_\theta|^{-\alpha}(\frac{1-r}{r} x^{1/r-2} (\theta y_L^{1-2/r} z_L + (1-\theta) y_R^{1-2/r} z_R)) \\
        \sim & | (\theta z_L + (1-\theta) z_R) \xi_1^{1-1/r} |^{-1-\alpha} x^{2/r-2} \xi_2^{2-4/r} (\theta z_L + (1-\theta) z_R)^2                      \\
             & + | (\theta z_L + (1-\theta) z_R) \xi_1^{1-1/r} |^{-\alpha} x^{1/r-2} (\theta z_L + (1-\theta) z_R)y_R^{1-2/r}                           \\
        \sim & z_\theta^{1-\alpha} x^{-1-\alpha + (\alpha-1)/r}
    \end{aligned}
\end{equation}


3.

\begin{equation}
    u''(y_\theta) \le C y_\theta^{\alpha/2-2} \sim x^{\alpha/2-2}
\end{equation}


\begin{equation}
    \begin{aligned}
        (u''(y_\theta))' & = u'''(y_\theta) x^{1/r-1}(\theta y_L^{1-1/r} + (1-\theta) y_R^{1-1/r}) \\
                         & \le C y_\theta^{\alpha/2-3}x^{1/r-1} y_R^{1-1/r} \sim x^{\alpha/2-3}
    \end{aligned}
\end{equation}

\begin{equation}
    \begin{aligned}
        (u''(y_\theta))'' = & u''''(y_\theta) (x^{1/r-1}(\theta y_L^{1-1/r} + (1-\theta) y_R^{1-1/r}))^2                  \\
                            & + u'''(y_\theta)\frac{1-r}{r} x^{1/r-2}(\theta y_L^{1-2/r}z_L + (1-\theta) y_R^{1-2/r}z_R)  \\
        \sim                & y_\theta^{\alpha/2-4} (x^{1/r-1} y_R^{1-1/r})^2 + z_\theta y_R^{\alpha/2-3 +1-2/r}x^{1/r-2} \\
        <                   & x^{\alpha/2-4}
    \end{aligned}
\end{equation}


\begin{gather}
    u''vw \sim h^3 x^{1-3/r} \; z_\theta^{1-\alpha} x^{1-\alpha + (\alpha-1)/r}\; x^{\alpha/2-2}
    \sim h^3 z_\theta^{1-\alpha} x^{-\alpha/2-2/r+(\alpha-2)/r}   \\
    uv''w \sim h^3 x^{3-3/r} \; z_\theta^{1-\alpha} x^{-1-\alpha + (\alpha-1)/r} \; x^{\alpha/2-2}
    \sim h^3z_\theta^{1-\alpha}x^{-\alpha/2-2/r+(\alpha-2)/r}  \\
    uvw'' \sim h^3 x^{3-3/r} \; z_\theta^{1-\alpha} x^{1-\alpha + (\alpha-1)/r} \; x^{\alpha/2-4}
    \sim h^3z_\theta^{1-\alpha}x^{-\alpha/2-2/r+(\alpha-2)/r}   \\
    u'v'w \sim h^3 x^{2-3/r}\; z_\theta^{1-\alpha} x^{-\alpha + (\alpha-1)/r} \; x^{\alpha/2-2}
    \sim h^3z_\theta^{1-\alpha}x^{-\alpha/2-2/r+(\alpha-2)/r}   \\
    uv'w' \sim h^3 x^{3-3/r} \; z_\theta^{1-\alpha} x^{-\alpha + (\alpha-1)/r} \; x^{\alpha/2-3}
    \sim h^3z_\theta^{1-\alpha}x^{-\alpha/2-2/r+(\alpha-2)/r}   \\
    u'vw' \sim h^3 x^{2-3/r}\;  z_\theta^{1-\alpha} x^{1-\alpha + (\alpha-1)/r} \; x^{\alpha/2-3}
    \sim h^3z_\theta^{1-\alpha}x^{-\alpha/2-2/r+(\alpha-2)/r}
\end{gather}

因此
\begin{equation}
    K_1''(\xi) \sim h^3 z_\theta^{1-\alpha} x_i^{-\alpha/2-2/r+(\alpha-2)/r} , \xi \in [x_{i-1}, x_{i+1}]
\end{equation}

现在我们处理第二部分

\begin{equation}
    \begin{aligned}
        \frac{2}{h_i + h_{i+1}}
                             & ( \frac{1}{h_{i+1}}  h_{j+1}^{4} u'''(\eta_{1,j+1}^\theta) |x_{i+1} - y_{j+1}^\theta|^{1-\alpha} \\
        - (\frac{1}{h_{i}} + & \frac{1}{h_{i+1}}) h_j^{4} \; u'''(\eta_{1,j}^\theta) |x_i - y_j^\theta|^{1-\alpha}              \\
                             & +  \frac{1}{h_{i}} h_{j-1}^{4} u'''(\eta_{1,j-1}^\theta) |x_{i-1} - y_{j-1}^\theta|^{1-\alpha} )
    \end{aligned}
\end{equation}

这次我们只用一阶差分
\begin{equation}
    \begin{aligned}
        \frac{1}{h_{i}} ( h_{j}^{4} u'''(\eta_{1,j}^\theta) |x_{i} - y_{j}^\theta|^{1-\alpha}
        - h_{j-1}^{4} \; u'''(\eta_{1,j-1}^\theta) |x_{i-1} - y_{j-1}^\theta|^{1-\alpha} )
    \end{aligned}
\end{equation}

为了方便计算,我们还是用辅助函数来对上面一项进行估计。

\begin{equation}
    K_2(x) = h_J^4(x)|x-y_\theta(x)|^{1-\alpha}
\end{equation}

\begin{equation}
    \begin{aligned}
        K_2'(x) & = (h_J^4)'|x-y_\theta(x)|^{1-\alpha} + h_J^4 (|x-y_\theta(x)|^{1-\alpha})'                  \\
        % &= 4h_J^3 x^{1/r-1}(y_R^{1-1/r} - y_L^{1-1/r})|x-y_\theta|^{1-\alpha} \\
        %     &\quad + h_J^4 (1-\alpha)|x-y_\theta|^{-\alpha}x^{1/r-1} (x^{1-1/r} - (\theta y_L^{1-1/r} + (1-\theta)y_R^{1-1/r})) \\
                & \sim h^3 x^{3-3/r} x^{1/r-1} h y_R^{1-2/r} \; z_\theta^{1-\alpha} x^{1-\alpha+(\alpha-1)/r} \\
                & \quad + h^4 y_R^{4-4/r} z_\theta^{1-\alpha} x^{-\alpha + (\alpha-1)/r}                      \\
                & \sim h^4 z_\theta^{1-\alpha} x^{4-5/r-\alpha+\alpha/r}
    \end{aligned}
\end{equation}

那么,上面就等于
\begin{equation}
    \begin{aligned}
        \frac{1}{h_i} & (K_2(x_{i}) u'''(\eta_{1,j}^\theta) - K_2(x_{i-1}) u'''(\eta_{1,j-1}^\theta))                                                                                      \\
        =             & \frac{1}{h_i} K_2(x_{i}) ( u'''(\eta_{1,j}^\theta) - u'''(\eta_{1,j-1}^\theta) ) + \frac{1}{h_i} (K_2(x_{i}) - K_2(x_{i-1}) u'''(\eta_{1,j-1}^\theta))             \\
        \le           & h_i^{-1}K_2(x_i) u''''(\eta_{j}^\theta) (x_j-x_{j-2}) + K_2'(\xi)u'''(\eta_{1,j-1}^\theta)    \quad (\eta_j^\theta \in [x_{j-2}, x_{j}], \xi \in [x_{j-1}, x_{j}]) \\
        \sim          & h_i^{-1} h_j^4 |x_i-y_{j}^\theta|^{1-\alpha}\; C (\eta_{j}^\theta)^{\alpha/2-4} 2h_j                                                                               \\
        % & + u'''(\eta_{1,j-1}^\theta)(4h_J^3 \xi^{1/r-1}(y_R^{1-1/r} - y_L^{1-1/r})|\xi-y_\theta|^{1-\alpha} \\
        % &\quad + h_J^4 (1-\alpha)|x-y_\theta|^{-\alpha}x^{1/r-1} (x^{1-1/r} - (\theta y_L^{1-1/r} + (1-\theta)y_R^{1-1/r})))    \\
                      & + h^4 z_\theta^{1-\alpha} \xi^{4-5/r-\alpha+\alpha/r} (\eta_{j}^\theta)^{\alpha/2-3}                                                                               \\
        \sim          & h^4 x_i^{4-4/r}z_\theta^{1-\alpha} x_i^{1-\alpha+(\alpha-1)/r} x_i^{\alpha/2-4}
        + h^4 z_\theta^{1-\alpha} x_i^{4-5/r-\alpha+\alpha/r} x_i^{\alpha/2-3}                                                                                                             \\
        \sim          & hx_i^{1-1/r} \; h^3 z_\theta^{1-\alpha} x_i^{-\alpha/2-2/r+(\alpha-2)/r}
    \end{aligned}
\end{equation}


因此,

\begin{equation}
    \begin{aligned}
        \frac{2}{h_{i}+h_{i+1}} \frac{1}{h_i} & (K_2(x_{i}) u'''(\eta_{1,j}^\theta) - K_2(x_{i-1}) u'''(\eta_{1,j-1}^\theta)) \\
        \sim                                  & h^3 z_\theta^{1-\alpha} x_i^{-\alpha/2-2/r+(\alpha-2)/r}
    \end{aligned}
\end{equation}

最终我们得到,当 \(i/2+2\le j \le 2i-1 < N\) 时,有

\begin{equation}
    \begin{aligned}
        \frac{2}{h_i + h_{i+1}} & \left( \frac{1}{h_{i+1}} T_{i+1, j+1} - (\frac{1}{h_{i}}+\frac{1}{h_{i+1}}) T_{i,j}+  \frac{1}{h_{i}} T_{i-1, j-1} \right) \\
                                & \le C h^3 \left(\frac{|i-j|+1}{N}\right)^{1-\alpha} x_i^{-\alpha/2-2/r+(\alpha-2)/r}
    \end{aligned}
\end{equation}

那么我们得到

\begin{equation}
    \begin{aligned}
        I_3 & \le C \sum_{j=i/2+2}^{2i-1} \left(\frac{1}{N}\right)^3 \left(\frac{|i-j|+1}{N}\right)^{1-\alpha} x_i^{-\alpha/2-2/r+(\alpha-2)/r} \\
            & \le C  \left(\frac{1}{N}\right)^2 x_i^{-\alpha/2-2/r+(\alpha-2)/r} \left(\frac{2i}{N}\right)^{2-\alpha}                           \\
            & \le C \left(\frac{1}{N}\right)^2 x_i^{-\alpha/2-2/r+(\alpha-2)/r} x_i^{(2-\alpha)/r}                                              \\
            & = C \left(\frac{1}{N}\right)^2 x_i^{-\alpha/2-2/r}
    \end{aligned}
\end{equation}
\\


现在我们处理 \(I_2\), 记 \(k = i/2+1\)
\begin{equation}
    \begin{aligned}
        I_2 & = \frac{2}{h_i + h_{i+1}}
        \left( \frac{1}{h_{i+1}} (T_{i+1, i/2+1} +  T_{i+1, i/2+2})
        - (\frac{1}{h_{i}}+\frac{1}{h_{i+1}}) (T_{i,i/2+1}) \right)                                          \\
            & = \frac{2}{h_i + h_{i+1}}
        \left( \frac{1}{h_{i+1}} (T_{i+1, k} -  T_{i, k})
        + \frac{1}{h_{i+1}} (T_{i+1, k+1} - T_{i,k}) + (\frac{1}{h_{i+1}} - \frac{1}{h_{i}}) T_{i,k} \right) \\
            & = J_1 + J_2 + J_3
    \end{aligned}
\end{equation}


\begin{equation}
    \begin{aligned}
        J_1 & = \frac{2}{h_i + h_{i+1}} \left( \frac{1}{h_{i+1}} (T_{i+1, k} -  T_{i, k}) \right)                                \\
            & = \frac{2}{h_i + h_{i+1}} \int_{x_{k-1}}^{x_k} D(y) \frac{|x_{i+1}-y|^{1-\alpha} - |x_i-y|^{1-\alpha}}{h_{i+1}} dy \\
            & \le C x_i^{\alpha/2-2} h_k^2 x_i^{-\alpha}                                                                         \\
            & \le C h^2 x_i^{-\alpha/2-2/r}
    \end{aligned}
\end{equation}

\begin{equation}
    \begin{aligned}
        J_2 & = \frac{2}{h_i + h_{i+1}}\frac{1}{h_{i+1}} \left(T_{i+1, k+1} -T_{i,k} \right)                                                                                                \\
            & = \frac{2}{h_i + h_{i+1}} \int_{0}^1 \frac{h_{k+1}D(y_{k+1}^\theta)|x_{i+1}-y_{k+1}^\theta|^{1-\alpha} - h_k D(y_{k}^\theta)|x_{i}-y_{k}^\theta|^{1-\alpha}}{h_{i+1}} d\theta
    \end{aligned}
\end{equation}

我们看他的两个积分项
\begin{equation}
    \begin{aligned}
         & \frac{K_1(x_{i+1}) - K_1(x_{i})}{h_{i+1}} = K_1'(\xi)                                       \\
         & \quad \sim h^3 x^{2-3/r} \; z_\theta^{1-\alpha} x^{1-\alpha+(\alpha-1)/r} \; x^{\alpha/2-2} \\
         & \qquad + h^3 x^{3-3/r} \; z_\theta^{1-\alpha} x^{-\alpha+(\alpha-1)/r} \; x^{\alpha/2-2}    \\
         & \qquad + h^3 x^{3-3/r} \; z_\theta^{1-\alpha} x^{1-\alpha+(\alpha-1)/r} \; x^{\alpha/2-3}   \\
         & \quad \sim hx^{1-1/r} h^2 z_\theta^{1-\alpha} x^{-\alpha/2+\alpha/r-3/r}                    \\
         & \quad \sim hx^{1-1/r} h^2 x^{(1-\alpha)/r} x^{-\alpha/2+\alpha/r-3/r}                       \\
         & \quad \sim hx^{1-1/r} h^2 x^{-\alpha/2-2/r}                                                 \\
    \end{aligned}
\end{equation}

第二部分研究过了
\begin{equation}
    \begin{aligned}
        \frac{1}{h_i} & (K_2(x_{i+1}) u'''(\eta_{1,k+1}^\theta) - K_2(x_{i}) u'''(\eta_{1,k}^\theta)) \\
        \sim          & hx_i^{1-1/r} \; h^3 z_\theta^{1-\alpha} x_i^{-\alpha/2-2/r+(\alpha-2)/r}      \\
        \sim          & hx_i^{1-1/r} \; h^3 x_i^{(1-\alpha)/r} x_i^{-\alpha/2-2/r+(\alpha-2)/r}       \\
        \sim          & hx_i^{1-1/r} \; h^3 x_i^{-\alpha/2-2/r-1/r}                                   \\
    \end{aligned}
\end{equation}

因此
\begin{equation}
    J_2 \le C h^2 x^{-\alpha/2-2/r}
\end{equation}

现在考虑 \(J_3\)
\begin{equation}
    \begin{aligned}
        J_3 & = \frac{2}{h_i + h_{i+1}}(\frac{1}{h_{i+1}} - \frac{1}{h_{i}}) T_{i,k}                                                            \\
            & = -\frac{2}{h_i + h_{i+1}}\frac{h_{i+1} -h_{i}}{h_{i}h_{i+1}} \int_{x_{k-1}}^{x_k} D(y_k^\theta) |x_i - y_k^\theta|^{1-\alpha} dy \\
            & \sim h_i^{-1}x_i^{-1} h_k^3 x_i^{\alpha/2-2} x_i^{1-\alpha}                                                                       \\
            & \sim h^2 x_i^{-\alpha/2-2/r}
    \end{aligned}
\end{equation}

因此我们有
\begin{equation}
    I_2 \le C h^2 x_i^{-\alpha/2-2/r}
\end{equation}

\(I_4\) 类似。

\newpage

现在考虑 \(I_5\)
\begin{equation}
    \begin{aligned}
        I_5 = & \sum_{j=2i+1}^{N} + \sum_{j=N+1}^{2N-1} + \sum_{2N} \; \frac{2}{h_i + h_{i+1}}
        \left( \frac{1}{h_{i+1}} T_{i+1, j} - (\frac{1}{h_{i}}+\frac{1}{h_{i+1}}) T_{i,j} + \frac{1}{h_{i}} T_{i-1, j} \right) \\
        =     & \int_{x_{2i}}^{x_{N}} +\int_{x_{N}}^{x_{2N-1}} + \int_{x_{2N-1}}^{2N}                                          \\
              & D(y) \frac{2}{h_i + h_{i+1}}
        ( \frac{1}{h_{i+1}} |x_{i+1}-y|^{1-\alpha}
        - (\frac{1}{h_{i}}+\frac{1}{h_{i+1}}) |x_{i}-y|^{1-\alpha}
        +  \frac{1}{h_{i}}|x_{i-1}-y|^{1-\alpha} )  dy                                                                         \\
        =     & J_1 + J_2 + J_3
    \end{aligned}
\end{equation}
\begin{equation}
    \begin{aligned}
        J_1 = & \int_{x_{2i}}^{1/2}
        D(y) \frac{2}{h_i + h_{i+1}}
        ( \frac{1}{h_{i+1}} |x_{i+1}-y|^{1-\alpha}
        - (\frac{1}{h_{i}}+\frac{1}{h_{i+1}}) |x_{i}-y|^{1-\alpha}
        +  \frac{1}{h_{i}}|x_{i-1}-y|^{1-\alpha} )  dy                                     \\
        \le   & C \int_{x_{2i}}^{1/2} y^{\alpha/2-2} (hy^{1-1/r})^2 |y-x_i|^{-1-\alpha} dy \\
        \le   & C \int_{x_{2i}}^{1/2} h^2 y^{\alpha/2-2 + 2-2/r -1-\alpha} dy              \\
        \le   & C h^2 x_i^{-\alpha/2-2/r}
    \end{aligned}
\end{equation}
\begin{equation}
    \begin{aligned}
        J_2 = & \int_{1/2}^{x_{2N-1}}
        D(y) \frac{2}{h_i + h_{i+1}}
        ( \frac{1}{h_{i+1}} |x_{i+1}-y|^{1-\alpha}
        - (\frac{1}{h_{i}}+\frac{1}{h_{i+1}}) |x_{i}-y|^{1-\alpha}
        +  \frac{1}{h_{i}}|x_{i-1}-y|^{1-\alpha} )  dy                                               \\
        \le   & C \int_{1/2}^{x_{2N-1}} (1-y)^{\alpha/2-2} (h(1-y)^{1-1/r})^2 |y-x_i|^{-1-\alpha} dy \\
        \le   & C \int_{1/2}^{x_{2N-1}} h^2 (1-y)^{\alpha/2-2 + 2-2/r} dy                            \\
        \le   & C h^2 (C + h^{r(\alpha/2-2/r+1)})                                                    \\
        \le   & Ch^2 + C h^{r\alpha/2+r}
    \end{aligned}
\end{equation}

\begin{equation}
    J_3 \le C h_{2N}^{\alpha/2+1} \le C h^{r\alpha/2+r}
\end{equation}


全部加起来,我们得到

\begin{equation}
    R_i \le C h^{r\alpha/2+r}x_i^{-1-\alpha} + Ch^2 x_i^{-\alpha/2-2/r}
\end{equation}







% \textcolor{red}{错了!不对称,要补上}
% \begin{equation}
%     R_i \le C h^{r\alpha/2+r} x_i^{-1-\alpha} + h^2 x_i^{-\alpha/2-2/r}
% \end{equation}

% \begin{equation}
%     \begin{aligned}
%         I_4 = & \sum_{j=i+1}^{N-1} \frac{2}{h_i + h_{i+1}} 
%         \left( \frac{1}{h_{i+1}} T_{i+1, j+1} 
%         - (\frac{1}{h_{i}}+\frac{1}{h_{i+1}}) T_{i,j}
%         +  \frac{1}{h_{i}} T_{i-1, j-1} \right)   \\
%     \end{aligned}
% \end{equation}

% 类似的,我们要研究的就变成

% \(y_L^{1/r} = x^{1/r} + z_L, \; y_R^{1/r} = x^{1/r} + z_R, \quad z_L = 2^{-1/r}\frac{j-i-1}{N}, \; z_R =2^{-1/r}\frac{j-i}{N} \)

% \begin{equation}
%     K_1(x) = h_J^3(x) |y_\theta(x)-x|^{1-\alpha} u''(y_\theta(x))
% \end{equation}

% 在网格 \(x_{i-1}, x_i , x_{i+1}\) 的数值二阶差商。
% \begin{equation}
%     \frac{2}{h_i + h_{i+1}} 
%     ( \frac{1}{h_{i+1}}  K_1(x_{i+1}) - (\frac{1}{h_{i}} + \frac{1}{h_{i+1}}) K_1(x_{i}) + \frac{1}{h_{i}} K_1(x_{i-1})) = K_1''(\xi), \;\xi\in [x_{i-1}, x_{i+1}]
% \end{equation}

% 同样的
% \begin{gather}
%     \frac{d y_R}{dx} = x^{1/r-1} y_R^{1-1/r}    \\
%     \frac{d^2 y_R}{dx^2} = \frac{1-r}{r} x^{1/r-2} y_R^{1-2/r}z_R   
% \end{gather}

% 因此

% 1. 

% \begin{equation}
%     h_J = y_R-y_L \sim h y_R^{1-1/r}
% \end{equation}
% \begin{equation}
%     \begin{aligned}
%         (h_J^3)' &= 3 h_J^2 (y_R' - y_L ') \\
%         &= 3 h_J^2 x^{1/r-1}(y_R^{1-1/r} - y_L^{1-1/r}) \\
%         & \sim h_J^2  x^{1/r-1} h y_R^{1-2/r}    \\
%     \end{aligned}
% \end{equation}


% \begin{equation}
%     \begin{aligned}
%         (h_J^3)'' &= 6 h_J x^{2/r-2}(y_R^{1-1/r}-y_L^{1-1/r})^2 + 3 h_J^2 \frac{1-r}{r} x^{1/r-2}(y_R^{1-2/r}z_R-y_L^{1-2/r}z_L)    \\
%         &\sim h_J x^{2/r-2} (hy_R^{1-2/r})^2 + \frac{1-r}{r} h_J^2 x^{1/r-2}(z_R h y_R^{1-3/r} + hy_L^{1-2/r})    
%     \end{aligned}
% \end{equation}


% 2.

% 由于
% \begin{equation}
%     y_L - x =  (x^{1/r} + z_L)^r - (x^{1/r})^{r} = z_L \xi^{r-1}
% \end{equation}

% \begin{equation}
%     \begin{aligned}
%         |y_\theta-x|^{1-\alpha} &= |\theta y_L + (1-\theta) y_R -x|^{1-\alpha}   \\
%         &\sim z_\theta^{1-\alpha} \xi^{1-\alpha + (\alpha-1)/r}, \quad \xi\in [y_L, x]
%     \end{aligned}
% \end{equation}

% \begin{equation}
%     \begin{aligned}
%         (|y_\theta-x|^{1-\alpha})' &= (1-\alpha)|y_\theta-x|^{-\alpha}(x^{1/r-1}( \theta y_L^{1-1/r} + (1-\theta)y_R^{1-1/r})-1 )     \\
%         & = (1-\alpha)|y_\theta-x|^{-\alpha}x^{1/r-1} ( (\theta y_L^{1-1/r} + (1-\theta)y_R^{1-1/r}) - x^{1-1/r}) \\
%         % & \sim |x-y_\theta|^{-\alpha} x^{1/r-1} (\theta z_L + (1-\theta) z_R) \xi_2^{1-2/r} \\
%         % & \sim | (\theta z_L + (1-\theta) z_R) \xi_1^{1-1/r} |^{-\alpha} x^{1/r-1} (\theta z_L + (1-\theta) z_R)\xi_2^{1-2/r} \\
%         % & \sim z_\theta^{1-\alpha} x^{-\alpha + (\alpha-1)/r}
%     \end{aligned}
% \end{equation}

% \begin{equation}
%     \begin{aligned}
%         (|y_\theta-x|^{1-\alpha})'' 
%         = & \alpha(\alpha-1)|y_\theta-x|^{-1-\alpha} (x^{1/r-1}(\theta y_L^{1-1/r} + (1-\theta)y_R^{1-1/r})-1 )^2 \\
%             &+ (1-\alpha)|y_\theta-x|^{-\alpha}(\frac{1-r}{r} x^{1/r-2} (\theta y_L^{1-2/r} z_L + (1-\theta) y_R^{1-2/r} z_R))  \\
%         % \sim & | (\theta z_L + (1-\theta) z_R) \xi_1^{1-1/r} |^{-1-\alpha} x^{2/r-2} \xi_2^{2-4/r} (\theta z_L + (1-\theta) z_R)^2 \\
%         %     &+ | (\theta z_L + (1-\theta) z_R) \xi_1^{1-1/r} |^{-\alpha} x^{1/r-2} (\theta z_L + (1-\theta) z_R)y_R^{1-2/r}   \\
%         % \sim & z_\theta^{1-\alpha} x^{-1-\alpha + (\alpha-1)/r}
%     \end{aligned}
% \end{equation}


% 3.

% \begin{equation}
%     u''(y_\theta) \le C y_\theta^{\alpha/2-2}
% \end{equation}


% \begin{equation}
%     \begin{aligned}
%         (u''(y_\theta))' &= u'''(y_\theta) x^{1/r-1}(\theta y_L^{1-1/r} + (1-\theta) y_R^{1-1/r})   \\
%         & \le C y_\theta^{\alpha/2-3}x^{1/r-1} y_R^{1-1/r}
%     \end{aligned}
% \end{equation}

% \begin{equation}
%     \begin{aligned}
%         (u''(y_\theta))'' =& u''''(y_\theta) (x^{1/r-1}(\theta y_L^{1-1/r} + (1-\theta) y_R^{1-1/r}))^2 \\
%         & + u'''(y_\theta)\frac{1-r}{r} x^{1/r-2}(\theta y_L^{1-2/r}z_L + (1-\theta) y_R^{1-2/r}z_R) \\
%         % \sim & y_\theta^{\alpha/2-4} (x^{1/r-1} y_R^{1-1/r})^2 + z_\theta y_R^{\alpha/2-3 +1-2/r}x^{1/r-2}  \\
%         % < & x^{\alpha/2-4}
%     \end{aligned}
% \end{equation}

\newpage

\subsection{N/2<i<N}

当 \(N/2<i<N\) , 即 \((\frac{1}{4})^r<x_i<\frac{1}{2}\) 时。


\begin{equation}
    \begin{aligned}
        R_i
        = & \sum_{j=1}^{2N} \frac{2}{h_i + h_{i+1}}
        \left( \frac{1}{h_{i+1}} T_{i+1, j}
        - (\frac{1}{h_{i}}+\frac{1}{h_{i+1}}) T_{i,j}
        +  \frac{1}{h_{i}} T_{i-1, j} \right)                     \\
        = & \sum_{j=1}^{i/2} \frac{2}{h_i + h_{i+1}}
        \left( \frac{1}{h_{i+1}} T_{i+1, j}
        - (\frac{1}{h_{i}}+\frac{1}{h_{i+1}}) T_{i,j}
        +  \frac{1}{h_{i}} T_{i-1, j} \right)                     \\
          & + \frac{2}{h_i + h_{i+1}}
        \left( \frac{1}{h_{i+1}} (T_{i+1, i/2+1} +  T_{i+1, i/2+2})
        - (\frac{1}{h_{i}}+\frac{1}{h_{i+1}}) T_{i,i/2+1} \right) \\
          & + \sum_{j=i/2+2}^{3N/2-1} \frac{2}{h_i + h_{i+1}}
        \left( \frac{1}{h_{i+1}} T_{i+1, j+1}
        - (\frac{1}{h_{i}}+\frac{1}{h_{i+1}}) T_{i,j}
        +  \frac{1}{h_{i}} T_{i-1, j-1} \right)                   \\
        %   & + \sum_{j=i+1}^{N-1} + \sum_{j=N}^{N+1} + \sum_{N+2}^{3N/2-1}\frac{2}{h_i + h_{i+1}}
        % \left( \frac{1}{h_{i+1}} T_{i+1, j+1}
        % - (\frac{1}{h_{i}}+\frac{1}{h_{i+1}}) T_{i,j}
        % +  \frac{1}{h_{i}} T_{i-1, j-1} \right)                                                   \\
          & + \frac{2}{h_i + h_{i+1}}
        \left( \frac{1}{h_{i-1}} (T_{i-1, 3N/2} +  T_{i-1, 3N/2-1})
        - (\frac{1}{h_{i}}+\frac{1}{h_{i+1}}) T_{i,3N/2} \right)  \\
          & + \sum_{j=3N/2+1}^{2N} \frac{2}{h_i + h_{i+1}}
        \left( \frac{1}{h_{i+1}} T_{i+1, j}
        - (\frac{1}{h_{i}}+\frac{1}{h_{i+1}}) T_{i,j}
        +  \frac{1}{h_{i}} T_{i-1, j} \right)                     \\
        = & I_1 + I_2 + I_3 + I_4 + I_5
    \end{aligned}
\end{equation}

\(I_1, I_2, I_4, I_5\) 略。

\begin{equation}
    \begin{aligned}
        I_3 = & \sum_{j=i/2+2}^{N-1}  + \sum_{j=N}^{N+1} + \sum_{N+2}^{2N-i} + \sum_{2N-i+1}^{3N/2-1} \\
              & \frac{2}{h_i + h_{i+1}}
        \left( \frac{1}{h_{i+1}} T_{i+1, j+1}
        - (\frac{1}{h_{i}}+\frac{1}{h_{i+1}}) T_{i,j}
        +  \frac{1}{h_{i}} T_{i-1, j-1} \right)                                                       \\
        =     & J_1 + J_2 + J_3 + J_4
    \end{aligned}
\end{equation}

\begin{equation}
    \begin{aligned}
        J_1 & \le C \sum_{j=i/2+2}^{N-1} \left(\frac{1}{N}\right)^3 \left(\frac{|i-j|+1}{N}\right)^{1-\alpha} x_i^{-\alpha/2-2/r+(\alpha-2)/r}                   \\
            & \le C  \left(\frac{1}{N}\right)^2 x_i^{-\alpha/2-2/r+(\alpha-2)/r} (\left(\frac{i}{2N}\right)^{2-\alpha} + \left(\frac{N-i}{N}\right)^{2-\alpha} ) \\
            & \le C \left(\frac{1}{N}\right)^2 x_i^{-\alpha/2-2/r+(\alpha-2)/r} (x_i^{(2-\alpha)/r} + |\frac{1}{2}-x_i|^{2-\alpha} x_i^{(2-\alpha)(1/r-1)} )     \\
            & = C h^2 |\frac{1}{2}-x_i|^{2-\alpha} + Ch^2 \le C h^2
    \end{aligned}
\end{equation}

对于\(J_3,J_4\),由于 \(i<N<j-1\)

\begin{equation}
    x_i^{1/r} + (1-x_j)^{1/r} = 2^{-1/r} (\frac{i}{N} + \frac{2N-j}{N}) = 2^{-1/r} (2-\frac{j-i}{N}) = x_{i-1}^{1/r} + (1-x_{j-1})^{1/r}
\end{equation}

那么令
\begin{equation}
    y_L = 1 - (z_L - x^{1/r})^{r}, \quad y_R = 1 - (z_R - x^{1/r})^{r}
\end{equation}

其中
\begin{equation}
    z_L = 2^{-1/r} (2-\frac{j-i-1}{N}), \quad z_R = 2^{-1/r} (2-\frac{j-i}{N})
\end{equation}

那么类似的
\begin{gather}
    \frac{d y_R}{dx} = x^{1/r-1} (1-y_R)^{1-1/r}    \\
    \frac{d^2 y_R}{dx^2} = \frac{1-r}{r} x^{1/r-2} (1-y_R)^{1-2/r}z_R
\end{gather}

那么我们考虑
\begin{equation}
    K_1(x) = h_J^3 |y_\theta-x|^{1-\alpha} u''(y_\theta)
\end{equation}
的二阶导数

其中 \(\frac{1}{4}< x< \frac{1}{2}, \frac{1}{2} < y_\theta <\frac{3}{4}\).

\newpage
1.

\begin{equation}
    h_J^3 \sim h^3 (1-y_R)^{3-3/r} \sim h^3
\end{equation}
\begin{equation}
    \begin{aligned}
        (h_J^3)' & = 3 h_J^2 (y_R' - y_L ')                               \\
                 & = 3 h_J^2 x^{1/r-1}((1-y_R)^{1-1/r} - (1-y_L)^{1-1/r}) \\
                 & \sim h^3 (1-y_R)^{2-2/r} x^{1/r-1} (1-y_R)^{1-2/r}     \\
                 & \sim h^3
    \end{aligned}
\end{equation}


\begin{equation}
    \begin{aligned}
        (h_J^3)'' & = 6 h_J x^{2/r-2}((1-y_R)^{1-1/r}-(1-y_L)^{1-1/r})^2                             \\
                  & \quad + 3 h_J^2 \frac{1-r}{r} x^{1/r-2}((1-y_R)^{1-2/r}z_R-(1-y_L)^{1-2/r}z_L)   \\
                  & \sim h (1-y_R)^{1-1/r} x^{2/r-2} (h(1-y_R)^{1-2/r})^2                            \\
                  & \quad - h^2 (1-y_R)^{2-2/r} x^{1/r-2}(z_R h (1-y_R)^{1-3/r} + h (1-y_L)^{1-2/r}) \\
                  & \sim h^3
    \end{aligned}
\end{equation}


2.

由于
\begin{equation}
    y-x \sim \begin{cases}
        \frac{1}{2} - x , & 1<y<1-x \\  \frac{1}{2} - y , & y> 1-x
    \end{cases}
\end{equation}

% \begin{equation}
%     \begin{aligned}
%         |y_\theta-x|^{1-\alpha} & = |x-\theta y_L - (1-\theta) y_R |^{1-\alpha} \sim | (\theta z_L + (1-\theta) z_R) \xi^{1-1/r} |^{1-\alpha} \\
%                                 & \sim z_\theta^{1-\alpha} \xi^{1-\alpha + (\alpha-1)/r}, \quad \xi\in [y_L, x]
%     \end{aligned}
% \end{equation}

\begin{equation}
    \begin{aligned}
        (|y_\theta-x|^{1-\alpha})'
         & = (1-\alpha)|y_\theta-x|^{-\alpha}(x^{1/r-1}(\theta (1-y_L)^{1-1/r} + (1-\theta)(1-y_R)^{1-1/r}) - 1)                \\
         & = (1-\alpha)|y_\theta-x|^{-\alpha}x^{1/r-1} (\theta (1-y_L)^{1-1/r} + (1-\theta)(1-y_R)^{1-1/r} - (1-(1-x))^{1-1/r}) \\
         & \sim |y_\theta-x|^{-\alpha} (y_\theta - (1-x))                                                                       \\
    \end{aligned}
\end{equation}

\begin{equation}
    \begin{aligned}
        (|y_\theta-x|^{1-\alpha})''
        =    & \alpha(\alpha-1)|y_\theta-x|^{-1-\alpha} (x^{1/r-1}(\theta (1-y_L)^{1-1/r} + (1-\theta)(1-y_R)^{1-1/r}) - 1)^2            \\
             & - (1-\alpha)|y_\theta-x|^{-\alpha}(\frac{1-r}{r} x^{1/r-2} (\theta (1-y_L)^{1-2/r} z_L + (1-\theta) (1-y_R)^{1-2/r} z_R)) \\
        \sim & |y_\theta-x|^{-1-\alpha} (y_\theta - (1-x))^2 + |y_\theta-x|^{-\alpha}
    \end{aligned}
\end{equation}


3.

\begin{equation}
    u''(y_\theta) \le C (1-y_\theta)^{\alpha/2-2} \sim 1
\end{equation}


\begin{equation}
    \begin{aligned}
        (u''(y_\theta))' & = u'''(y_\theta) x^{1/r-1}(\theta (1-y_L)^{1-1/r} + (1-\theta) (1-y_R)^{1-1/r}) \\
                         & \sim 1
    \end{aligned}
\end{equation}

\begin{equation}
    \begin{aligned}
        (u''(y_\theta))'' =
             & u''''(y_\theta) (x^{1/r-1}(\theta (1-y_L)^{1-1/r} + (1-\theta) (1-y_R)^{1-1/r}))^2                 \\
             & + u'''(y_\theta)\frac{1-r}{r} x^{1/r-2}(\theta (1-y_L)^{1-2/r}z_L + (1-\theta) (1-y_R)^{1-2/r}z_R) \\
        \sim & 1
    \end{aligned}
\end{equation}

那么 \textcolor{red}{大概齐,要补上细节}

\begin{equation}
    \begin{aligned}
        J_3 & \le C \sum_{j=N+2}^{2N-i} h^3 ( |\frac{1}{2}-x_i|^{1-\alpha} +  |\frac{1}{2}-x_i|^{-\alpha} ) \\
            & \le C h^2 |\frac{1}{2}-x_i|^{-\alpha} \int_{1/2}^{1-x_i} 1 \; dy                              \\
            & \le C h^2 |\frac{1}{2}-x_i|^{1-\alpha}
    \end{aligned}
\end{equation}

\begin{equation}
    \begin{aligned}
        J_4 & \le C \int_{1-x_i}^{3/4} h^2 |y-x_i|^{-\alpha} \\
            & \le C h^2 + C h^2 |\frac{1}{2}-x_i|^{1-\alpha}
    \end{aligned}
\end{equation}

综上,
\begin{equation}
    I_3 \le C h^2 |\frac{1}{2}-x_i|^{1-\alpha}
\end{equation}



% \textcolor{red}{暂时证不出来 2024.8.23}

\newpage
\subsection{i=N}

\(h_N = h_{N+1}, \quad x_N = \frac{1}{2}\).

\begin{equation}
    R_N = \frac{1}{h_{N}^2} \int_0^1 D(y)
    (  |x_{N+1}-y|^{1-\alpha} - 2|x_{N}-y|^{1-\alpha} + |x_{N-1}-y|^{1-\alpha} )  dy
\end{equation}


\section{收敛性分析}

\begin{lemma} \label{lmm:D2simd2}
    若 \(g(x)\) 三阶连续可微,那么\\
    1.
    \begin{gather*}
        g(x_{i-1}) = g(x_{i}) - (x_{i}-x_{i-1}) g'(x_{i}) + \frac{(x_{i}-x_{i-1})^2}{2} g''(\xi_1), \quad \xi_1 \in [x_{i-1}, x_{i}]        \\
        g(x_{i+1}) = g(x_{i}) + (x_{i+1}-x_{i}) g'(x_{i}) + \frac{(x_{i+1}-x_{i})^2}{2} g''(\xi_2), \quad \xi_2 \in [x_{i}, x_{i+1}]
    \end{gather*}
    2.
    \begin{equation}
        \frac{2}{h_i + h_{i+1}} \left( \frac{1}{h_{i+1}} g(x_{i+1}) - (\frac{1}{h_{i}}+\frac{1}{h_{i+1}})g(x_{i}) + \frac{1}{h_{i}} g(x_{i-1}) \right) = g''(\xi), \quad \xi \in [x_{i-1}, x_{i+1}]
    \end{equation}
    % 3.
    % \begin{gather*}
    %     g(x_{i+1}) = g(x_{i}) - h_{i} g'(x_{i}) + \frac{h_{i}^2}{2} g''(x_{i}) - \frac{h_{i}^3}{3!}g'''(\xi_1), \quad \xi_1 \in [x_{i-1}, x_{i}]        \\
    %     g(x_{i+1}) = g(x_{i}) + h_{i+1} g'(x_{i}) + \frac{h_{i+1}^2}{2} g''(x_{i}) + \frac{h_{i+1}^3}{3!}g'''(\xi_2), \quad \xi_2 \in [x_{i}, x_{i+1}]  
    % \end{gather*}
    % 4.
    % \begin{equation}
    %     \begin{aligned}
    %         \frac{2}{h_i + h_{i+1}} & \left( \frac{1}{h_{i+1}} g(x_{i+1}) - (\frac{1}{h_{i}}+\frac{1}{h_{i+1}})g(x_{i}) + \frac{1}{h_{i}} g(x_{i-1}) \right) \\
    %         &= g''(x_{i}) + \frac{2}{h_i + h_{i+1}}\left( \frac{h_{i+1}^2}{3!}g'''(\xi_2) - \frac{h_{i}^2}{3!}g'''(\xi_1) \right)
    %     \end{aligned}
    % \end{equation}
\end{lemma}


\begin{lemma}
    \begin{equation}
        \begin{aligned}
            \sum_{j=1}^{2N-1} \tilde{a}_{ij}
             & = \frac{1}{(2-\alpha)(3-\alpha)} \left( \frac{|x_i-x_0|^{3-\alpha} - |x_i-x_1|^{3-\alpha}}{h_1} + \frac{|x_{2N}-x_i|^{3-\alpha} - |x_{2N-1}-x_i|^{3-\alpha}}{h_{2N}} \right)
        \end{aligned}
    \end{equation}
\end{lemma}

令
\begin{equation}
    g(x) = g_{0}(x) + g_{2N}(x)
\end{equation}
其中
\begin{gather*}
    g_{0}(x) := \frac{C_R}{(2-\alpha)(3-\alpha)} \frac{|x_i-x_0|^{3-\alpha} - |x_i-x_1|^{3-\alpha}}{h_1}    \\
    g_{2N}(x) := \frac{C_R}{(2-\alpha)(3-\alpha)} \frac{|x_{2N}-x_i|^{3-\alpha} - |x_{2N-1}-x_i|^{3-\alpha}}{h_{2N}}
\end{gather*}




\begin{lemma}\label{lmm:AisM}
    % \begin{equation}
    %     \begin{aligned}
    %         S_i =\sum_{j=1}^{2N-1}a_{ij} = & \frac{C_R}{(2-\alpha)(3-\alpha)} \frac{2}{h_i + h_{i+1}}                                               \\
    %         & ( \frac{1}{h_{i+1}} \frac{|x_{i+1}-x_0|^{3-\alpha} - |x_{i+1}-x_1|^{3-\alpha}}{x_1 - x_0}              \\
    %         & - (\frac{1}{h_{i}}+\frac{1}{h_{i+1}})\frac{|x_{i}-x_0|^{3-\alpha} - |x_{i}-x_1|^{3-\alpha}}{x_1 - x_0} \\
    %         & +  \frac{1}{h_{i}} \frac{|x_{i-1}-x_0|^{3-\alpha} - |x_{i-1}-x_1|^{3-\alpha}}{x_1 - x_0} )             \\
    %         & + ...                                                                                                  \\
    %         \ge                            & C (x_i^{-\alpha} + (1-x_i)^{-\alpha})
    %     \end{aligned}
    % \end{equation}
    \(A\) 是 \(M\) 矩阵。且
    \begin{equation}
        \begin{aligned}
            S_i := & \sum_{j=1}^{2N-1}a_{ij}                                                                                                                        \\
            =      & \frac{2}{h_i + h_{i+1}} \left( \frac{1}{h_{i+1}} g(x_{i+1}) - (\frac{1}{h_{i}}+\frac{1}{h_{i+1}})g(x_{i}) + \frac{1}{h_{i}} g(x_{i-1}) \right) \\
            \ge    & C (x_i^{-\alpha} + (1-x_i)^{-\alpha})
        \end{aligned}
    \end{equation}
\end{lemma}
\begin{proof}
    事实上,当 \(i\ge 2\) 时,
    \begin{equation}
        \begin{aligned}
            \frac{2}{h_i + h_{i+1}} & \left( \frac{1}{h_{i+1}} g_0(x_{i+1}) - (\frac{1}{h_{i}}+\frac{1}{h_{i+1}})g_0(x_{i}) + \frac{1}{h_{i}} g_0(x_{i-1}) \right) \\
            =                       & g_0''(\xi), \quad \xi \in [x_{i-1}, x_{i+1}]
        \end{aligned}
    \end{equation}
    又有
    \begin{equation}
        \begin{aligned}
            g_0''(\xi) & =  C_R \frac{|\xi-x_0|^{1-\alpha} - |\xi-x_1|^{1-\alpha}}{h_1}                        \\
                       & = -C_R(\alpha-1)|\xi-\eta|^{-\alpha} , \quad \eta\in [x_0, x_1]                       \\
                       & \ge -C_R(\alpha-1) x_{i+1}^{-\alpha}  \ge -C_R(\alpha-1) 2^{-r\alpha} x_{i}^{-\alpha}
        \end{aligned}
    \end{equation}

    当 \(i=1\) 时
    \begin{equation}
        \begin{aligned}
            \frac{2}{h_{1} + h_{2}} & \left( \frac{1}{h_{2}} g_0(x_{2}) - (\frac{1}{h_{1}}+\frac{1}{h_{2}})g_0(x_{1}) + \frac{1}{h_{1}} g_0(x_{0}) \right)                                                  \\
            =                       & \frac{-2C_R}{(2-\alpha)(3-\alpha)} \frac{h_1^{3-\alpha}+h_2^{3-\alpha} + 2h_1^{2-\alpha}h_2 - (h_1+h_2)^{3-\alpha} }{(h_{1} + h_{2})h_1 h_2}                          \\
            =                       & \frac{-2C_R}{(2-\alpha)(3-\alpha)} \frac{h_1^{3-\alpha}+h_2^{3-\alpha} + 2h_1^{2-\alpha}h_2 - (h_1+h_2)^{3-\alpha} }{(h_{1} + h_{2})h_1^{1-\alpha} h_2} h_1^{-\alpha} \\
            =                       & \frac{-2C_R}{(2-\alpha)(3-\alpha)} \frac{1+(2^r-1)^{3-\alpha} + 2(2^r-1) - (2^r)^{3-\alpha} }{2^r (2^r-1)} h_1^{-\alpha}
        \end{aligned}
    \end{equation}
    因为 \(3-\alpha > 1\),我们有 \(1+(2^r-1)^{3-\alpha} \ge (2^r)^{3-\alpha}\),因此
    \begin{equation}
        RHS \ge \frac{-2C_R}{(2-\alpha)(3-\alpha)} 2^{1-r} x_1^{-\alpha}
    \end{equation}
    关于\(g_{2N}(x)\),完全对称,所以存在 \(C=C(\alpha,r)\) 使得
    \begin{equation}
        S_i \ge C (x_i^{-\alpha} + (1-x_i)^{-\alpha})
    \end{equation}
\end{proof}

令
\begin{equation}
    G = \text{diag}(x_1, ..., x_N(=1-x_N), 1-x_{N+1}, ... , 1-x_{2N-1})
\end{equation}

\begin{lemma}\label{lmm:AGhasSingularity}
    矩阵 \(B := AG\) 主对角元为正,其他元为负。且存在 \(C=C(\alpha,r)>0\) 使得
    \begin{equation}
        M_i := \sum_{j=1}^{2N-1} b_{ij}
        \ge -C(x_i^{-\alpha} + (1-x_i)^{-\alpha}) + C\begin{cases}
            |\frac{1}{2} - x_{i-1}|^{1-\alpha} , & i\le N \\
            |x_{i+1} - \frac{1}{2}|^{1-\alpha} , & i\ge N
        \end{cases}
    \end{equation}
\end{lemma}
\begin{proof}
    令
    \begin{equation}
        g(x) = \begin{cases}
            x, & 0<x\le 1/2 \\ 1-x , & 1/2<x<1
        \end{cases}
    \end{equation}
    因为 \(g(x)\) 的线性插值就是他自身 \(g(x)\equiv g_h(x)\),所以有

    \begin{equation}
        \begin{aligned}
            \tilde{M_i} & := \sum_{j=1}^{2N-1} \tilde{b}_{ij}
            = \sum_{j=1}^{2N-1} \tilde{a}_{ij} g(x_j)                                                                                                         \\
                        & = \int_{0}^1 |x_i-y|^{1-\alpha} g(y) dy                                                                                             \\
                        & = -\frac{2}{(2-\alpha)(3-\alpha)}|\frac{1}{2}-x_i|^{3-\alpha} + \frac{1}{(2-\alpha)(3-\alpha)}(x_i^{3-\alpha} + (1-x_i)^{3-\alpha}) \\
                        & := w(x_i) = p(x_i) + q(x_i)
        \end{aligned}
    \end{equation}

    所以
    \begin{equation}
        \begin{aligned}
            M_i & := \sum_{j=1}^{2N-1} a_{ij} g(x_j) \\
                & = C_R\frac{2}{h_i + h_{i+1}}
            \left( \frac{1}{h_{i+1}} w(x_{i+1})
            - (\frac{1}{h_{i}}+\frac{1}{h_{i+1}}) w(x_{i})
            +  \frac{1}{h_{i}} w(x_{i-1}) \right)
        \end{aligned}
    \end{equation}

    其中 \(C_R < 0\).

    特别的,
    \begin{equation}
        \begin{aligned}
            P_N & := C_R \frac{2}{h_N + h_{N+1}}
            \left( \frac{1}{h_{N+1}} p(x_{N+1})
            - (\frac{1}{h_{N}}+\frac{1}{h_{N+1}}) p(x_{N})
            +  \frac{1}{h_{N}} p(x_{N-1}) \right)                                        \\
                & = \frac{-8 C_R}{(2-\alpha)(3-\alpha) h_N^2} h_N^{3-\alpha}             \\
                & = \frac{-8 C_R}{(2-\alpha)(3-\alpha)} (\frac{1}{2}-x_{N-1})^{1-\alpha}
            % \ge \frac{-2 C_R}{(2-\alpha)(3-\alpha)} (\frac{r}{2})^{1-\alpha} h^{1-\alpha}
        \end{aligned}
    \end{equation}

    \begin{equation}
        \begin{aligned}
            P_{N-1} & := \frac{2C_R}{h_{N-1} + h_{N}}
            \left( \frac{1}{h_{N}} p(x_{N})
            - (\frac{1}{h_{N-1}}+\frac{1}{h_{N}}) p(x_{N-1})
            +  \frac{1}{h_{N-1}} p(x_{N-2}) \right)                                                                                           \\
                    & = \frac{-2 C_R}{(2-\alpha)(3-\alpha)} \frac{2}{h_{N-1} + h_{N}}
            \left( - (\frac{1}{h_{N-1}}+\frac{1}{h_{N}}) h_N^{3-\alpha}
            +  \frac{1}{h_{N-1}} (h_{N-1}+h_{N})^{3-\alpha} \right)                                                                           \\
                    & = \frac{-4 C_R}{(2-\alpha)(3-\alpha) h_{N-1}}
            \left( - h_{N}^{2-\alpha}
            +  (h_{N-1}+h_{N})^{2-\alpha} \right)                                                                                             \\
                    & = \frac{-4 C_R}{(3-\alpha)} \xi^{1-\alpha}    \quad \xi \in [h_N, h_{N-1}+h_{N}]                                        \\
                    & \ge \frac{-4 C_R}{(3-\alpha)} (h_{N-1}+h_{N})^{1-\alpha} = \frac{-4 C_R}{(3-\alpha)} (\frac{1}{2} - x_{N-2})^{1-\alpha} \\
            % & \ge -4 C_R r^{1-\alpha} h^{1-\alpha} 
        \end{aligned}
    \end{equation}

    对于 \(1\le i<N-1\),应用引理 \ref{lmm:D2simd2} 得
    \begin{equation}
        \begin{aligned}
            P_{i} & = -2 C_R |\frac{1}{2} - \xi|^{1-\alpha}  \quad \xi \in [x_{i-1}, x_{i+1}] \\
                  & \ge -2 C_R |\frac{1}{2} - x_{i-1}|^{1-\alpha}
        \end{aligned}
    \end{equation}

    而 \(2N-i\) 完全对称,综合起来有
    \begin{equation}
        P_i \ge C\begin{cases}
            |\frac{1}{2} - x_{i-1}|^{1-\alpha} , & i\le N \\
            |x_{i+1} - \frac{1}{2}|^{1-\alpha} , & i\ge N
        \end{cases}
    \end{equation}


    并且对于 \(1<i\le N\), 我们有不等式
    \begin{equation}
        \begin{aligned}
            Q_i & := C_R\frac{2}{h_i + h_{i+1}}
            \left( \frac{1}{h_{i+1}} q(x_{i+1})
            - (\frac{1}{h_{i}}+\frac{1}{h_{i+1}}) q(x_{i})
            +  \frac{1}{h_{i}} q(x_{i-1}) \right)                                    \\
                & = C_R q''(\xi)        \quad \xi \in [x_{i-1}, x_{i+1}]             \\
                & = C_R (\xi^{1-\alpha} + (1-\xi)^{1-\alpha})                        \\
                & \ge C_R (x_{i-1}^{1-\alpha} + (1-x_{i+1})^{1-\alpha})              \\
                & \ge C_R 2^{-r(1-\alpha)} (x_{i}^{1-\alpha} + (1-x_{i})^{1-\alpha})
        \end{aligned}
    \end{equation}

    \begin{equation}
        \begin{aligned}
            {_L}Q_1 & = \frac{C_R}{(2-\alpha)(3-\alpha)} \frac{2}{h_{1} + h_{2}}
            \left( - (\frac{1}{h_{1}}+\frac{1}{h_{2}}) h_{1}^{3-\alpha}
            +  \frac{1}{h_{2}} (h_{1}+h_{2})^{3-\alpha} \right)                                       \\
                    & = \frac{2 C_R}{(2-\alpha)(3-\alpha) h_{2}}
            \left( - h_{1}^{2-\alpha}
            +  (h_{1}+h_{2})^{2-\alpha} \right)                                                       \\
                    & = \frac{2 C_R}{(3-\alpha)} \xi^{1-\alpha}    \quad \xi \in [h_N, h_{N-1}+h_{N}] \\
                    & \ge \frac{2 C_R}{(3-\alpha)} h_1^{1-\alpha}
        \end{aligned}
    \end{equation}

    能得到
    \begin{equation}
        Q_i \ge -C (x_i^{1-\alpha} + (1-x_i)^{1-\alpha})
    \end{equation}
\end{proof}




\begin{theorem}\label{thm:ALGisM}
    存在 \(\lambda>0\) ,以及 \(C=C(\alpha,r) > 0\) ,使得 \(B := A(\lambda I+G)\) 也是 \(M\) 矩阵,且
    \begin{equation}
        M_i := \sum_{j=1}^{2N-1} b_{ij} \ge C(x_i^{-\alpha} + (1-x_i)^{-\alpha}) + C\begin{cases}
            |\frac{1}{2} - x_{i-1}|^{1-\alpha} , & i\le N \\
            |x_{i+1} - \frac{1}{2}|^{1-\alpha} , & i\ge N
        \end{cases}
    \end{equation}
\end{theorem}
\begin{proof}
    由引理 \ref{lmm:AGhasSingularity} 以及定理 \ref{lmm:AisM}  只需令 \(\lambda > 2C_1/C_2\), 则
    \begin{equation}
        \begin{aligned}
            M_i & \ge C_1 \begin{cases}
                              |\frac{1}{2} - x_{i-1}|^{1-\alpha} , & i\le N \\
                              |x_{i+1} - \frac{1}{2}|^{1-\alpha} , & i\ge N
                          \end{cases}
            - C_1 (x_i^{1-\alpha} + (1-x_i)^{1-\alpha}) + \lambda C_2 (x_i^{-\alpha} + (1-x_i)^{-\alpha}) \\
                & \ge C_1 \begin{cases}
                              |\frac{1}{2} - x_{i-1}|^{1-\alpha} , & i\le N \\
                              |x_{i+1} - \frac{1}{2}|^{1-\alpha} , & i\ge N
                          \end{cases}  +  C_1 (x_i^{-\alpha} + (1-x_i)^{-\alpha})
        \end{aligned}
    \end{equation}
\end{proof}


\begin{theorem} \label{thm:un-uniform-convergence}
    那么由定理 \ref{thm:un-uniform-truncation-error} 以及定理 \ref{thm:ALGisM}  我们可以得到
    \begin{equation}
        \max_i |\frac{\epsilon_i}{\lambda+g(x_{i})}| = |\frac{\epsilon_{i_0}}{\lambda+g(x_{i_0})}| \le \frac{|R_{i_0}|}{M_{i_0}} \le C h^{\min\{2,r\alpha/2\}}
    \end{equation}


    从而

    \begin{equation}
        |\epsilon_i| \le C (\lambda+\frac{1}{2}) h^{\min\{2,r\alpha/2\}}
    \end{equation}
\end{theorem}



\end{document}