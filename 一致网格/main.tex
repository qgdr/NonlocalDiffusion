\documentclass{ctexart}
\usepackage{amsmath}
\usepackage{graphicx}

\usepackage{setspace}
\usepackage{xcolor}
\usepackage{listings}
\usepackage[CJKbookmarks=true]{hyperref}	% 添加pdf标签
\usepackage{float}
\usepackage{amsfonts}
\usepackage{mathrsfs}


\begin{document}

\section{问题}

对于 \(\Omega=(0,1)\), \(1<\alpha<2\), 假设 \(f\in C^2(\Omega)\)

\begin{equation}
    \begin{cases}
        (-\Delta)^{\frac{\alpha}{2}} u(x) = f(x),    & x \in \Omega  \\
        u(x) = 0,   & x \in \mathbb{R} \setminus \Omega
    \end{cases}
\end{equation}

其中

\begin{equation}
    (-\Delta)^{\frac{\alpha}{2}} u(x) = -\frac{\partial^\alpha u}{\partial |x|^\alpha} 
    = C_R \frac{d^2}{dx^2} \int_\Omega \frac{u(y)}{|x-y|^{\alpha-1}} dy
\end{equation}


\section{数值格式}

用线性插值代替原函数,中心差分代替二阶导数,记 \(u_h(x)\) 为 \(u(x)\) 在网格点上的线性插值。

我们解这样的数值解

\begin{equation}
    \begin{aligned}
        C_R ( & \frac{2}{h_{i+1} (h_i + h_{i+1})}\int_\Omega \frac{u_h(x)}{ |x_{i+1}-y|^{\alpha-1} } dy \\
                & -\frac{2}{h_i h_{i+1}}\int_\Omega \frac{u_h(x)}{ |x_i-y|^{\alpha-1} } dy \\
                & + \frac{2}{h_i (h_i+ h_{i+1})}\int_\Omega \frac{u_h(x)}{ |x_{i-1}-y|^{\alpha-1} } dy
            ) \\
        = F_i
    \end{aligned}
\end{equation}


矩阵 \(A\) 是 \(M\) 矩阵,主队角正,其他负,严格对角占优。


\section{一致网格}

当 \(r=1\) , 网格成为一致网格, \(x_i = ih, h=\frac{1}{2N}, i=0, ..., 2N\).

\(A\) 等于

\begin{equation}
    \begin{aligned}
    a_{ij} &= \frac{C_R}{(2-\alpha)(3-\alpha)} h^{-\alpha} \\
        & \left( |i-j-2|^{3-\alpha} - 4 |i-j-1|^{3-\alpha} + 6 |i-j|^{3-\alpha} - 4 |i-j+1|^{3-\alpha} +  |i-j+2|^{3-\alpha} \right)
    \end{aligned}
\end{equation}

矩阵行和

\begin{equation}
    \begin{aligned}
        S_i = \sum_{j=1}^{2N-1} a_{ij} & = \frac{C_R}{(2-\alpha)(3-\alpha)} h^{-\alpha} 
            ( |i+1|^{3-\alpha} - 3 |i|^{3-\alpha} + 3 |i-1|^{3-\alpha} - |i-2|^{3-\alpha}  + ... 2N )
    \end{aligned}
\end{equation}


我们得到

\begin{equation}
    S_i \ge  C (x_i^{-\alpha} + (1-x_i)^{-\alpha})
\end{equation}


下面估计截断误差 \(R_i\).

\begin{equation}
    \begin{aligned}
        R_i = & \int_0^1 D(y) \frac{ |y-x_{i-1}|^{1-\alpha} - 2|y-x_i |^{1-\alpha} + |y-x_{i+1}|^{1-\alpha} }{h^2} dy    \\
        % &+ \underbrace{( I(x_{i-1}) -2I(x_{i}) + I({x_{i+1}}) ) / h^2 - f(x_i)}_{O(h^2)}
    \end{aligned}
\end{equation}



目标是

\begin{equation}
    R_i \le C h^{\alpha/2} S_i
\end{equation}

这样我们就有

\begin{equation}
    \epsilon \le \max_i \frac{R_i}{S_i} \le Ch^{\alpha/2}
\end{equation}

考虑 \(R_1\)

\begin{equation}
    \begin{aligned}
        R_1 =& \int_\Omega (u(y) - u_h(y)) \frac{ |y|^{1-\alpha} - 2|y-h|^{1-\alpha} + |y-2h|^{1-\alpha} }{h^2} dy      \\
    \end{aligned}
\end{equation}


我们有

\begin{equation}
    \begin{aligned}
        R_1 = \int_0^{3h} + \int_{3h}^{1/2}
    \end{aligned}
\end{equation}

当 \(y>3h\), 

\begin{equation}
    \frac{ |y|^{1-\alpha} - 2|y-h|^{1-\alpha} + |y-2h|^{1-\alpha} }{h^2} \le C |y|^{-1-\alpha}
\end{equation}

那么

\begin{equation}
    \begin{aligned}
            I_2 &\le  C \int_{3h}^{1/2} |y|^{-1-\alpha} u''(y) h^2 dy   \\
            & \le  C \int_{3h}^{1} |y|^{-1-\alpha} y^{\alpha/2-2} h^2 dy    \\
            & \le C h^{2}  \int_{3h}^{1} y^{-3-\alpha/2} dy  \\
            & \le C h^{2} h^{-2-\alpha/2} = C h^{-\alpha/2}   \\
            & \le C h^{\alpha/2} x_1^{-\alpha} \le C h^{\alpha/2} S_1
    \end{aligned}
\end{equation}

在考虑

\begin{equation}
    \begin{aligned}
        I_1 &= \int_0^{3h}  \frac{ u(y) - u_h(y) }{h^2} (|y|^{1-\alpha} - 2|y-h|^{1-\alpha} + |y-2h|^{1-\alpha}) dy \\
        &= \int_0^h + \int_h^{3h} = J_1 + J_2
    \end{aligned}
\end{equation}

\begin{equation}
    J_2 \le C u''(\eta) h^{2-\alpha} \le C h^{\alpha/2-2} h^{2-\alpha} \le C h^{-\alpha/2}
\end{equation}

因为

\begin{equation}
    \begin{aligned}
        |u(x) - u_h(x)| &\le \int_0^{x_1} |u'(y)| dy    \\
        &\le C \int_0^{x_1} y^{\alpha/2-1} dy   \\
        &\le C x_1^{\alpha/2}    \quad , x\in (0, h)
    \end{aligned}
\end{equation}


\begin{equation}
    \begin{aligned}
        J_1 & = \int_0^h \frac{ u(y) - u_h(y) }{h^2} (|y|^{1-\alpha} - 2|y-h|^{1-\alpha} + |y-2h|^{1-\alpha}) dy \\
        &\le C h^{\alpha/2-2} h^{2-\alpha} = C h^{-\alpha/2}
    \end{aligned}
\end{equation}

所以有

\begin{equation}
    R_1 \le C h^{-\alpha/2} \le C h^{\alpha/2} h^{-\alpha} \le C h^{\alpha/2} S_1, \quad (S_1\ge C x_1^{-\alpha})
\end{equation}


\(R_1, R_2, R_3\) 全部类似。

\subsection{猜想}

\begin{equation}
    R_i \le C h^{\alpha/2+1} (x_i^{-\alpha-1} + (1-x_i)^{-\alpha-1}) \quad (then \le C h^{\alpha/2} S_i)
\end{equation}

为了简便,我们记 \(D(y) := u(y) - u_h(y)\).

当 \(3<i\le N\) 时,

\begin{equation}
    \begin{aligned}
        R_i = &\int_0^1 D(y) \frac{ |y-x_{i-1}|^{1-\alpha} - 2|y-x_i |^{1-\alpha} + |y-x_{i+1}|^{1-\alpha} }{h^2} dy    \\
        = &\int_0^{x_1} D(y) \frac{ |y-x_{i-1}|^{1-\alpha} - 2|y-x_i |^{1-\alpha} + |y-x_{i+1}|^{1-\alpha} }{h^2} dy \\
        & + \int_{x_1}^{x_{\lceil \frac{i}{2}\rceil}} D(y) \frac{ |y-x_{i-1}|^{1-\alpha} - 2|y-x_i |^{1-\alpha} + |y-x_{i+1}|^{1-\alpha} }{h^2} dy \\
        &+ \int_{x_{\lceil \frac{i}{2}\rceil }}^{x_{\lceil \frac{i}{2}\rceil +1 }} \frac{  D(y+h) - D(y) }{h^2} |y-x_i|^{1-\alpha}  + D(y)\frac{ |y-x_{i+1}|^{1-\alpha} - |y-x_{i}|^{1-\alpha} }{h^2} dy    \\
        & + \int_{x_{\lceil \frac{i}{2}\rceil +1}}^{x_{i}} \frac{  D(y-h) -2D(y) +  D(y+h) }{h^2} |y-x_i|^{1-\alpha} dy  \\
        % & + \cdots (2N-i)     \\
        & + \int_{x_{i}}^{x_{N + \lfloor \frac{i}{2}\rfloor -1}} \frac{  D(y-h) -2D(y) +  D(y+h) }{h^2} |y-x_i|^{1-\alpha} dy     \\
        & + \int_{x_{N + \lfloor \frac{i}{2}\rfloor -1}}^{x_{N + \lfloor \frac{i}{2}\rfloor}} \frac{  D(y-h) - D(y) }{h^2} |y-x_i|^{1-\alpha}  + D(y)\frac{ |y-x_{i-1}|^{1-\alpha} - |y-x_{i}|^{1-\alpha} }{h^2} dy   \\
        & +  \int_{x_{N + \lfloor \frac{i}{2}\rfloor}}^{x_{2N-1}} + \int_{x_{2N-1}}^1 D(y) \frac{ |y-x_{i-1}|^{1-\alpha} - 2|y-x_i |^{1-\alpha} + |y-x_{i+1}|^{1-\alpha} }{h^2} dy     \\
        = & I_1 + I_2 + I_3 + I_4 + \cdots
    \end{aligned}
\end{equation}

1. 

\begin{equation}
    \begin{aligned}
        I_1 &= \int_0^{x_1} (u(y) - u_h(y)) \frac{ |y-x_{i-1}|^{1-\alpha} - 2|y-x_i |^{1-\alpha} + |y-x_{i+1}|^{1-\alpha} }{h^2} dy     \\
        &\le C h^{\alpha/2} \int_0^{h} |y-x_i|^{-1-\alpha} dy   \\
        &\le C h^{\alpha/2 + 1} x_{i}^{-1-\alpha}       \\
    \end{aligned}
\end{equation}

% 显然
% \begin{equation}
%     I_1 \le C h^{\alpha/2+1} h^{-1} x_i^{-\alpha} \le  C h^{\alpha/2} S_i
% \end{equation}

% 且当 \(i\) 与 \(N\) 相当时,有 \(h^{\alpha/2+1}\).
% \newline


2.

\begin{equation}
    \begin{aligned}
        I_2 &= \int_{x_1}^{x_{\lceil \frac{i}{2}\rceil}} D(y) \frac{ |y-x_{i-1}|^{1-\alpha} - 2|y-x_i |^{1-\alpha} + |y-x_{i+1}|^{1-\alpha} }{h^2} dy       \\
        &\le C \int_{x_1}^{x_{\lceil \frac{i}{2}\rceil}} y^{\alpha/2 -2} h^2 |x_i - y |^{-1-\alpha} dy  \\
        &\le C h^{\alpha/2-1} h^2 x_i^{-1-\alpha} \le C h^{\alpha/2+1} x_i^{-1-\alpha}
    \end{aligned}
\end{equation}


3. 


\begin{equation}
    \begin{aligned}
        I_3 & = \int_{x_{\lceil \frac{i}{2}\rceil }}^{x_{\lceil \frac{i}{2}\rceil +1 }} \frac{  D(y+h) - D(y) }{h^2} |y-x_i|^{1-\alpha} + D(y)\frac{ |y-x_{i+1}|^{1-\alpha} - |y-x_{i}|^{1-\alpha} }{h^2} dy \\
        &\le \int_{x_{\lceil \frac{i}{2}\rceil }}^{x_{\lceil \frac{i}{2}\rceil +1 }} u'''(\eta_1) h |x_i-y|^{1-\alpha} + u''(\eta_2) h |x_i-y|^{-\alpha}  dy      \\
        &\le C h^2 x_i^{-2-\alpha/2} \le C h^{1+\alpha/2} x_i^{-1-\alpha}
    \end{aligned}
\end{equation}





4.

\begin{equation}
    \begin{aligned}
        I_4 &= \int_{x_{\lceil \frac{i}{2}\rceil +1}}^{x_{i}} \frac{  D(y-h) -2D(y) +  D(y+h) }{h^2} |y-x_i|^{1-\alpha} dy      \\
        &\le \int_{x_{\lceil \frac{i}{2}\rceil +1}}^{x_{i}} u''''(\eta) h^2 |x_i - y|^{1-\alpha} dy \\
        &\le C x_{i}^{\alpha/2 -4} h^2 x_i^{2-\alpha}       \\
        &\le C h^2 x_i^{-2-\alpha/2} \le C h^{1+\alpha/2} x_i^{-1-\alpha}
    \end{aligned}
\end{equation}


猜想证毕,一致网格证完。


\section{非一致}

\(r > 1\), 

\begin{equation}
    \begin{cases}
        x_i = \frac{1}{2} \left(\frac{i}{N}\right)^r   , & 0 \le i \le N  \\
        x_i = 1 - \frac{1}{2} \left(\frac{2N-i}{N}\right)^r, &  N \le i \le 2N
    \end{cases}
\end{equation}

令 \(h=\frac{1}{2N}\) , 那么

当 \(i < N, x_i < \frac{1}{2}\) 时

\begin{equation}
    h_i = \frac{1}{2} \left(\left(\frac{i}{N}\right)^r - \left(\frac{i-1}{N}\right)^r\right) 
        \le C(r) \left(\frac{i}{N}\right)^{r-1} \frac{1}{N} = C h x_i^{(r-1)/r} 
\end{equation}

当 \(i \ge N, x_i \ge \frac{1}{2}\) 时

\begin{equation}
    h_i = \frac{1}{2} \left(\left(\frac{2N-i+1}{N}\right)^r - \left(\frac{2N-i}{N}\right)^r\right) 
        \le C(r) \left(\frac{2N-i+1}{N}\right)^{r-1} \frac{1}{N} = C h (1-x_{i-1})^{(r-1)/r} 
\end{equation}


我们声明

\begin{equation}
    \begin{aligned}
        S_i =\sum_{j=1}^{2N-1}a_{ij} = & \frac{C_R}{(2-\alpha)(3-\alpha)} \frac{2}{h_i + h_{i+1}} \\
        & ( \frac{1}{h_{i+1}} \frac{|x_{i+1}-x_0|^{3-\alpha} - |x_{i+1}-x_1|^{3-\alpha}}{x_1 - x_0} \\ 
        &- (\frac{1}{h_{i}}+\frac{1}{h_{i+1}})\frac{|x_{i}-x_0|^{3-\alpha} - |x_{i}-x_1|^{3-\alpha}}{x_1 - x_0}     \\ 
        &+  \frac{1}{h_{i}} \frac{|x_{i-1}-x_0|^{3-\alpha} - |x_{i-1}-x_1|^{3-\alpha}}{x_1 - x_0} )\\
        \ge & C (x_i^{-\alpha} + (1-x_i)^{-\alpha})
    \end{aligned}
\end{equation}


\begin{equation}
    R_i = \int_0^1 D(y) \frac{2}{h_i + h_{i+1}} 
    ( \frac{1}{h_{i+1}} |x_{i+1}-y|^{1-\alpha} 
    - (\frac{1}{h_{i}}+\frac{1}{h_{i+1}}) |x_{i}-y|^{1-\alpha}
    +  \frac{1}{h_{i}}|x_{i-1}-y|^{1-\alpha} )  dy 
\end{equation}



下面讨论 \(R_1\)

\begin{equation}
    \begin{aligned}
        R_1 =& \int_0^{x_1} + \int_{x_1}^{x_3} + \int_{x_3}^{1/2} + \int_{1/2}^{x_{2N-1}} + \int_{x_{2N-1}}^1  \\
        & D(y) \frac{2}{h_1 + h_{2}} 
        ( \frac{1}{h_{2}} |x_{2}-y|^{1-\alpha} 
        - (\frac{1}{h_{1}}+\frac{1}{h_{2}}) |x_{1}-y|^{1-\alpha}
        +  \frac{1}{h_{1}}|y|^{1-\alpha} )  dy \\
        := & I_1 + I_2 + I_3 + I_4 + I_5
    \end{aligned}
\end{equation}

与一致网格时相似,

1. 

\begin{equation}
    \begin{aligned}
        |u(x) - u_h(x)| &\le \int_0^{x_1} |u'(y)| dy    \\
        &\le C \int_0^{x_1} y^{\alpha/2-1} dy   \\
        &\le C x_1^{\alpha/2}    \quad , x\in (0, x_1)
    \end{aligned}
\end{equation}

因为 \(1-\alpha > -1\)

\begin{equation}
    \begin{aligned}
        I_1 & \le C \int_0^{x_1} \frac{D(y)}{h_1^2} 
        ( |x_{2}-y|^{1-\alpha} 
        +2 |x_{1}-y|^{1-\alpha}
        +|y|^{1-\alpha} )  dy       \\
        &\le C  x_1^{\alpha/2-2} x_1^{2-\alpha} = C x_1^{-\alpha/2} = C h^{-r\alpha/2}
    \end{aligned}
\end{equation}

2.

\begin{equation}
    \begin{aligned}
        I_2 \le C u''(\eta) x_3^{2-\alpha} \le C x_1^{\alpha/2-2} x_3^{2-\alpha} \le C h^{-r\alpha/2}
    \end{aligned}
\end{equation}

3.

\begin{equation}
    \begin{aligned}
        I_3 &= \int_{x_3}^{1/2}D(y) \frac{2}{h_1 + h_{2}} 
        ( \frac{1}{h_{2}} |x_{2}-y|^{1-\alpha} 
        - (\frac{1}{h_{1}}+\frac{1}{h_{2}}) |x_{1}-y|^{1-\alpha}
        +  \frac{1}{h_{1}}|y|^{1-\alpha} )  dy      \\
        &\le C \int_{x_3}^{1/2} y^{\alpha/2-2} (h y^{(r-1)/r})^2 y^{-1-\alpha} dy \\
        &\le C h^2 \int_{x_3}^{1/2} y^{\alpha/2 - 2/r -1 - \alpha} dy       \\
        &\le C h^2 (h^r)^{-2/r-\alpha/2} = C h^{-r\alpha/2}
    \end{aligned}
\end{equation}


4. 

\begin{equation}
    \begin{aligned}
        I_4 &= \int_{1/2}^{x_{2N-1}}D(y) \frac{2}{h_1 + h_{2}} 
        ( \frac{1}{h_{2}} |x_{2}-y|^{1-\alpha} 
        - (\frac{1}{h_{1}}+\frac{1}{h_{2}}) |x_{1}-y|^{1-\alpha}
        +  \frac{1}{h_{1}}|y|^{1-\alpha} )  dy      \\
        &\le C \int_{1/2}^{x_{2N-1}} (1-y)^{\alpha/2-2} (h (1-y)^{(r-1)/r})^2 y^{-1-\alpha} dy  \\
        &\le C h^2 \int_{1/2}^{x_{2N-1}} (1-y)^{\alpha/2-2 + 2 - 2/r}       \\
        &\le C h^2 (C + h_{2N}^{\alpha/2-2/r+1})        \\
        &= C h^2 (C + h^{r\alpha/2-2+r}) \le C h^{\min\{2, r\alpha/2+r\}}
    \end{aligned}
\end{equation}

5.

\begin{equation}
    I_5 \le C h_{2N}^{\alpha/2+1} \le C h^{r\alpha/2+r}
\end{equation}

综合有

\begin{equation}
    R_1 \le C h^{-r\alpha/2}
\end{equation}

\(R_1, R_2, R_3\) 一样。\\








\subsection{一般的 i}



\(R_i, 3<i\le N\) 比较困难。


我们记

\begin{equation}
    T_{ij} = \int_{x_{j-1}}^{x_{j}} D(y) |x_i - y|^{1-\alpha} dy
\end{equation}

那么

\begin{equation}
    \begin{aligned}
        R_i =& \sum_{j=1}^{2N} T_{ij}    \\
        =& \sum_{j=1}^{2N} \frac{2}{h_i + h_{i+1}} 
        \left( \frac{1}{h_{i+1}} T_{i+1, j} 
        - (\frac{1}{h_{i}}+\frac{1}{h_{i+1}}) T_{i,j}
        +  \frac{1}{h_{i}} T_{i-1, j} \right)   \\
        = & \sum_{j=1}^{i/2} \frac{2}{h_i + h_{i+1}} 
        \left( \frac{1}{h_{i+1}} T_{i+1, j} 
        - (\frac{1}{h_{i}}+\frac{1}{h_{i+1}}) T_{i,j}
        +  \frac{1}{h_{i}} T_{i-1, j} \right)   \\
          & + \frac{2}{h_i + h_{i+1}} 
          \left( \frac{1}{h_{i+1}} (T_{i+1, i/2+1} +  T_{i+1, i/2+2})
          - (\frac{1}{h_{i}}+\frac{1}{h_{i+1}}) (T_{i,i/2+1}) \right)   \\
          & + \sum_{j=i/2+2}^{i} \frac{2}{h_i + h_{i+1}} 
          \left( \frac{1}{h_{i+1}} T_{i+1, j+1} 
          - (\frac{1}{h_{i}}+\frac{1}{h_{i+1}}) T_{i,j}
          +  \frac{1}{h_{i}} T_{i-1, j-1} \right)   \\
          & + ...       \\
        = & I_1 + I_2 + I_3 + ...
    \end{aligned}
\end{equation}


\begin{equation}
    \begin{aligned}
        I_1 =& \int_0^{x_1} +\int_{x_1}^{x_{\lceil \frac{i}{2}\rceil}}               \\ 
        & D(y) \frac{2}{h_i + h_{i+1}} 
        ( \frac{1}{h_{i+1}} |x_{i+1}-y|^{1-\alpha} 
        - (\frac{1}{h_{i}}+\frac{1}{h_{i+1}}) |x_{i}-y|^{1-\alpha}
        +  \frac{1}{h_{i}}|x_{i-1}-y|^{1-\alpha} )  dy 
    \end{aligned}
\end{equation}


1. 

\begin{equation}
    \begin{aligned}
        J_1 \le C x_1^{\alpha/2+1} x_i^{-1-\alpha} \le C h^{r\alpha/2 +r} x_i^{-1-\alpha}
    \end{aligned}
\end{equation}


2.

\begin{equation}
    \begin{aligned}
        J_2 &\le C \int_{x_1}^{x_{\lceil \frac{i}{2}\rceil}} y^{\alpha/2-2} (h y^{(r-1)/r})^2 |x_i - y|^{-1-\alpha} dy      \\
        &\le C h^2 x_i^{-1-\alpha} \int_{x_1}^{x_{\lceil \frac{i}{2}\rceil}} y^{\alpha/2-2/r} dy    \\
        &\le C h^2 x_i^{-1-\alpha} (h^{r\alpha/2-2+r} + x_i^{\alpha/2-2/r+1})       \\
    \end{aligned}
\end{equation}
\\

我们先研究 \(I_3\),考虑

\begin{equation}
    \frac{2}{h_i + h_{i+1}} 
    \left( \frac{1}{h_{i+1}} T_{i+1, j+1} 
    - (\frac{1}{h_{i}}+\frac{1}{h_{i+1}}) T_{i,j}
    +  \frac{1}{h_{i}} T_{i-1, j-1} \right)
\end{equation}

在此之前我们做一些准备工作。    \\
对于 \(y\in [x_{j-1}, x_j]\), 我们记 \(y_j^\theta = \theta x_{j-1} + (1-\theta) x_j\)

\begin{equation}
    \begin{aligned}
        D(y_j^\theta) =& \frac{\theta (1-\theta)}{2} h_j^2 u''(y_j^\theta) + \frac{\theta (1-\theta)(1-2\theta)}{3!} h_j^3 u'''(y_j^\theta) \\
        &+ \frac{\theta (1-\theta)}{4!} h_j^4 (\theta^3 u'''(\eta_1) + (1-\theta)^3 u'''(\eta_2))
    \end{aligned}
\end{equation}

那么



\begin{equation}
    \begin{aligned}
        T_{ij} &= \int_{x_{j-1}}^{x_{j}} D(y) |x_i - y|^{1-\alpha} dy   \\
        & = \int_0^1 \frac{\theta (1-\theta)}{2} h_j^{3} u''(y_j^\theta) |x_i - y_j^\theta|^{1-\alpha} d\theta + ...
    \end{aligned}
\end{equation}

现在回到原来的问题,我们要研究

\begin{equation}
    \begin{aligned}
        \frac{2}{h_i + h_{i+1}} 
        & ( \frac{1}{h_{i+1}}  h_{j+1}^{3} u''(y_{j+1}^\theta) |x_{i+1} - y_{j+1}^\theta|^{1-\alpha} \\
        - (\frac{1}{h_{i}} +& \frac{1}{h_{i+1}}) h_j^{3} \; u''(y_j^\theta) |x_i - y_j^\theta|^{1-\alpha}    \\
        &+  \frac{1}{h_{i}} h_{j-1}^{3} u''(y_{j-1}^\theta) |x_{i-1} - y_{j-1}^\theta|^{1-\alpha} )
    \end{aligned}
\end{equation}

我们希望把他看成一个函数的二阶导,注意到当 \(j\le i\le N\) 时

\begin{equation}
    x_i^{1/r} - x_j^{1/r} = x_{i+1}^{1/r} - x_{j+1}^{1/r} = 2^{-1/r}\frac{i-j}{N}
\end{equation}

那么我们将其他的相都表示成 \(x_i\) 的函数。

\begin{equation}
        y_L = (x^{1/r} - z_1)^r , \quad y_R =  (x^{1/r} - z)^r
\end{equation}

其中 \(z = 2^{-1/r}\frac{i-j}{N}, z_1 = 2^{-1/r}\frac{i-j+1}{N}\)

\begin{gather}
    y_\theta = \theta y_L + (1-\theta) y_R      \\
    h_J = y_R - y_L
\end{gather}

那么我么要研究的就是函数

\begin{equation}
    h_J^3 |x - y_\theta|^{1-\alpha} u''(y^\theta)
\end{equation}

在网格 \(x_{i-1}, x_i , x_{i+1}\) 的数值二阶差商。\\

由 Leibniz 公式 
\begin{equation}
    (uvw)'' = u''vw + uv''w + uvw'' + 2u'v'w + 2uv'w' + 2u'vw'
\end{equation} 

由 \(y_R^{1/r} = x^{1/r} - z\), 我们得到

\begin{gather}
    \frac{d y_R}{dx} = x^{1/r-1} y_R^{1-1/r}    \\
    \frac{d^2 y_R}{dx^2} = \frac{r-1}{r} x^{1/r-2} y_R^{1-2/r}z    \\
\end{gather}








\end{document}